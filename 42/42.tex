\documentclass{article}
\usepackage{graphicx} % Required for inserting images
\usepackage{amsmath,amssymb,amsthm}
\usepackage{physics}
\usepackage{graphicx,float}
\graphicspath{{images/}}
\usepackage[none]{hyphenat}
\usepackage{blindtext}
\usepackage{parskip}
\usepackage[letterpaper,top=3cm, left= 3cm,bottom=3cm]{geometry}
\usepackage{subcaption}
\usepackage{derivative}

\title{42}
\author{Polaris}

\begin{document}

\maketitle

\begin{equation}
    1 + 1 = 2
\end{equation}

\begin{equation}
    \frac{\sin \alpha}{a} = \frac{\sin \beta}{b} = \frac{\sin \gamma}{c}
\end{equation}

\begin{equation}
    c^2 = a^2 + b^2 - 2ab\cos \gamma
\end{equation}

\begin{equation}
    \log xy = \log x + \log y
\end{equation}

\begin{equation}
    e = \lim_{n\to \infty} \left(1+\frac{1}{n}\right)^n
\end{equation}

\begin{equation}
    \dfrac{d}{dx}f(x) = \lim_{h\to 0}\frac{f(x+h) - f(x)}{h}
\end{equation}

\begin{equation}
    \int_{a}^{b} f(x) \mathrm{d}x = F(b) - F(a)
\end{equation}

\begin{equation}
    i^2 = -1
\end{equation}

\begin{equation}
    e^{i x}  = \cos x + i\sin x
\end{equation}

\begin{equation}
    r = \frac{p}{1 + e \cos \theta}
\end{equation}

\begin{equation}
    f(x) = \sum_{n=0}^{\infty} \frac{f^{(n)}(a)}{n!}(x-a)^n
\end{equation}

\begin{equation}
    \widehat{f}(\xi) = \int_{-\infty}^{\infty} f(x) e^{2\pi i x \xi} \mathrm{d}x
\end{equation}

\begin{equation}
    F(s) = \int_{0}^{\infty} f(t) e^{-st} \mathrm{d}t
\end{equation}

\begin{equation}
    \int_{\partial \Omega} \omega = \int_\Omega \mathrm{d} \omega
\end{equation}

\begin{equation}
    \oint_C (L\, dx + M\, dy) = \iint_{D} \left(\frac{\partial M}{\partial x} - \frac{\partial L}{\partial y}\right) dA
\end{equation}

\begin{equation}
    f(x) = \frac{1}{\sqrt{2\pi\sigma^2}}e^{-\frac{(x-\mu)^2}{2\sigma^2}}
\end{equation}

\begin{equation}
    H = -\sum_{x}^{}p(x)\log p(x)
\end{equation}

\begin{equation}
    \mathbf{F} = \frac{GMm}{r^3}\mathbf{r}
\end{equation}

\begin{equation}
    \mathbf{F} =\frac{d}{dt}\mathbf{p} = m\mathbf{a}
\end{equation}

\begin{equation}
    E_{mech} = K + U
\end{equation}

\begin{equation}
    \mathbf{L} = \frac{d}{dt}\mathrm{\mathbf{\tau}} = \mathbf{r} \cross \mathbf{p} = I\omega
\end{equation}

\begin{equation}
    F_{con} = -\frac{\partial U}{\partial x}
\end{equation}

\begin{equation}
    \nabla^2 \Psi = 4\pi G\rho
\end{equation}

\begin{equation}
    \nabla \cdot \mathbf{E} = \frac{\rho}{\epsilon_0}\\
\end{equation}

\begin{equation}
    \nabla \cdot \mathbf{B} = 0\\
\end{equation}

\begin{equation}
    \nabla \times \mathbf{E} = -\pdv{\mathbf{B}}{t}\\
\end{equation}

\begin{equation}
    \nabla \times \mathbf{B} = \mu_0\left(\mathbf{J} + \epsilon \pdv{\mathbf{E}}{t}\right)
\end{equation}

\begin{equation}
    p + \frac{1}{2}\rho v^2 + \rho g h = C
\end{equation}

\begin{equation}
    pV = nRT
\end{equation}

\begin{equation}
    \mathrm{d}E_{int} = \delta Q - \delta W
\end{equation}

\begin{equation}
    \oint \frac{\delta Q}{T} \leq 0
\end{equation}

\begin{equation}
    \frac{\partial^2 u}{\partial t^2} = c^2 \frac{\partial^2 u}{\partial x^2}
\end{equation}

\begin{equation}
    \rho \left[\frac{\partial \mathbf{v}}{\partial t} + \mathbf{v} \cdot \nabla \mathbf{v}\right] = -\nabla p + \mu\nabla^2 \mathbf{v}^2 + \mathbf{f}
\end{equation}

\begin{equation}
    \frac{1}{\mu} + \frac{1}{\nu} = \frac{1}{f}
\end{equation}

\begin{equation}
    m_1 - m_2 = -2.5 \lg \left(\frac{F_1}{F_2}\right)
\end{equation}

\begin{equation}
    f(v) = 4\pi v^2 \left(\frac{m}{2\pi kT}\right)^{3/2} \exp{-\frac{mv^2}{2kT}}
\end{equation}

\begin{equation}
    B_\nu (T) = \frac{2h\nu^3}{c^2}\frac{1}{\exp{\frac{h\nu}{kT}}-1}
\end{equation}

\begin{equation}
    E = \gamma mc^2
\end{equation}

\begin{equation}
    \Delta x \Delta p \geq \frac{\hbar}{2}
\end{equation}

\begin{equation}
    i\hbar \frac{\partial}{\partial t}\Psi (x,t) = \left[-\frac{\hbar^2}{2m}\frac{\partial^2}{\partial x^2} + V(x,t)\right]\Psi(x,t)
\end{equation}

\begin{equation}
    G_{\mu \nu} = -\frac{8\pi G}{c^4} T_{\mu \nu}
\end{equation}

\begin{equation}
    H^2 = \frac{8\pi G}{3c^2}\epsilon - \frac{\kappa c^2}{R_0^2 a^2} + \frac{\Lambda}{3}
\end{equation}

\end{document}