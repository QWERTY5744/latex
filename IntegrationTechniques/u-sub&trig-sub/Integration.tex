\documentclass{article}
\usepackage{graphicx} % Required for inserting images
\usepackage{amsmath,amssymb,amsthm}
\usepackage{physics}
\usepackage{graphicx,float}
\graphicspath{{images/}}
\usepackage[none]{hyphenat}
\usepackage{blindtext}
\usepackage{parskip}
\usepackage[letterpaper,top=3cm, left= 3cm,bottom=3cm]{geometry}
\usepackage{subcaption}
\numberwithin{equation}{section}

\title{Integration using subsitution}
\author{assassin3552}
\date{2025/03/26}

\begin{document}
\maketitle

\section{U-subsitution}
\subsection{Indefinite integral}
U-subsutition is the first techniques we will learn, let's first look at an example question:

Evaluate
\[
\int 2x e^{-x^2} \mathrm{d}x
\]

To solve this integral: Let $u = -x^2$, then $\mathrm{d}u = -2x \mathrm{d}x$, let's first substitute $u$ back to the integral:

\[
\int 2x e^u \mathrm{d}x
\]

At first glance it seems that we make this integral more complicated as we introduce 2 variables in,
but if we examine the integral carefully, we notice that we already have $2x\mathrm{d}x$ present in the integrand, we just need a minus sign.
So if we add the minus sign like this, we can substitute $\mathrm{d}u$ in and evaluate the integral:

\[
\int - e^u(-2x \mathrm{d}x) = \int -e^u \mathrm{d}u = -e^u+C
\]

All that is left is to do is substitute $u=-x^2$ back and we get the final result:

\[
\int 2xe^{-x^2} \mathrm{d}x = -e^{-x^2}+C
\]

If we differentiate our results, we will arrive at the integrand, meaning that our process is correct.

This is essentially U-substitution, our thought process can be summarized as follow:
\begin{enumerate}
    \item Let something equals to $u$
    \item Calculate $\mathrm{d}u = \text{some expression} \cdot \mathrm{d}x$
    \item Manipulate the integral so we found the $\text{some expression} \cdot \mathrm{d}x$
    \item Substitute $\mathrm{d}u$ and evaluate the integral
    \item Replace $u$ with $x$ and finish the integral
\end{enumerate}

\newpage
Let's look at another example:
\[
\int 2x \sin(x^2) \mathrm{d}x
\]
\end{document}