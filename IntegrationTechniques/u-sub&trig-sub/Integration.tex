\documentclass{article}
\usepackage{graphicx} % Required for inserting images
\usepackage{amsmath,amssymb,amsthm}
\usepackage{physics}
\usepackage{graphicx,float}
\graphicspath{{images/}}
\usepackage[none]{hyphenat}
\usepackage{blindtext}
\usepackage{parskip}
\usepackage[letterpaper,top=3cm, left= 3cm,bottom=3cm]{geometry}
\usepackage{subcaption}
\numberwithin{equation}{section}

\title{Integration using Subsitution}
\author{assassin3552}
\date{2025/03/26}

\begin{document}
\maketitle

\section{Indefinite integral}
U-subsutition is the first techniques we will learn, let's first look at an example question:

Evaluate
\[
\int 2x e^{-x^2} \mathrm{d}x
\]

To solve this integral: Let $u = -x^2$, then $\mathrm{d}u = -2x \mathrm{d}x$, let's first substitute $u$ back to the integral:

\[
\int 2x e^u \mathrm{d}x
\]

At first glance it seems that we make this integral more complicated as we introduce 2 variables in,
but if we examine the integral carefully, we notice that we already have $2x\mathrm{d}x$ present in the integrand, we just need a minus sign.
So if we add the minus sign like this, we can substitute $\mathrm{d}u$ in and evaluate the integral:

\[
\int - e^u(-2x \mathrm{d}x) = \int -e^u \mathrm{d}u = -e^u+C
\]

All that is left is to do is substitute $u=-x^2$ back and we get the final result:

\[
\int 2xe^{-x^2} \mathrm{d}x = -e^{-x^2}+C
\]

If we differentiate our results, we will arrive at the integrand, meaning that our process is correct.

This is essentially U-substitution, our thought process can be summarized as follow:
\begin{enumerate}
    \item Let something equals to $u$
    \item Calculate $\mathrm{d}u = \text{some expression} \cdot \mathrm{d}x$
    \item Manipulate the integral so we found the $\text{some expression} \cdot \mathrm{d}x$
    \item Substitute $\mathrm{d}u$ and evaluate the integral
    \item Replace $u$ with $x$ and finish the integral
\end{enumerate}

\newpage
Let's look at another example:
\[
\int 2x \sin(x^2) \mathrm{d}x
\]

We can complete this integral with u-substitution, let $u=x^2$, $\mathrm{d}u = 2x \mathrm{d}x$, thus $x = \sqrt{u}$ and $\mathrm{d}x = \frac{1}{2x}\mathrm{d}u$, the integral turns to:
\[
\int 2\sqrt{u} \sin(u) \frac{1}{2\sqrt{u}}\mathrm{d}u = -\cos(u) + C
\]
Substitute $u = x^2$ back to the integral:
\[
\int 2x \sin(x^2)\mathrm{d}x = -\cos(x^2) + C
\]

Another example would be:
\[
\int \frac{x}{x^2+8}\mathrm{d}x
\]
This looks tricky, but we do see a $x^2$ term and $x$ term, and the derivative of $x^2$ is $2x$.
Let $u = x^2+8$, $\mathrm{d}u = 2x \mathrm{d}x$, thus $x = \sqrt{u - 8}$, $\mathrm{d}x = 1/2x \mathrm{d}u$:
\[
\int \frac{\sqrt{u - 8}}{u} \frac{1}{2\sqrt{u - 8}}\mathrm{d}u\int \frac{1}{2u} \mathrm{d}u = \frac{1}{2} \ln{\abs{u}} + C = \frac{1}{2} \ln{\abs{x^2+8}} + C
\]

\section{Definite Integral}
Definite integral can be treated the same as indefinite integral, however we need to account for the upper and lower bound.

Let's take a look at an example:
\[
\int_{0}^{2} \frac{x}{x^2+8}\mathrm{d}x
\]
In our previous example, we let $u(x) = x^2+8$, we introduce a new function to simplify the integrand,
so we also need to change the upper and lower bound of integration to match the newly created integrand.

Since $u(0) = 8$ and $u(2) = 12$, we substitute this into the integration bound:

\[
\int_{0}^{2} \frac{x}{x^2+8}\mathrm{d}x = \int_{8}^{12} \frac{1}{2u}\mathrm{d}u = \frac{1}{2} (\ln{12}-\ln{8}) = \ln{\frac{\sqrt{6}}{2}}
\]

Thought process for definite integral would be:
\begin{enumerate}
    \item Let something equals to $u$
    \item Calculate $\mathrm{d}u = \text{some expression} \cdot \mathrm{d}x$
    \item Replace the upper and lower bound with $u(a)$ and $u(b)$, where $a$ and $b$ are old integration bounds
    \item Manipulate the integral so we found the $\text{some expression} \cdot \mathrm{d}x$
    \item Substitute $\mathrm{d}u$ and evaluate the integral
\end{enumerate}
\end{document}