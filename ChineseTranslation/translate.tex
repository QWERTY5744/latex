\documentclass{ctexart}
\title{当代天体物理学导论答案翻译}
\author{勾陈一}
\usepackage{graphicx} % Required for inserting images
\usepackage{amsmath,amssymb,amsthm}
\usepackage{graphicx,float}
\usepackage{physics}
\graphicspath{{images/}}
\usepackage[none]{hyphenat}
\usepackage{blindtext}
\usepackage{parskip}
\usepackage[letterpaper,top=3cm, left= 3cm,bottom=3cm]{geometry}
\numberwithin{equation}{section}

\begin{document}
\maketitle
\newpage
\tableofcontents

\newpage

\section{第一章:天球}
上网好不容易找到的答案是拿英文写的,因此想翻译成中文。(本翻译纯属个人工作,可以免费传播)

\subsection{推导行星的会合周期与其公转周期的关系}
在一段周期($S$)内,地球绕太阳公转了$\frac{S}{P_e}$圈,某一行星绕太阳公转了$\frac{S}{P}$,当地球比该行星多转一圈时,它们再次回合(初中数学的追及问题),因此:
\begin{equation}
    \frac{S}{P_e} = \frac{S}{P} + 1
\end{equation}
由此可以解出:
\begin{equation}
    \frac{1}{S} = \frac{1}{P_e} - \frac{1}{P}
\end{equation}
上式是对于外行星的,采用类似的思路可以证明内行星的公式:
\begin{equation}
    \frac{1}{S} = \frac{1}{P} - \frac{1}{P_e}
\end{equation}

\subsection{设计推算行星轨道半径的方法}
首先要明确,这道题不能用开普勒定律。我们讨论内行星和外行星两种情况:
\subsubsection{内行星}
取内行星和太阳大距的时刻,此时会产生一个直角三角形。
\begin{figure}[H]
    \centering
    \includegraphics[width=8cm]{pics/Elongation.jpg}
    \caption{大距}
    \label{fig:elongation}
\end{figure}
可以发现,标蓝的角是可以被量出来的,记这个角为$\alpha$,那么根据余弦函数,有:
\begin{equation}
    \cos \alpha = \frac{b}{c}
\end{equation}
再定义$c = 1$的话(相当于定义一个天文单位),即可算出$b$的长度。

\subsubsection{外行星}
对于外行星,取其在方照的时候求:
\begin{figure}[H]
    \centering
    \includegraphics[width=8cm]{pics/quadrature.jpg}
    \caption{方照}
    \label{fig:quadrature}
\end{figure}
这个标蓝的角是可以被量出来的,记它为$\beta$,则根据余弦函数,有:
\begin{equation}
    \cos \beta = \frac{b}{c}
\end{equation}
和之前几乎一样,这次$b$定义为一天文单位,$c$定义为外行星的轨道半径。

\subsection{求金星火星的公转周期}
\subsubsection{金星公转周期}
对会合周期的式子稍微变下形:
\begin{equation}
    P_v = \frac{S\cdot P_e}{S + P_e}
\end{equation}
带入数值,可以求出$P_v = 244.695$天
\subsubsection{火星的公转周期}
对会合周期的式子稍微变下形:
\begin{equation}
    P_m = \frac{S\cdot P_e}{S - P_e}
\end{equation}
带入数值,可以求出$P_e = 686.985$天
\subsubsection{哪个外行星的会合周期最短?}
对于外行星,会合周期是:
\begin{equation}
    \frac{1}{S} = \frac{1}{P_e} - \frac{1}{P}
\end{equation}
变下形:
\begin{equation}
    S = \frac{P\cdot P_e}{P - P_e}
\end{equation}
我们发现,当$P_e$很大的时候,$P - P_e \approx P$, 因此会合周期趋向于$P_e$,因此,行星的公转周期越大,其回合周期就越短,太阳系内,是海王星(冥王星与2008年被除名行星身份)
\subsection{太阳赤道坐标}
春分:$\alpha = 0^h, \delta = 0^{\circ}$\\
夏至:$\alpha = 6^h, \delta = 23.5^{\circ}$\\
秋分:$\alpha = 12^h, \delta = 0^{\circ}$\\
冬至:$\alpha = 18^h, \delta = -23.5^{\circ}$

\newpage
\section{第二章:天体力学}
\subsection{椭圆的标准方程}
\subsection{椭圆的面积}
\subsection{行星速度分量}
\subsection{机械能总和}
二体系统的机械能总和为
\[
E = \frac{1}{2} m_1 \abs{\mathbf{v}_1}^2 + \frac{1}{2} m_2 \abs{\mathbf{v}_2}^2 - \frac{Gm_1m_2}{r}
\]
根据质心系的定义有$\mathbf{r_1} = -\dfrac{\mu}{m_1} \mathbf{r}$和$\mathbf{r_2} = \dfrac{\mu}{m_2} \mathbf{r}$,求导并带入上式
\[
E = \frac{1}{2} m_1 \frac{\mu^2}{m_1^2} v^2 + \frac{1}{2} m_2 \frac{\mu^2}{m_1^2} v^2 - \frac{GM\mu}{r}
\]
注意到$\dot{\mathbf{r}} = \mathbf{v}$且$M\mu = m_1 m_2$,再次化简,就可以写出
\[
E = \frac{1}{2}\mu v^2 - \frac{GM\mu}{r}
\]
\subsection{角动量总和}
\subsection{太阳-木星系统角动量}
\subsection{逃逸速度}
\subsection{地球同步轨道}
\subsection{引力势能的平均}
根据平均的定义,有
\[
\bar{U} = -\frac{1}{T}\int_0^T \frac{GM\mu}{r(t)}dt
\]
同时注意到角动量守恒,有
\[
L = \mu r v = \mu r^2 \dot{\theta}
\]
可以列出 $dt = \dfrac{\mu r^2}{L} d\theta$,带入并换元,有
\[
\bar{U} = -\frac{1}{T} \int_0^{2\pi} \frac{GM\mu}{r} \frac{\mu r^2}{L}d\theta = \frac{GM\mu^2}{L T} \int_0^{2\pi} r(\theta)d\theta
\]
带入椭圆的极坐标方程,可以得到
\[
\bar{U} = -\frac{GM\mu^2}{L T} \int_0^{2\pi} \frac{a(1-e^2)}{1+e\cos \theta}d\theta = \frac{GM\mu^2 a(1-e^2)}{L T} \frac{2\pi}{\sqrt{1-e^2}}
\]
同时带入椭圆轨道角动量的表达式和开普勒第三定律,可以化简出
\[
\bar{U} = -\frac{GM\mu}{a}
\]
注意:本题的平均不可以这样求
\[
\bar{U} = \frac{1}{2\pi} \int_0^{2\pi} \frac{GM\mu}{r(\theta)}d\theta
\]
我们希望求的是周期的平均,而上式是关于真近点角求平均。
\end{document}