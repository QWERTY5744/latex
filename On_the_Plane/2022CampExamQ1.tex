\documentclass{article}
\usepackage{graphicx} % Required for inserting images
\usepackage{amsmath,amssymb,amsthm}
\usepackage{physics}
\usepackage{graphicx,float}
\graphicspath{{images/}}
\usepackage[none]{hyphenat}
\usepackage{blindtext}
\usepackage{parskip}
\usepackage[letterpaper,top=3cm, left= 3cm,bottom=3cm]{geometry}
\usepackage{subcaption}

\title{Surprise}
\author{Polaris}
\date{2025/06/01}

\begin{document}
\maketitle

\section{Stellar Interiors}
The density profile of a hypothetical star given by:
\[
\rho(r) = \rho_0\left(1-\alpha\left(\frac{r}{R}\right)^2\right)
\]
where $r$ is the distance of a random point from the center of the star, $R$ is the
radius of the star, and $\alpha$ is a constant.

\begin{enumerate}
    \item What region of this star has the density $\rho_0$
    \item What is the mean density of the star
    \item Find $m(r)$ and total mass of the star
    \item Find the change of pressure $P$ with respect to $r$, $\displaystyle \dfrac{dP}{dr}$,
    in terms of $G$, $M$, $R$ and $r$.
    \item Give a possible range of $\alpha$ and explain your reasoning
\end{enumerate}

\newpage
\section{Solutions}
\begin{enumerate}
    \item When $\rho(r) = \rho_0$, it is not hard to notice that $r = 0$ when this happens.
    Thus, the answer is the core of the star.

    \item By the definition of average, one can write:
    \begin{align*}
        \overline{\rho} &= \frac{1}{R-0} \int_{0}^{R} \rho(r) dr\\
        &= \frac{1}{R} \int_{0}^{R} \rho_0\left(1-\alpha\left(\frac{r}{R}\right)^2\right) dr\\
        &= \frac{1}{R} \int_{0}^{R} \left(\rho_0- \frac{1}{R} \rho_0\alpha\left(\frac{r}{R}\right)^2\right) dr\\
        &= \frac{1}{R} \int_{0}^{R}\rho_0 dr - \frac{1}{R} \int_{0}^{R}\rho_0 \alpha \left(\frac{r}{R}\right)^2 dr\\
        &= \frac{1}{R} \rho_0 R - \frac{1}{R} \frac{\rho_0 \alpha}{R^2} \int_{0}^{R} r^2 dr\\
        &= \rho_0 - \frac{\rho_0 \alpha}{R^3} \frac{1}{3}R^3\\
        &= \frac{3 - \alpha}{3}\rho_0\\
    \end{align*}

    \item $m(r)$ can be given by integrating density with a volume element
    \begin{align*}
        m(r) &= \int_{0}^{r}\rho(r) dV\\
        &= \int_{0}^{r} \rho_0\left(1-\alpha\left(\frac{r}{R}\right)^2\right) 4\pi r^2 dr\\
        &= 4\pi \rho_0 \int_{0}^{r}r^2\left(1-\alpha\left(\frac{r}{R}\right)^2\right) dr\\
        &= 4\pi \rho_0 \int_{0}^{r}r^2 dr - \frac{4\pi \alpha \rho_0}{R^2} \int_{0}^{r}r^4 dr\\
        &=\frac{4}{3}\pi\rho_0 r^3 - \frac{4\pi\alpha\rho_0}{5R^2}r^5\\
    \end{align*}
    Total mass is thus given by $M = m(R)$:
    \begin{align*}
        M = m(R) &= \frac{4}{3}\pi\rho_0 R^3 - \frac{4\pi\alpha\rho_0}{5R^2}R^5\\
        &= \frac{4}{3}\pi R^3\rho_0 - \frac{4}{5}\alpha\pi R^3\rho_0\\
        &= 4\pi R^3\rho_0(\frac{1}{3}- \frac{1}{5}\alpha)\\
        &= \frac{4}{15}\pi R^3\rho_0 (5-3\alpha)\\
    \end{align*}

    \item By hydrostatic equalibrium equation:
    \[
    \dfrac{dP}{dr} = -\frac{Gm(r)\rho(r)}{r^2}
    \]
    Subsitute both $\rho(r)$ and $m(r)$ in, one get:
    \begin{align*}
        \dfrac{dP}{dr} &= -\frac{G}{r^2} \left(\frac{4}{3}\pi\rho_0 r^3 - \frac{4\pi\alpha\rho_0}{5R^2}r^5\right) \rho_0\left(1-\alpha\left(\frac{r}{R}\right)^2\right)\\
        &= -4\pi G\rho_0^2 r \left(\frac{1}{3} - \frac{\alpha r^2}{5R^2}\right) \left(1-\alpha\left(\frac{r}{R}\right)^2\right)\\
        &= -4\pi G\rho_0^2 r \left(\frac{5R^2}{15R^2} - \frac{3\alpha r^2}{15R^2}\right) \left(\frac{R^2 - \alpha r^2}{R^2}\right)\\
        &= -4\pi G\rho_0^2 r \left(\frac{5R^2 - 3\alpha r^2}{15R^2}\right) \left(\frac{R^2 - \alpha r^2}{R^2}\right)\\ 
        &= -4\pi G\rho_0^2 r \frac{5R^4 - 8\alpha R^2 r^2 + 3 \alpha^2 r^4}{15R^4}\\
        &= -\frac{4\pi G\rho_0^2 r}{15R^4} 5R^4 + \frac{4\pi G\rho_0^2 r}{15R^4} 8\alpha R^2 r^2 - \frac{4\pi G\rho_0^2 r}{15R^4} 3\alpha^2 r^4\\
        &= -\frac{4\pi G\rho_0^2 r}{3} + \frac{32\pi G\rho_0^2 r^3}{15R^2} \alpha - \frac{4\pi G\rho_0^2 r^5}{5R^4} \alpha^2\\
    \end{align*}
\end{enumerate}
\end{document}