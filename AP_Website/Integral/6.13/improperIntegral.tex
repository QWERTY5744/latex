\documentclass{article}
\usepackage{graphicx} % Required for inserting images
\usepackage{amsmath,amssymb,amsthm}
\usepackage{physics}
\usepackage{graphicx,float}
\graphicspath{{images/}}
\usepackage[none]{hyphenat}
\usepackage{blindtext}
\usepackage{parskip}
\usepackage[letterpaper,top=3cm, left= 3cm,bottom=3cm]{geometry}
\usepackage{subcaption}



\title{Improper Integral}
\author{Polaris}
\date{2024/12/14}

\begin{document}

\maketitle

Welcome to this guide on Evaluating Improper Integral of the FiveHive Calculus BC course. This article will guide you through the following:
\begin{enumerate}
    \item What is an improper integral?
    \item How to define the convergence and divergence of the integral?
    \item How to calculate a convergent improper integral?
\end{enumerate}

\section{Definition}
We define improper integral as:
\begin{enumerate}
    \item In the integration bound $[a,b]$, there is a infinite discontinuity
    \item The integral has a upper/lower bound of $\infty$ or $-\infty$
\end{enumerate}

\section{Calculate by Definition}
\subsection{Infinite Discontinuity in Integration Bound}
Let's first take a look at how to calculate them:
\begin{equation}
    \int_0^1 \frac{1}{x}\mathrm{d}x
\end{equation}

\begin{figure}[H]
    \includegraphics[width = 10cm]{pictures/improperintegral1.png}
    \centering
    \caption{Graph of $\frac{1}{x}$}
\end{figure}
Since we know that $f(x) = \frac{1}{x}$ is not defined at $x=0$, we need to use some clever trick to calculate this integral.

Let's rewrite the lower bound as $a$:
\[
    \int_a^1\frac{1}{x}\mathrm{d}x
\]
where $a>0$.

\newpage
It is obvious that this integral is not equal to the original integral we want to find, since the lower bound is not equal,
but if we take the limit as $a\to 0$, this integral will approach the integral we want to find, then we can apply the fundamental theorem of calculus and find the result:
\[
    \begin{split}
        \lim_{a\to 0^-} \int_{a}^{1}\frac{1}{x}\mathrm{d}x &= \lim_{a\to 0^-}\ln a - \ln 1 \\
        & = \infty - 0 \\
        & = \infty
    \end{split}
\]
Hence we can see that this integral diverges (the limit goes to infinity).

Let's take a look at where the integral converges:
\[
    \int_0^1\frac{1}{\sqrt{x}}\mathrm{d}x
\]
We can apply the same trick we used:
\[
    \begin{split}
        \lim_{a\to 0^-} \int_{a}^{1}\frac{1}{\sqrt{x}}\mathrm{d}x &= \lim_{a\to 0^-} -2\sqrt{a} + 2\sqrt{1} \\
        & = 0 + 2 \\
        & = 2
    \end{split}
\]
Which means this integral converges (the limit has a finite value)

\subsection{Infinite Integration Bound}
We will use some example to illustrate the idea of integrating on a infinte bound:
\[
    \int_{1}^{\infty} \frac{1}{x}\mathrm{d}x
\]
Let's first replace the upper bound to $b$, and note when $b$ become very large, the integral will get very close to the original integral:
\[
    \begin{split}
        \int_{1}^{\infty} \frac{1}{x}\mathrm{d}x & = \lim_{b\to \infty} \int_{1}^{b} \frac{1}{x}\mathrm{d}x\\
        & = \lim_{b\to \infty} \ln b - \ln 1 \\
        & = \infty - 0\\
        & = \infty
    \end{split}
\]

Which means this integral diverges, we can see that there is no difference between this improper integral and the one listed above, it is both taking a limit, the same also apply for negative infinity,
the lower bound will approach negative infinity.

Let's see a integral that converges:
\[
    \int_{1}^{\infty} \frac{1}{x^2}\mathrm{d}x
\]
\[
    \begin{split}
        \int_{1}^{\infty} \frac{1}{x^2}\mathrm{d}x & = \lim_{b\to \infty} \int_{0}^{b}\frac{1}{x^2}\mathrm{d}x\\
        & = \lim_{b\to \infty} -\frac{1}{b} + 1 \\
        & = 1
    \end{split}
\]

\subsection{Both Infinite Discontinuity and Infinite Integration Bound}
There are other improper integral that can be seen as a combination of both case 1 and case 2, to determine their convergence, we need to consider things separately.
Let $\displaystyle \int_{a}^{b} f(x)\mathrm{d}x$ be an impropoer integral, the integral converges \emph{only if} $\displaystyle \int_{a}^{c} f(x)\mathrm{d}x \text{ and } \int_{c}^{b} f(x)\mathrm{d}x$ both converges.

This gives a method to calculate some other improper integrals:
\[
    \int_{-\infty}^{\infty} \frac{1}{1+x^2}\mathrm{d}x
\]
We can rewrite the integral as:
\[
    \begin{split}
        \int_{-\infty}^{\infty} \frac{1}{1+x^2}\mathrm{d}x & = \int_{-\infty}^{0} \frac{1}{1+x^2}\mathrm{d}x + \int_{0}^{\infty} \frac{1}{1+x^2} \mathrm{d}x \\
        & = \lim_{b\to -\infty} \int_{b}^{0} \frac{1}{1+x^2}\mathrm{d}x + \lim_{a\to \infty} \int_{0}^{a}\frac{1}{1+x^2}\mathrm{d}x\\
        & = \arctan 0 - \lim_{b\to -\infty} \arctan b + \arctan 0 - \lim_{a\to \infty} \arctan a \\
        & = -\frac{-\pi}{2} + \frac{\pi}{2} \\
        & = \pi
    \end{split}
\]
Which means this integral converges to $\pi$.

\[
    \int_{0}^{\infty} \frac{1}{x^2}\mathrm{d}x
\]
It is convenient to split this integral into two parts and analyze them separately:
\[
    \begin{split}
        \int_{0}^{\infty} \frac{1}{x^2}\mathrm{d}x &= \int_{0}^{1}\frac{1}{x^2}\mathrm{d}x + \int_{1}^{\infty}\frac{1}{x^2} \mathrm{d}x\\
        & = \lim_{b\to 0^+} \int_{b}^{1} \frac{1}{x^2}\mathrm{d}x + \lim_{a\to \infty} \int_{1}^{a} \frac{1}{x^2} \mathrm{d}x \\
        & = -\frac{1}{1} + \lim_{b\to 0^+} \frac{1}{b} + 1\\
        & = \infty
    \end{split}
\]

There is a quicker way to do this:
\[
    \begin{split}
        \int_{0}^{1}\frac{1}{x^2}\mathrm{d}x & = -\frac{1}{1} + \lim_{b\to 0^+} \frac{1}{b} \\
        & = -1 + \infty \\
        & = \infty \\
    \end{split}
\]
This integral diverges, which by the theorem introduce earlier, the entire integral diverges.

It is \emph{extremely important} to check the infinite discontinuity within the integration bound, consider the following example:
\[
\int_{-1}^{1} \frac{1}{x^2}dx
\]

If one ignore that at $x=0$, the function has an infinte discontinuity and directly apply the fundamental theorem of calculus, one will get the incorrect result of $-2$:
\begin{align*}
    \int_{-1}^{1}\frac{1}{x^2}dx &= -\frac{1}{x} \Big|_{-1}^1\\
    &= -1 - 1 \\
    &= -2\\
\end{align*}

The correct approach is to recognize there is an infinite discontinuity at $x=0$, thus
\begin{align*}
    \int_{-1}^{1} \frac{1}{x^2}dx &= \int_{-1}^{0} \frac{1}{x^2} dx + \int_{0}^{1} \frac{1}{x^2}dx\\
    &= \lim_{a\to 0^-} \int_{-1}^{a} \frac{1}{x^2}dx + \lim_{b\to 0^+}\int_{b}^{1}\frac{1}{x^2}dx\\
    &= \lim_{a\to 0^-} -\frac{1}{a} - 1 + (-1) - \lim_{b\to 0^+} -\frac{1}{b}\\
    &= \infty
\end{align*}
\emph{Do not} assume that 2 infinities can cancel each other, in other words $\infty - \infty \neq 0$, this expression is undefined.

\newpage
\section{Practice}
\begin{enumerate}

    \item $\displaystyle \int_{1}^{\infty} \frac{1}{x^4}dx$
    
    A. This integral diverges

    B. $1$\

    C. $\displaystyle\frac{1}{3}$

    D. $\displaystyle \frac{1}{9}$
    \item $\displaystyle \int_{0}^{1} \ln x dx$
    
    A. This integral diverges

    B. $-e$

    C. $1$

    D. $-1$

    \item $\displaystyle \int_{0}^{2} \frac{1}{(1-x)^2}dx$
    
    A. This integral diverges

    B. $-2$

    C. $2$

    D. $1$

    \item $\displaystyle \int_{0}^{1} \frac{1}{\sqrt{x}}$
    
    A. This integral diverges

    B. $\displaystyle \frac{1}{2}$

    C. $-2$

    D. $2$

    \item $\displaystyle \int_{0}^{\infty} e^{-x}dx$
    
    A. This integral diverges

    B. $1$

    C. $-1$

    D. $e$
\end{enumerate}

\newpage
\begin{enumerate}
    \item 
    \begin{align*}
        \int_{1}^{\infty} \frac{1}{x^4} dx &= \lim_{b\to \infty} \frac{1}{x^4} dx\\
        &= \lim_{b\to \infty} \left(-\frac{1}{3x^3}\Big|_1^b\right)\\
        &= -\lim_{b\to \infty} \frac{1}{3b^3} + \frac{1}{3}\\
        &= \frac{1}{3}\\
    \end{align*}
    The answer is C.

    \item Let $u = \ln x$, $dv = dx$, thus $\displaystyle du = \frac{1}{x}$, $v = x$.
    \begin{align*}
        \int \ln x dx &= x \ln x - \int x\frac{1}{x}dx\\
        &= x\ln x - \int dx\\
        &= x\ln x - x +C\\
    \end{align*}
    Thus 
    \begin{align*}
        \int_{0}^{1} \ln x dx &= \lim_{a\to 0^-} \int_{a}^{1} \ln x dx\\
        &= \lim_{a\to 0^-} (x\ln x - x) \Big|_a^1\\
        &= 0 - 1 - \lim_{a\to 0^-} (a\ln x - a)\\
        &= -1 - \lim_{a\to 0^-} a\ln a\\
        &= -1 - \lim_{a\to 0} \frac{\ln a}{\frac{1}{a}}\\
        &= -1 - \lim_{a\to 0} \frac{\frac{1}{a}}{-\frac{1}{a^2}}\\
        &= -1 -\lim_{a\to 0} a\\
        &= -1\\
    \end{align*}
    The answer is D.

    \item 
    \begin{align*}
        \int_{0}^{2} \frac{1}{(1-x)^2}dx &= \int_{0}^{1} \frac{1}{(1-x)^2}dx + \int_{1}^{2} \frac{1}{(1-x)^2}dx\\
        &= \lim_{b\to 1^-}\int_{0}^{b} \frac{1}{(1-x)^2}dx + \lim_{a\to 1^+} \int_{a}^{2} \frac{1}{(1-x)^2}dx\\
        &= \lim_{b\to 1^-}\frac{1}{1-x} \Big|_0^b + \lim_{a\to 1^+} \frac{1}{1-x}\Big|_a^2\\
        &= \infty\\
    \end{align*}
    The integral diverges, the answer is A.

    If one did not notice the infinite discontinuity at $x=1$, the incorrect result of $-2$ will be obtained.

    \item 
    \begin{align*}
        \int_{0}^{1} \frac{1}{\sqrt{x}}dx &= \lim_{a\to 0^+} \int_{a}^{1} x^{-1/2}dx\\
        &= \lim_{a\to 0^+} 2x^{1/2} \Big|_a^1\\
        &= \lim_{a\to 0^+} (2\sqrt{1} - 2\sqrt{a})\\
        &= 2\\
    \end{align*}
    The answer is D.

    \item 
    \begin{align*}
        \int_{0}^{\infty} e^{-x}dx &= \lim_{b\to \infty} \int_{0}^{b} e^{-x}dx\\
        &= \lim_{b\to \infty} \left(-e^{-x}\right) \Big|_0^b\\
        &= \lim_{b\to \infty} \left(-e^{-b}\right) - (-e^0)\\
        &= 0+1 = 1\\
    \end{align*}
    The answer is B
\end{enumerate}
\end{document}