\documentclass{article}
\usepackage{graphicx} % Required for inserting images
\usepackage{amsmath,amssymb,amsthm}
\usepackage{physics}
\usepackage{graphicx,float}
\graphicspath{{images/}}
\usepackage[none]{hyphenat}
\usepackage{blindtext}
\usepackage{parskip}
\usepackage[letterpaper,top=3cm, left= 3cm,bottom=3cm]{geometry}
\usepackage{subcaption}
\usepackage{polynom}
\usepackage{longdivision}
\numberwithin{equation}{section}

\title{Long Division and Complete the Square}
\author{assassin3552}
\date{2025/04/09}

\begin{document}
\maketitle

\section{Long Division}
Before we begin, we need to learn about polynomial long division,
it is division design for dividing two polynomials, for example, 
if we want to find the result of $\displaystyle \frac{x^3+x^2-1}{x-1}$, 
we can set up this division
\[
    \polylongdiv[stage=1]{x^3+x^2-1}{x-1}
\]
Which is just like division we learn in primary school. Now we want to bind a quadratic term multiplied with $x-1$ produce something
like $x^3 + x^2$ (just like normal division), first try $x^2$:
\[
    \polylongdiv[stage=2]{x^3+x^2-1}{x-1}
\]
Multiply $x^2(x-1) = x^3 -x^2$, we have:
\[
    \polylongdiv[stage=4]{x^3+x^2-1}{x-1}
\]
Apply the same logic, try $2x$ and multiply it with $x-1$:
\[
    \polylongdiv[stage=7]{x^3+x^2-1}{x-1}
\]
Finally, we try $2$ as a part of the quotient and multiply it with $x-1$:
\[
    \polylongdiv{x^3+x^2-1}{x-1}
\]
\newpage

The remainder of $1$ means there is a leftover term with $1$ as its numerator,
overall, we can write the following equation:
\[
\frac{x^3+x^2-1}{x-1} = x^2+2x+2 + \frac{1}{x-1}
\]
You can verify this by combining the two fraction and see if it returns 
to the original fraction.

We can apply this techniques in integrations, if we want to compute this integral:
\[
\int \frac{x^3+x^2-1}{x-1} \mathrm{d}x = \int \left(x^2+2x+2 + \frac{1}{x-1}\right) \mathrm{d}x
\]

We can split the integral and easily compute the result of this integral:
\[
\int \frac{x^3+x^2-1}{x-1} \mathrm{d}x = \frac{1}{3}x^3 + x^2 + 2x + \ln\abs{x-1} + C
\]

Let's take a look at another example:
\[
\int \frac{5x^2 + x - 1}{x + 1}\mathrm{d}x
\]
First apply polynomial division:
\[
    \polylongdiv{5x^2 + x - 1}{x + 1}
\]
This means that we can re-write this integral as and easily compute the integral:
\[
\int 5x - 4 + \frac{3}{x+1} \mathrm{d}x = \frac{5}{2}x^2 - 4x + 3\ln\abs{x+1} + C
\]

\section{Complete the Square}
Recall when learning about the quadratic function, we learnt about a form of quadratic function called
the standard form, which is:
\[
y = (x-h)^2 + k
\]
When we want to complete the square, we want to fit a quadratic equation into this form, for example:
\begin{align*}
    x^2 + 6x + 10 &= x^2 + 6x + \left(\frac{6}{2}\right)^2 - \left(\frac{6}{2}\right)^2 + 10\\
    &= (x+3)^2 + 1\\
\end{align*}

In general, consider a quadratic expression of $x^2 + bx + c$, in order to turn this into a form
that looks like the standard form of quadratic function, we apply the following transformation:
\begin{align*}
    x^2 + bx + c &= x^2 + bx + \left(\frac{b}{2}\right)^2 - \left(\frac{b}{2}\right)^2 + c\\
    &= \left(x+ \frac{b}{2}\right)^2 - \frac{b^2-4c}{4}
\end{align*}

\newpage
Let's take a look at an example:
\[
\int \frac{8}{x^2 + 6x + 10}\mathrm{d}x
\]
We know that $x^2 + 6x + 10 = (x+3)^2 +1$, thus:
\begin{align*}
    \int \frac{8}{x^2 + 6x + 10}\mathrm{d}x &= 8\int \frac{1}{(x+3)^2 + 1}\mathrm{d}x\\
    &= 8 \int \frac{1}{u^2 + 1}\mathrm{d}u\\
    &= 8\arctan{u}+C = 8 \arctan{x+3}+C
\end{align*}
Here we preformed an u-substitution of $u = x+3$ and $\mathrm{d}u = \mathrm{d}x$ 

Another example is:
\[
\int \frac{2}{\sqrt{-x^2 + 10x - 24}}\mathrm{d}x
\]
First complete the square for $-x^2 + 10 x -24$:
\begin{align*}
    -x^2 + 10x + 24 &= -(x^2 - 10x + 24)\\
    &= - (x^2 - 10x + 25 -25 + 24)\\
    &= 1-(x-5)^2\\
\end{align*}
Thus:
\begin{align*}
    \int \frac{2}{\sqrt{-x^2 + 10x - 24}}\mathrm{d}x &= 2\int \frac{1}{\sqrt{1-(x-5)^2}} \mathrm{d}x\\
    &= 2\int \frac{1}{\sqrt{1-u^2}}\mathrm{d}u\\
    &= 2 \arcsin{u} + C\\
    &= 2\arcsin{(x-5)} +C\\
\end{align*}
Here we used an u-substitution of $u=x-5$ and $\mathrm{d}u = \mathrm{d}x$
\end{document}