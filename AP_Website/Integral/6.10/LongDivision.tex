\documentclass{article}
\usepackage{graphicx} % Required for inserting images
\usepackage{amsmath,amssymb,amsthm}
\usepackage{physics}
\usepackage{graphicx,float}
\graphicspath{{images/}}
\usepackage[none]{hyphenat}
\usepackage{blindtext}
\usepackage{parskip}
\usepackage[letterpaper,top=3cm, left= 3cm,bottom=3cm]{geometry}
\usepackage{subcaption}
\usepackage{polynom}
\usepackage{longdivision}
\numberwithin{equation}{section}

\title{Long Division and Complete the Square}
\author{assassin3552}
\date{2025/04/09}

\begin{document}
\maketitle
Welcome to the Long Division and Complete the Square method for integration of the FiveHive Calculus BC course. This article will guide you through what is long division, what is complete the squares, and how to apply it to integration.
\section{Long Division}
\subsection{What is Long Division}
\emph{Long division} is also known as \emph{polynomial division}, it is division design for dividing two polynomials, for example, 
if we want to find the result of $\displaystyle \frac{x^3+x^2-1}{x-1}$, 
we can set up this division
\[
    \polylongdiv[stage=1]{x^3+x^2-1}{x-1}
\]
Which is just like division we learn in primary school. Now we want to find a quadratic term multiplied with $x-1$ produce something
like $x^3 + x^2$ (just like normal division), first try $x^2$:
\[
    \polylongdiv[stage=2]{x^3+x^2-1}{x-1}
\]
Multiply $x^2(x-1) = x^3 -x^2$, we have:
\[
    \polylongdiv[stage=4]{x^3+x^2-1}{x-1}
\]
Apply the same logic, try $2x$ and multiply it with $x-1$:
\[
    \polylongdiv[stage=7]{x^3+x^2-1}{x-1}
\]
Finally, we try $2$ as a part of the quotient and multiply it with $x-1$:
\[
    \polylongdiv{x^3+x^2-1}{x-1}
\]

The remainder of $1$ means there is a leftover term with $1$ as its numerator,
overall, we can write the following equation:
\[
\frac{x^3+x^2-1}{x-1} = x^2+2x+2 + \frac{1}{x-1}
\]
You can verify this by combining the two fraction and see if it returns 
to the original fraction.

\subsection{Application to Integration}
Consider this integral:
\[
\int \frac{x^3+x^2-1}{x-1} \mathrm{d}x
\]

It is hard to preform a u-substitution or a trig substitution, that's where we try long division, we know that the integrant can be rewritten as such:
\[
\int \frac{x^3+x^2-1}{x-1} \mathrm{d}x = \int \left(x^2+2x+2 + \frac{1}{x-1}\right) \mathrm{d}x
\]

We can split the integral and easily compute the result of this integral:
\[
\int \frac{x^3+x^2-1}{x-1} \mathrm{d}x = \int \left(x^2+2x+2 + \frac{1}{x-1}\right) \mathrm{d}x = \frac{1}{3}x^3 + x^2 + 2x + \ln\abs{x-1} + C
\]

Let's take a look at another example:
\[
\int \frac{5x^2 + x - 1}{x + 1}\mathrm{d}x
\]
First apply polynomial division:
\[
    \polylongdiv{5x^2 + x - 1}{x + 1}
\]
This means that we can re-write this integral as and easily compute the integral:
\[
\int 5x - 4 + \frac{3}{x+1} \mathrm{d}x = \frac{5}{2}x^2 - 4x + 3\ln\abs{x+1} + C
\]

\section{Complete the Square}
Recall when learning about the quadratic function, we learnt about a form of quadratic function called
the standard form, which is:
\[
y = (x-h)^2 + k
\]
When we want to complete the square, we want to fit a quadratic equation into this form, for example:
\begin{align*}
    x^2 + 6x + 10 &= x^2 + 6x + \left(\frac{6}{2}\right)^2 - \left(\frac{6}{2}\right)^2 + 10\\
    &= (x+3)^2 + 1\\
\end{align*}

In general, consider a quadratic expression of $x^2 + bx + c$, in order to turn this into a form
that looks like the standard form of quadratic function, we add $\displaystyle \left(\frac{b}{2}\right)^2$ to the equation:
\begin{align*}
    x^2 + bx + c &= x^2 + bx + \left(\frac{b}{2}\right)^2 - \left(\frac{b}{2}\right)^2 + c\\
    &= \left(x+ \frac{b}{2}\right)^2 - \frac{b^2-4c}{4}
\end{align*}

Let's take a look at an example:
\[
\int \frac{8}{x^2 + 6x + 10}\mathrm{d}x
\]
We know that $x^2 + 6x + 10 = (x+3)^2 +1$, thus:
\begin{align*}
    \int \frac{8}{x^2 + 6x + 10}\mathrm{d}x &= 8\int \frac{1}{(x+3)^2 + 1}\mathrm{d}x\\
    &= 8 \int \frac{1}{u^2 + 1}\mathrm{d}u\\
    &= 8\arctan{u}+C = 8 \arctan{(x+3)}+C
\end{align*}
Here we preformed an u-substitution of $u = x+3$ and $\mathrm{d}u = \mathrm{d}x$ 

Another example is:
\[
\int \frac{2}{\sqrt{-x^2 + 10x - 24}}\mathrm{d}x
\]
First complete the square for $-x^2 + 10 x -24$:
\begin{align*}
    -x^2 + 10x + 24 &= -(x^2 - 10x + 24)\\
    &= - (x^2 - 10x + 25 -25 + 24)\\
    &=-((x-5)^2 - 1)\\
    &= 1-(x-5)^2\\
\end{align*}
Thus:
\begin{align*}
    \int \frac{2}{\sqrt{-x^2 + 10x - 24}}\mathrm{d}x &= 2\int \frac{1}{\sqrt{1-(x-5)^2}} \mathrm{d}x\\
    &= 2\int \frac{1}{\sqrt{1-u^2}}\mathrm{d}u\\
    &= 2 \arcsin{u} + C\\
    &= 2\arcsin{(x-5)} +C\\
\end{align*}
Here we preformed an u-substitution of $u=x-5$ and $\mathrm{d}u = \mathrm{d}x$

Let's take a look at a final example:
\[
\int \frac{1}{x^2 - 4x + 10} \mathrm{d}x
\]
First complete the square:
\begin{align*}
    x^2 -4x + 10 &= x^2 - 4x + 4 + 6\\
    &=(x-2)^2 +6\\
\end{align*}
Thus
\[
\int \frac{1}{x^2 - 4x + 10} \mathrm{d}x = \int \frac{1}{(x-2)^2 +6} \mathrm{d}x
\]
However here we see that the constant behind the squared expression is not $1$, to make a $1$, we factor the $6$ out from the expression:
\begin{align*}
    \int \frac{1}{(x-2)^2 +6} \mathrm{d}x &= \int \frac{1}{\frac{1}{6}\left(6(x-2)^2 + 1\right)}\mathrm{d}x\\
    &= 6\int \frac{1}{\left(\frac{x-2}{\sqrt{6}}\right)^2 +1}\mathrm{d}x\\
    &= 6\sqrt{6} \int \frac{1}{u^2+1} \mathrm{d}u\\
    &= 6\sqrt{6} \arctan u +C\\
    &= 6\sqrt{6} \arctan \left(\frac{x-2}{\sqrt{6}}\right)+C
\end{align*}
Here we preformed a u-substitution of $\displaystyle u = \frac{x-2}{\sqrt{6}}$ and $\displaystyle \mathrm{d}u = \frac{1}{\sqrt{6}}\mathrm{d}x$
\section{Exercises}
\begin{enumerate}
    \item $\displaystyle \int \frac{x^3}{x+3}\mathrm{d}x$

    A. $\displaystyle \frac{1}{3}x^3 - \frac{3}{2}x^2 + 9x - 27\ln\abs{x+3}+C$
    
    B. $\displaystyle \frac{1}{3}x^3 - \frac{3}{2}x^2 + 9x - 9\ln\abs{x+3} + C$
    
    C. $\displaystyle \frac{1}{3}x^3 - \frac{3}{2}x^2 + 9x +C$

    D. $\displaystyle \frac{1}{3}x^3 + \frac{3}{2} x^2 + 9x +C$
    \item $\displaystyle \int \frac{6x^3-7x^2+1}{2x-1}\mathrm{d}x$
    
    A. $\displaystyle \frac{1}{2} \arctan \left(\frac{3x-1}{2}x\right) + C$

    B. $\displaystyle \frac{9}{8} \arctan \left(\frac{6x-2}{9}\right) +C$

    C. $\displaystyle -x^3 + x^2 +x +C$

    D. $\displaystyle x^3 - x^2 - x + C$

    \item $\displaystyle \int \frac{2x+3}{x^2 + 3x + 10}\mathrm{d}x$
    
    A. $\ln \left(x^2+3x\right)+C$

    B. $\displaystyle \frac{1}{(x^2+3x+10)^2} +C$

    C. $\ln \left(x^2+3x+10\right)+C$

    D. $\ln \abs{x^2+3x+10}+C$
 
    \item $\displaystyle \int \frac{1}{x^2-2x+5}\mathrm{d}x$
    
    A. $\displaystyle \arctan (x-2)+C$

    B. $\displaystyle \frac{1}{2}\arctan \left(\frac{x-1}{2}\right) +C$

    C. $\displaystyle \frac{1}{4}\arctan\left(\frac{x-1}{4}\right) +C$

    D. $\displaystyle \frac{1}{2}\arcsin \left(\frac{x-1}{2}\right) +C$

    \item $\displaystyle \int \frac{3}{3x^2-5x+4}\mathrm{d}x$

    A. $\displaystyle \frac{36}{23} \arctan (x+\frac{5}{6}) +C$

    B. $\displaystyle \frac{216}{23\sqrt{23}}\arctan \left(\frac{\sqrt{23}\left(x+\frac{5}{6}\right)}{6}\right) +C$

    C. $\displaystyle \frac{6\sqrt{23}}{23}\arctan \left(\frac{6x+5}{\sqrt{23}}\right) +C$

    D. $\displaystyle \frac{6\sqrt{23}}{23}\arcsin \left(\frac{6x+5}{\sqrt{23}}\right) +C$
\end{enumerate}
Finish the exercises first and then check your answer

\section{Solutions}
\begin{enumerate}
    \item First, preform long division
    \[
    \polylongdiv{x^3}{x+3}
    \]
    We can convert the integral into the following
    \begin{align*}
        \int \frac{x^3}{x+3}\mathrm{d}x &= \int \left(x^2 - 3x + 9 - \frac{27}{x+3}\right) \mathrm{d}x\\
        &= \frac{1}{3}x^3 - \frac{3}{2} x^2 + 9x - 27\ln{\abs{x+3}} + C\\
    \end{align*}
    The correct answer is A.
    \item First, preform a long division
    \[
    \polylongdiv{6x^3-7x^2+1}{2x-1}
    \]
    We can now split the integral as such
    \begin{align*}
        \int \frac{6x^3-7x^2+1}{2x-1}\mathrm{d}x &= \int \left(3x^2-2x-1\right) \mathrm{d}x\\
        &= x^3 - x^2 - x + C\\
    \end{align*}
    The correct answer is D
    \item Surprisingly, this question does not involve any of the method taught in this article, it is a review of simple u-substitution.
    
    Let $u = x^2 + 3x + 10$, thus $\mathrm{d}u = (2x+3)\mathrm{d}x$, we can turn the integral into
    \begin{align*}
        \int \frac{2x+3}{x^2 + 3x + 10}\mathrm{d}x &= \int \frac{1}{u}\mathrm{d}u\\
        &= \ln \abs{u} + C = \ln \abs{x^2+3x+10}+C
    \end{align*}

    The correct answer is D
    \item \begin{align*}
        \int \frac{1}{x^2-2x+5}\mathrm{d}x &= \int \frac{1}{(x-1)^2 + 4}\mathrm{d}x\\
        &= \frac{1}{4}\int \frac{1}{\left(\frac{x-1}{2}\right)^2+1} \mathrm{d}x\\
        &= \frac{2}{4}\int \frac{1}{u^2 + 1}\mathrm{d}u\\
        &=\frac{1}{2}\arctan u + C \\
        &= \frac{1}{2}\arctan \left(\frac{x-1}{2}\right) + C\\
    \end{align*}
    Here we preformed a u-substitution of $u = \frac{x-1}{2}$ and $\mathrm{d}u = \frac{1}{2}\mathrm{d}x$

    The correct answer is B.
    \item \begin{align*}
        \int \frac{3}{3x^2-5x+4}\mathrm{d}x &= \int \frac{3}{3(x^2-\frac{5}{3}x+\frac{4}{3})}\mathrm{d}x\\
        &= \int \frac{1}{x^2 - \frac{5}{3}x + \left(\frac{5}{6}\right)^2 - \left(\frac{5}{6}\right)^2 + \frac{4}{3}}\mathrm{d}x\\
        &= \int \frac{1}{\left(x+\frac{5}{6}\right)^2 + \frac{23}{36}}\mathrm{d}x\\
        &=\int \frac{1}{\frac{23}{36}\left(\left(\frac{36}{23}\left(x+\frac{5}{6}\right)^2\right)+1\right)}\mathrm{d}x\\
        &= \frac{36}{23} \int \frac{1}{\left(\frac{6}{\sqrt{23}}\left(x+\frac{5}{6}\right)^2\right)+1}\mathrm{d}x\\
        &=\frac{36}{23}\frac{\sqrt{23}}{6}\int\frac{1}{u^2+1}\mathrm{d}u\\
        &= \frac{6\sqrt{23}}{23}\arctan u + C\\
        &= \frac{6\sqrt{23}}{23}\arctan \left(\frac{6x+5}{\sqrt{23}}\right) +C\\
    \end{align*}
    Here we preform a u-substitution of $\displaystyle u = \frac{6}{\sqrt{23}}\left(x+\frac{5}{6}\right) = \frac{6x+5}{\sqrt{23}}$, $\displaystyle\mathrm{d}u = \frac{6}{\sqrt{23}}\mathrm{d}x$
    
    The correct answer is C.
\end{enumerate}
\end{document}