\documentclass{article}
\usepackage{graphicx} % Required for inserting images
\usepackage{amsmath,amssymb,amsthm}
\usepackage{physics}
\usepackage{graphicx,float}
\graphicspath{{images/}}
\usepackage[none]{hyphenat}
\usepackage{blindtext}
\usepackage{parskip}
\usepackage[letterpaper,top=3cm, left= 3cm,bottom=3cm]{geometry}
\usepackage{subcaption}
\numberwithin{equation}{section}

\title{Integration by parts}
\author{assassin3552}
\date{2025/03/27}

\begin{document}
\maketitle
This article will introduce the integration method of \emph{integration by parts}, the derivation of its formula and its application
\section{Proving Integration by Parts Formula}
Start from the product rule of differentiation:
\[
\frac{\mathrm{d}}{\mathrm{d}x} (uv) = u'v + v'u
\]

If we multiply both side by $\mathrm{d}x$ on both side and take the indefinite integral, we have:
\[
\int \mathrm{d}(uv) = \int u'v \mathrm{d}x + \int v'u \mathrm{d}x
\]
Evaluate the integral, we have 
\[
\int u \mathrm{d}v = uv - \int v\mathrm{d}u
\]
Here we introduced a new variable that $v = v'\mathrm{d}x$ and $u = u'\mathrm{d}x$, you can see them as completely new variable and have no relation with the original $u$ and $v$

\newpage
\section{Indefinite Integral}
The formula for integration by parts is simple:
\[
\int u \mathrm{d}v = uv - \int v \mathrm{d}u
\]

\subsection{Basic Examples}
First let's start with an example question:
\[
\int x e^x \mathrm{d}x
\]
At first glance we don't see the $\mathrm{d}v$ structure, but notice that $e^x \mathrm{d}x = \mathrm{d}(e^x)$:
\[
\int x e^x \mathrm{d}x = \int x \mathrm{d}(e^x)
\]
Now we completed the $u\mathrm{d}v$ structure and we can use integration by parts:
\[
    \int x \mathrm{d}(e^x) = x e^x - \int e^x \mathrm{d}x = x e^x - e^x + C
\]

The key of integration by parts is turn $\text{some expression}\cdot \mathrm{d}x$ into $\mathrm{d}(\text{some expression})$, 
thus finding the expression becomes a skill, here are some expression to consider:
\begin{enumerate}
    \item $e^x$
    \item $\sin x$ and $\cos x$
    \item $x^n$
\end{enumerate}

Let's take a look at another another example:
\[
\int x\cos x \mathrm{d}x
\]
Let $u = x$ and $\mathrm{d}v = \mathrm{d}(\sin x) = \cos x \mathrm{d}x$, we can apply integration by parts:
\[
\int x \mathrm{d}(\sin x) = x\sin x \int \sin x \mathrm{d}x = x \sin x - \cos x + C
\]

\newpage
\subsection{Applying Integraion by Parts more than once}
Sometimes we need to apply integration by parts more than once, for example:
\[
\int e^x \sin x \mathrm{d}x
\]
Let $u = \sin x$, $\mathrm{d}v = \mathrm{d}(e^x) = e^x \mathrm{d}x$, by integration by parts:
\[
\int e^x \sin x \mathrm{d}x = \int \sin x \mathrm{d}(e^x) = e^x \sin x - \int e^x \cos x \mathrm{d}x
\]
Here we arrived at a new integral of $\int e^x \cos x \mathrm{d}x$, which again can be evaluated by integration by parts:

Let $u = \sin x$, $\mathrm{d}v = \mathrm{d}(e^x) = e^x \mathrm{d}x$, by integration by parts:
\[
\int e^x \cos x \mathrm{d}x = \int \cos x \mathrm{d}(e^x) = e^x \cos x + \int e^x \sin x \mathrm{d}x
\]

Here we see a problem, it seems that we need to evaluate our original integral to get an expression for our original integral, but this can be easliy bypassed,
notice that
\[
\int e^x \sin x \mathrm{d}x = e^x \sin x - e^x \cos x - \int e^x \sin x \mathrm{d}x
\]
We can treat our original integral as an unknown value and solve this equation, thus:
\[
2 \int e^x \sin x \mathrm{d}x = e^x \sin x - e^x \cos x + C
\]
It is not hard to see that
\[
\int e^x \sin x \mathrm{d}x = \frac{1}{2}e^x (\sin x - \cos x) + C
\]

\subsection{Inverse Trigonometric Function}
Integration by parts can be used to calculate the indefinite integral of inverse trig function:
\[
\int \arcsin x \mathrm{d}x
\]
We immediately see a $u\mathrm{d}v$ structure, let $u=\arcsin x$ and $\mathrm{d}v = \mathrm{d}x$:
\[
\int \arcsin x \mathrm{d}x = x\arcsin x - \int x \mathrm{d}(\arcsin x) = x\arcsin x - \int \frac{x}{\sqrt{1-x^2}} \mathrm{d}x
\]
The last integral can be evaluate with a u-substitution, thus we arrive at our final example:
\[
\int \arcsin x \mathrm{d}x = x\arcsin x + \sqrt{1-x^2} + C
\]
As an exersice, prove that
\[
\int \arctan{x} \mathrm{d}x = x\arctan{x} -\frac{1}{2}\ln{\abs{x^2 + 1}} + C
\]

\newpage
\section{Definite Integral}
For definite integral, the formula for integration by parts turn to:
\[
\int_{a}^{b} u\mathrm{d}v = uv \Big|_a^b - \int_{a}^{b} v\mathrm{d}u
\]

An example would be:
\[
\int_{0}^{\frac{1}{2}} \arcsin x \mathrm{d}x
\]
Previously we derived that
\begin{align*}
    \int_{0}^{\frac{1}{2}} \arcsin x \mathrm{d}x &= x\arcsin x \Big|_0^{\frac{1}{2}}- \int_{0}^{\frac{1}{2}} \frac{x}{\sqrt{1-x^2}} \mathrm{d}x\\
    &= \frac{1}{2} \arcsin \frac{1}{2} + \sqrt{1-x^2} \Big|_{0}^{\frac{1}{2}}\\
    &= \frac{\pi}{12} + \frac{\sqrt{3}}{2} - 1
\end{align*}

\end{document}