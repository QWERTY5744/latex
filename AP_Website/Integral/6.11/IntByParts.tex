\documentclass{article}
\usepackage{graphicx} % Required for inserting images
\usepackage{amsmath,amssymb,amsthm}
\usepackage{physics}
\usepackage{graphicx,float}
\graphicspath{{images/}}
\usepackage[none]{hyphenat}
\usepackage{blindtext}
\usepackage{parskip}
\usepackage[letterpaper,top=3cm, left= 3cm,bottom=3cm]{geometry}
\usepackage{subcaption}
\numberwithin{equation}{section}

\title{Integration by parts}
\author{assassin3552}
\date{2025/03/27}

\begin{document}
\maketitle
Welcome to this guide on Integration by Parts of the FiveHive Calculus BC course. This article will guide you through what is integration by parts and how to preform this technique.
\section{Deriving Integration by Parts Formula}
Start from the product rule of differentiation:
\[
\frac{\mathrm{d}}{\mathrm{d}x} (uv) = u'v + v'u
\]

If we multiply both side by $\mathrm{d}x$ on both side and take the indefinite integral, we have:
\[
\int \mathrm{d}(uv) = \int u'v \mathrm{d}x + \int v'u \mathrm{d}x
\]
Evaluate the integral, we have 
\[
\int u \mathrm{d}v = uv - \int v\mathrm{d}u
\]
Here we introduced a new variable that $v = v'\mathrm{d}x$ and $u = u'\mathrm{d}x$, you can see them as completely new variable and have no relation with the original $u$ and $v$

\newpage
\section{Indefinite Integral}
The formula for integration by parts is simple:
\[
\int u \mathrm{d}v = uv - \int v \mathrm{d}u
\]

\subsection{Basic Examples}
First let's start with an example question:
\[
\int x e^x \mathrm{d}x
\]
Let $u=x$, $dv = e^x dx$, thus $du = dx$, $v = e^x$, by integration of parts, we have 
\[
\int x e^x \mathrm{d}x = xe^x - \int e^x dx = xe^x - e^x +C
\]

The key of integration by parts is to find the correct $dv$, it must be easy to integrate, here are some expressions you should consider choosing as $dv$:
\begin{enumerate}
    \item $e^x dx$
    \item $\sin x dx$ and $\cos x dx$
    \item $x^n dx$
\end{enumerate}

Let's take a look at another another example:
\[
\int x\cos x \mathrm{d}x
\]
Let $u = x$ and $\mathrm{d}v = \cos x \mathrm{d}x$, thus $du = dx$ and $v = \sin x$, we can apply integration by parts:
\[
\int x \cos x dx = x\sin x - \int \sin x \mathrm{d}x = x \sin x - \cos x + C
\]

\newpage
\subsection{Applying Integraion by Parts more than once}
Sometimes we need to apply integration by parts more than once, for example:
\[
\int e^x \sin x \mathrm{d}x
\]
Let $u = \sin x$, $\mathrm{d}v = e^x \mathrm{d}x$, thus $du = \cos x dx$, $v = e^x$, by integration by parts:
\[
\int e^x \sin x \mathrm{d}x = e^x \sin x - \int e^x \cos x \mathrm{d}x
\]
Here we arrived at a new integral of $\int e^x \cos x \mathrm{d}x$, which again can be evaluated by integration by parts:

Let $u = \sin x$, $\mathrm{d}v = e^x \mathrm{d}x$, thus $du = -\cos x dx$ and $v = e^x$, by integration by parts:
\[
\int e^x \cos x \mathrm{d}x = \int \cos x \mathrm{d}(e^x) = e^x \cos x + \int e^x \sin x \mathrm{d}x
\]

Here we see a problem, it seems that we need to evaluate our original integral to get an expression for our original integral, but this can be easliy bypassed,
notice that
\[
\int e^x \sin x \mathrm{d}x = e^x \sin x - e^x \cos x - \int e^x \sin x \mathrm{d}x
\]
We can treat our original integral as an unknown value and solve this equation, thus:
\[
2 \int e^x \sin x \mathrm{d}x = e^x \sin x - e^x \cos x + C
\]
It is not hard to see that
\[
\int e^x \sin x \mathrm{d}x = \frac{1}{2}e^x (\sin x - \cos x) + C
\]

\subsection{Inverse Trigonometric Function}
Integration by parts can be used to calculate the indefinite integral of inverse trig function:
\[
\int \arcsin x \mathrm{d}x
\]
We immediately see a $u\mathrm{d}v$ structure, let $u=\arcsin x$ and $\mathrm{d}v = \mathrm{d}x$, thus $du = \frac{1}{\sqrt{1-x^2}} dx$, $v = x$:
\[
\int \arcsin x \mathrm{d}x = x\arcsin x - \int \frac{x}{\sqrt{1-x^2}} \mathrm{d}x
\]
The last integral can be evaluate with a u-substitution, thus we arrive at our final example:
\[
\int \arcsin x \mathrm{d}x = x\arcsin x + \sqrt{1-x^2} + C
\]

\newpage
\section{Definite Integral}
For definite integral, the formula for integration by parts turn to:
\[
\int_{a}^{b} u\mathrm{d}v = uv \Big|_a^b - \int_{a}^{b} v\mathrm{d}u
\]

An example would be:
\[
\int_{0}^{\frac{1}{2}} \arcsin x \mathrm{d}x
\]
Previously we derived that
\begin{align*}
    \int_{0}^{\frac{1}{2}} \arcsin x \mathrm{d}x &= x\arcsin x \Big|_0^{\frac{1}{2}}- \int_{0}^{\frac{1}{2}} \frac{x}{\sqrt{1-x^2}} \mathrm{d}x\\
    &= \frac{1}{2} \arcsin \frac{1}{2} + \sqrt{1-x^2} \Big|_{0}^{\frac{1}{2}}\\
    &= \frac{\pi}{12} + \frac{\sqrt{3}}{2} - 1
\end{align*}

\newpage
\section{Practice Problems}
Evaluate the following integrals
\begin{enumerate}
    \item $\displaystyle\int \arctan{x} \mathrm{d}x$
    
    A. $\displaystyle x\arctan{x} -\frac{1}{2}\ln{(x^2 + 1)} + C$

    B. $\displaystyle x\arctan x + \frac{1}{2}\ln (x^2+1)+C$

    C. $\displaystyle x\arctan{x} - \ln(x^2+1)+C$

    D.$\displaystyle x\arctan{x} -+ \sqrt{1-x^2} +C$

    \item $\displaystyle \int x\sin x dx$
    
    A. $x\cos x + \sin x +C$

    B. $-x\cos x - \sin x +C$

    C. $-x\cos x + \sin x +C$

    D. $x\cos x -\sin x +C$

    \item $\displaystyle \int e^{2x} \sin x \mathrm{d}x$
    
    A. $\displaystyle \frac{1}{2} e^{2x} \sin x - \frac{1}{2}e^{2x} \cos x + C$

    B. $\displaystyle \frac{2}{3}e^{2x}\sin x - \frac{1}{3} e^{2x} \cos x + C$

    C. $\displaystyle \frac{2}{5} e^{2x} \sin x - \frac{1}{5} e^{2x} \cos x + C$

    D. $\displaystyle \frac{1}{3} e^{2x} \sin x - \frac{1}{3} e^{2x} \cos x +C$

    \item $\displaystyle \int_{0}^{1} x e^{-x} dx$
    
    A. $\displaystyle \frac{2}{e} - 1$

    B. $\displaystyle -\frac{2}{e} +1$

    C. $\displaystyle -\frac{2}{e} - 1$

    D. $\displaystyle -\frac{2}{e} +1$

    \item $\displaystyle \int_{0}^{\frac{\pi}{2}} e^{2x}\cos x dx$
    
    A. $\displaystyle \frac{e^\pi - 2}{3}$

    B. $\displaystyle \frac{2-e^\pi}{5}$

    C. $\displaystyle \frac{-e^\pi - 2}{3}$

    D. $\displaystyle \frac{e^\pi - 2}{5}$
\end{enumerate}

\newpage 
\section{Solution}
\begin{enumerate}
    \item Let $u = \arctan x$, $dv = dx$, thus $du = \frac{1}{1+x^2} dx$, $v=x$:
    \begin{align*}
        \int \arctan x dx &= x\arctan x - \int \frac{x}{1+x^2}dx\\
        &= x\arctan x - \frac{1}{2}\int \frac{2x}{1+x^2} dx\\
        &= x\arctan x - \frac{1}{2} \int \frac{1}{u} du\\
        &= x\arctan x - \frac{1}{2} \ln \abs{u} +C\\
        &= x\arctan x - \frac{1}{2} \ln (x^2+1)+C\\
    \end{align*}
    Here a u-substitution of $u = 1+x^2$ and $du = 2x dx$ is preformed, note that the absolute sign is dropped because $x^2 + 1 > 0$ always holds true.

    The answer is A.

    \item Let $u = x$, $dv = \sin x dx = -d(\cos x)$, thus $du = dx$, $v = -\cos x$
    \begin{align*}
        \int x\sin x dx &= -x\cos x - \left(-\int \cos x dx\right)\\
        &= -x\cos x + \sin x +C\\
    \end{align*}
    The answer is C.

    \item Let $\displaystyle I = \int e^{2x} \sin x \mathrm{d}x$, $u = \sin x$, $dv = e^{2x} dx$, thus $du = \cos x dx$, $v = \frac{1}{2}e^{2x}$
    \begin{align*}
        \int e^{2x} \sin x \mathrm{d}x & = \frac{1}{2} e^{2x} \sin x - \frac{1}{2} \int e^{2x} \cos x \mathrm{d}x \\
    \end{align*}
    $\text{Let } u=\cos x \text{, } dv = e^{2x}dx \text{ thus, } du = -\sin x dx \text{ , } v = \frac{1}{2}e^{2x}$
    \begin{align*}
        & = \frac{1}{2} e^{2x} \sin x - \frac{1}{2}\left(e^{2x} \cos x -\int e^{2x} \mathrm{d}(\cos x)\right) \\
        & = \frac{1}{2} e^{2x} \sin x - \frac{1}{4} e^{2x} \cos x -\frac{1}{4}\int e^{2x} \sin x \mathrm{d}x \\
        I & = \frac{1}{2} e^{2x} \sin x - \frac{1}{4} e^{2x} \cos x -\frac{1}{4}I \\
        \frac{5}{4} I & = \frac{1}{2} e^{2x} \sin x - \frac{1}{4} e^{2x} \cos x \\
        I & =  \frac{2}{5} e^{2x} \sin x - \frac{1}{5} e^{2x} \cos x + C \\
    \end{align*}
    The answer is C.

    \item Let $u = x$, $dv = e^{-x}dx$, thus $du = dx$, $v = -e^{-x}$
    \begin{align*}
        \int_{0}^{1} x e^{-x} dx &= -xe^{-x} \Big|_0^1 + \int_{0}^{1} e^{-x} dx\\
        &= -\frac{1}{e} - e^{-x} \Big|_0^1\\
        &= -\frac{1}{e} -\left(\frac{1}{e} - 1\right)\\
        &= -\frac{2}{e} + 1
    \end{align*}
    The answer is B

    \item Let $I = \displaystyle \int_{0}^{\frac{\pi}{2}} e^{2x}\cos x dx$, $u = \cos x$, $dv = e^{2x}dx$, thus $ du = -\sin x dx$, $v = \frac{1}{2}e^{2x}$
    \begin{align*}
        I = \int_{0}^{\frac{\pi}{2}} e^{2x}\cos x dx &= \frac{1}{2}e^{2x} \cos x \Big|_0^{\frac{\pi}{2}} + \frac{1}{2}\int_{0}^{\frac{\pi}{2}} e^{2x} \sin x dx\\
    \end{align*}
    Let $u = \sin x$, $dv = e^{2x}dx$, thus $du = \cos x dx$, $v = \frac{1}{2}e^{2x}$
    \begin{align*}
        I &= \frac{1}{2} (0-1) + \frac{1}{2} \left(\frac{1}{2}e^{2x} - \frac{1}{2}\int_{0}^{\frac{\pi}{2}} e^{2x}\cos x\right)\\
        &= -\frac{1}{2} + \frac{1}{4}e^{2x} \sin x \Big|_0^{\frac{\pi}{2}} - \frac{1}{4} \int_{0}^{\frac{\pi}{2}} e^{2x} \cos x\\
        I &= -\frac{1}{2} + \frac{1}{4}(e^\pi - 0) - \frac{1}{4} I\\
        \frac{5}{4} I &= \frac{1}{2} + \frac{1}{4}e^\pi\\
        I &= \frac{4}{5} \left(\frac{1}{2} + \frac{1}{4} e^\pi\right)\\
        I &= \frac{e^\pi - 2}{5}
    \end{align*}
\end{enumerate}
\end{document} 