\documentclass{article}
\usepackage{graphicx} % Required for inserting images
\usepackage{amsmath,amssymb,amsthm}
\usepackage{physics}
\usepackage{graphicx,float}
\graphicspath{{images/}}
\usepackage[none]{hyphenat}
\usepackage{blindtext}
\usepackage{parskip}
\usepackage[letterpaper,top=3cm, left= 3cm,bottom=3cm]{geometry}
\usepackage{subcaption}

\title{Finding Basic Antiderivative}
\author{Polaris}
\date{2025/04/16}

\begin{document}

\maketitle

\section{What is Indefinite integral}
Recall definite integral and Fundamental Theorem of Calculus:
\[
\int_{a}^{b}f(x)dx = F(b) - F(a)
\]
Where $F'(x) = f(x)$, note definite integral only returns a number, 
what if we define a new operation that will return a function that is kind of like integrals?

This operation is called indefinite integral, given a function $f(x)$, it returns a function which 
has a derivative of $f(x)$, we denote this operation as this:
\[
\int f(x)dx = F(x)
\]
Where $F'(x) = f(x)$.

However this is not the complete result, recall that the derivative of a constant is 0,
which means $\displaystyle \dfrac{d}{dx}F(x)+C = f(x)$, thus for an indefinite integral, the correct answer would be
\[
\int f(x)dx = F(x)+C
\]
Where $C$ is any constant and $F'(x) = f(x)$

Indefinite follows two important rules:
\begin{enumerate}
    \item \[
    \int (f(x) + g(x)) dx = \int f(x) dx + \int g(x) dx
    \]
    \item \[
    \int cf(x) dx = c\int f(x)dx
    \]
\end{enumerate}

\newpage
\section{How to calculate Indefinite Integral}
Let's take a look at an example question:
\[
\int (\cos x + 7e^x)dx
\]
We can split this integral into 2 and evaluate them separately:
\[
    \int (\cos x + 7e^x)dx = \int \cos x dx + \int 7e^x dx
\]
Recall the defintion of indefinite integral, if we want to find the indefintite integral of
$\cos x$, we are simply finding a function that has a derivative of $\cos x$, and this function is $\sin x$, thus we have:
\[
\int \cos x dx = \sin x +C_1
\]
Same logic apply for $7e^x$, we have
\[
\int 7e^x dx = 7e^x +C_2
\]
Sum everything up, we have
\[
    \int (\cos x + 7e^x)dx = \sin x + C_1 + 7e^x +C_2
\]
Here $C_1$ and $C_2$ are simply constant, we denote their sum as a new constant $C$, thus we have
\[
    \int (\cos x + 7e^x)dx = \sin x + 7e^x +C
\]

\newpage
\section{List of Basic Integral}
We know how to calculate basic integrals now, but it is ineffcient to go back and use the definition everytime,
here we start from derivatives and created a list for basic integrals, these are the building blocks of more complex integral
and must be remember:
\begin{enumerate}
    \item $\displaystyle \int dx = x + C$
    \item $\displaystyle \int x^n dx = \frac{x^{n+1}}{n+1} +C$ (if $n\neq -1$)
    \item $\displaystyle \int \frac{1}{x}dx = \ln{\abs{x}}+C$ (remember the absolute value)
    \item $\displaystyle \int e^x dx = e^x +C$
    \item $\displaystyle \int b^x dx = \frac{1}{\ln b} b^x+C$
    \item $\displaystyle \int \sin x dx= -\cos x +C$
    \item $\displaystyle \int \cos x dx = \sin x +C$
    \item $\displaystyle \int \sec^2 x dx = \tan x +C$
    \item $\displaystyle \int \csc^2 x dx -\cot x+C$
    \item $\displaystyle \int \sec x \tan x dx = \sec x+C$
    \item $\displaystyle \int \csc x \cot x = -\csc x+C$
    \item $\displaystyle \int \frac{1}{\sqrt{1-x^2}}dx = \arcsin x$ (if $\abs{x}<1$)
    \item $\displaystyle \int \frac{1}{1+x^2}dx = \arctan x+C$
    \item $\displaystyle \int \frac{1}{x\sqrt{x^2-1}dx} = \arcsec x +C$
\end{enumerate}
If any of those feel rusty, consult back to the derivative chapter. 

With this chart, we can solve for basic integrals, here are some example question:
\begin{align*}
    \int \left(\frac{1}{\sqrt[3]{x^2}}-\frac{4}{5x}+\frac{8}{1+x^2}dx\right) &= \int \frac{1}{\sqrt[3]{x^2}}dx - \int \frac{4}{5x}dx + \int\frac{8}{1+x^2}dx\\
    &=\int x^{-\frac{2}{3}} dx- \frac{4}{5} \int \frac{1}{x}dx + 8 \int \frac{1}{1+x^2}dx\\
    &= 3x^{\frac{1}{3}} - \frac{4}{5}\ln{\abs{x}} + 8\arctan(1+x^2)+C
\end{align*}
Here we used equation 2, 3 and 13.
\newpage
Let's take a look at another example:
\begin{align*}
    \int \tan^2 x dx &= \int (\sec^2 x -1)dx\\
    &= \int \sec^2 x dx - \int dx\\
    &= \tan x - x +C
\end{align*}
Here we used an trig equation of $\sec^2 x = \tan^2 x +1$ and equation 8.
\end{document}