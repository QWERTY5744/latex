\documentclass{article}
\usepackage{graphicx} % Required for inserting images
\usepackage{amsmath,amssymb,amsthm}
\usepackage{physics}
\usepackage{graphicx,float}
\graphicspath{{images/}}
\usepackage[none]{hyphenat}
\usepackage{blindtext}
\usepackage{parskip}
\usepackage[letterpaper,top=3cm, left= 3cm,bottom=3cm]{geometry}
\usepackage{subcaption}
\numberwithin{equation}{section}

\title{Finding Taylor or Maclaurin Series for a Function}
\author{Polaris}
\date{2025/07/25}

\begin{document}

\maketitle

\section{Taylor Series}
Talyor series can approximate an elementary function.

\subsection{Intuitive understanding}
Consider someone walking, which cannot be modeled using limited polynomials, if we know where the person started, his velocity, his acceleration at his starting point, we can approximate his path using kinematic equations.
(assuming constant acceleration)
\[
    r = r_0 + v_0t + \frac{1}{2}a_0t^2
\]
If we know the jerk (derivative of acceleration) at a point, we can furthur approximate its path:
\[
    r = r_0 + v_0t + \frac{1}{2}a_0t^2 + \frac{1}{6} j_0 t^3
\]
We can do this further if we know the snap (derivative of jerk), crackle (derivative of snap) and pop (derivative of crackle), we can approximate his path as this:
\[
    r = r_0 + v_0t + \frac{1}{2}a_0 t^2 + \frac{1}{6} j_0 t^3 + \frac{1}{24} s_0 t^4 + \frac{1}{120} c_0 t^5 + \frac{1}{720} p_0 t^6
\]
(These are all real terms)

This is essentially Taylor Series, it finds the change of a function at one point, then the change of change of function at one point, and so on.

\subsection{Formula}
Assume we can approximate the function like this:
\[
    P_n(x) = a_0 + a_1 (x-c) + a_2 (x-c)^2 + ... + a_n (x-c)^n
\]
We require that $P_n(c) = f(c)$, $P'_n(c) = f'(c)$, $P''_n(c) = f''(c)$ and so on.

Thus Taylor Series is:
\[
    f(x) = \sum_{n = 0}^{\infty} \frac{f^{(n)}(a)}{n!}(x-a)^n
\]
If $a=0$, then the series is called the Maclaurin Series.

\subsection{Example Question}
Find the Maclaurin Series of the following function
\begin{enumerate}
    \item $f(x) = e^x$
    
    To find the Maclaurin series of this function, first we need to find $f^n(0)$, it is not hard to see $f^n(x) = e^x$, thus $f^n(0) = e^0 = 1$, hence the Maclaurin series is 
    \[
    f(x) = \sum_{n=0}^{\infty} \frac{f^n(0)}{n!} (x-0)^n = \sum_{n=0}^{\infty} \frac{x^n}{n!} = 1 + x + \frac{x^2}{2!} + \frac{x^3}{3!} + ... + \frac{x^n}{n!}
    \]

    \item $f(x) = \sin x$
    
    To find the Maclaurin of this function, we need to find $f^n(0)$ first. We shall first find the derivatives of $f(x)$:
    \begin{enumerate}
        \item $f(x) = \sin x$, $f(0) = 0$
        \item $f'(x) = \cos x$, $f'(0) = 1$
        \item $f''(x) = -\sin x$, $f''(0) = 0$
        \item $f^3 (x) = -\cos x$, $f^3(0) = -1$
        \item $f^4(x) = \sin x$, $f^4(0) = 0$
        \item $f^5(x) = \cos x$, $f^5(0) = 1$
        \item $f^6(x) = -\sin x$, $f^6(0) = 0$
        \item $f^7(x) = -\cos x$, $f^7(0) = -1$
        \item $f^8(x) = \sin x$, $f^8(0) = 0$
    \end{enumerate}
    It is not hard to notice there is a cycle of $0$, $1$, $0$, $-1$ going on, thus the Maclaurin series turns into
    \[
    f(x) = \sum_{n=0}^{\infty} \frac{f^n(0)}{n!} (x-0)^n = 0\cdot 1 + 1\cdot x + 0\cdot \frac{x^2}{2!} - 1\cdot\frac{x^3}{3!} + ... + \frac{(-1)^nx^{2n+1}}{(2n+1)!} = \sum_{n=0}^{\infty} \frac{(-1)^nx^{2n+1}}{(2n+1)!}
    \]

    \item $\displaystyle f(x) = \frac{1}{1-x}$
    
    One can definitely repeat the method we used before to obtain the Maclaurin series of this function, I encourage you to work the series out in this way, but there are faster approaches, recall that for a convergent geometric series, the following equation holds true:
    \[
    \sum_{n=0}^{\infty} a_0 \cdot r^n = \frac{a_0}{1-r}
    \]
    Let $a_0 = 1$ and $r = x$, we have
    \[
    \sum_{n=0}^{\infty} x^n = \frac{1}{1-x}
    \]
    The left side of the equality is merely a infinte sum of power functions, which fits the form of Maclaurin series, thus we can conclude
    \[
    \frac{1}{1-x} = 1 + x + x^2 + x^3 + ... + x^n = \sum_{n=0}^{\infty} x^n
    \]

    Before we continue to do more question, here are some important Maclaurin Series you should memorize:
    \begin{enumerate}
    \item $\displaystyle e^x = \sum_{n=0}^{\infty} \frac{x^n}{n!} = 1 + x + \frac{x^2}{2!} + \frac{x^3}{3!} + ... + \frac{x^n}{n!}$
    \item $\displaystyle \sin x = \sum_{n=0}^{\infty} \frac{(-1)^nx^{2n+1}}{(2n+1)!} = x - \frac{x^3}{3!} + \frac{x^5}{5!} - \frac{x^7}{7!} + ... + \frac{(-1)^nx^{2n+1}}{(2n+1)!}$
    \item $\displaystyle \cos x = \sum_{n=0}^{\infty} \frac{(-1)^n x^{2n}}{(2n)!} = 1 - \frac{x^2}{2!} + \frac{x^4}{4!} - \frac{x^6}{6!} + ... + \frac{(-1)^n x^{2n}}{(2n)!}$
    \item $\displaystyle \frac{1}{1-x} = \sum_{n=0}^{\infty} x^n = 1 + x + x^2 + x^3 + ... + x^n$ (note this series only converge if $x \in (-1,1)$)
    \end{enumerate}

    \item $f(x) = x\sin x$
    
    Using the old method for this function will be to time consuming, here is a faster approach. We already worked out the Maclaurin series for $\sin x$, we can direcly substitute this into the function
    \[
    x\sin x = x \cdot \sum_{n=0}^{\infty} \frac{(-1)^nx^{2n+1}}{(2n+1)!} = \sum_{n=0}^{\infty} \frac{(-1)^nx^{2n+1}}{(2n+1)!}
    \]
\end{enumerate}

\section{Practice Problems}
Find the Maclaurin series of the following function

1. $f(x) = e^{-x}$

A. $\displaystyle \sum_{n=0}^{\infty} \frac{x^n}{n!}$

B. $\displaystyle \sum_{n=0}^{\infty} \frac{(-1)^n x^n}{n!}$

C. $\displaystyle \sum_{n=0}^{\infty} \frac{(-1)^{n+1} x^n}{n!}$

D. $\displaystyle \sum_{n=0}^{\infty} -\frac{x^n}{n!}$

2. $f(x) = \sin x^2$

A. $\displaystyle \sum_{n=0}^{\infty} \frac{(-1)^n x^{2n+1}}{(2n+1)!}$

B. $\displaystyle \sum_{n=0}^{\infty} \frac{x^{2n+1}}{(2n+1)!}$

C. $\displaystyle \sum_{n=0}^{\infty} \frac{(-1)^n x^{4n+2}}{(2n+1)!}$

D. $\displaystyle \sum_{n=0}^{\infty} \frac{x^{4n+2}}{(2n+1)!}$

3. $f(x) = \dfrac{1}{1+x^2}$, $x\in (-1,1)$

A. $\displaystyle \sum_{n=0}^{\infty} (-1)^{n+1}x^n$

B. $\displaystyle \sum_{n=0}^{\infty} (-1)^{n} x^n$

C. $\displaystyle \sum_{n=0}^{\infty} (-1)^{n+1} x^{2n}$

D. $\displaystyle \sum_{n=0}^{\infty} (-1)^n x^{2n}$

4. $f(x) = x\cos x^2$

A. $\displaystyle \sum_{n=0}^{\infty} \frac{(-1)^n x^{4n+1}}{(2n)!}$

B. $\displaystyle \sum_{n=0}^{\infty} \frac{(-1)^n x^{4n}}{(2n)!}$

C. $\displaystyle \sum_{n=0}^{\infty} \frac{(-1)^n x^{2n+1}}{(2n)!}$

D. $\displaystyle \sum_{n=0}^{\infty} \frac{(-1)^n x^{2n}}{(2n)!}$

5. $f(x) = \dfrac{1}{x}$ centered at $x=1$

A. $\displaystyle \sum_{n=0}^{\infty} (-1)^n x^n$

B. $\displaystyle \sum_{n=0}^{\infty} (-1)^{n+1} x^n$

C. $\displaystyle \sum_{n=0}^{\infty} (-1)^n (x-1)^n$

D. $\displaystyle \sum_{n=0}^{\infty} (-1)^{n+1} (x-1)^n$

\newpage
\section{Solution}

1. Recall the Maclaurin Series of $e^x$
\[
e^x = \sum_{n=0}^{\infty} \frac{x^n}{n!}
\]
Substitute $x$ as $-x$, the Maclaurin series become
\[
e^{-x} = \sum_{n=0}^{\infty} \frac{(-x)^n}{n!} = \sum_{n=0}^{\infty} \frac{(-1)^n x^n}{n!}
\]
The answer is B

2. Recall the Maclaurin series of $\sin x$,
\[
\sin x = \sum_{n=0}^{\infty} \frac{(-1)^nx^{2n+1}}{(2n+1)!}
\]
Substitute $x = x^2$ (it looks problematic but essentially what is means is $x$ is a notation for the variable of this function, and this variable is $x^2$), the Maclaurin series become
\[
\sin x^2 = \sum_{n=0}^{\infty} \frac{(-1)^n x^{2n+1}}{(2n+1)!} 
\]
The answer is C.

3. Recall the Maclaurin series of $\dfrac{1}{1-x}$
\[
\frac{1}{1-x} = 1 + x + x^2 + ... + x^n + ... = \sum_{n=0}^{\infty} x^n
\]
First, find the Maclaurin series of $\dfrac{1}{1+x}$
\[
\frac{1}{1+x} = \frac{1}{1- (-x)} = 1 + (-x) + (-x)^2 + (-x)^3 + ...= 1-x+x^2-x^3+x^4-...+(-1)^n x^n = \sum_{n=0}^{\infty} (-1)^n x^n
\]
Then find the Maclaurin series of $\dfrac{1}{1+x^2}$
\[
\frac{1}{1+x^2} = 1 - x^2 + x^4 - x^6 +... + (-1)^n x^{2n} = \sum_{n=0}^{\infty} (-1)^n x^{2n}
\]
The answer is D.

4. Recall the Maclaurin series of $\cos x$
\[
\cos x = \sum_{n=0}^{\infty} \frac{(-1)^n x^{2n}}{(2n)!}
\]
Thus the Maclaurin series of $\cos x^2$ can be easily obtained 
\[
\cos x^2 = \sum_{n=0}^{\infty} \frac{(-1)^n x^{4n}}{(2n)!}
\]
Multiply $x$ to this series, the final series become
\[
x\cos x^2 = \sum_{n=0}^{\infty} \frac{(-1)^n x^{4n+1}}{(2n)!}
\]
The answer is A.

4. First find the derivatives of $\dfrac{1}{x}$,
    \begin{enumerate}
        \item $f(x) = \dfrac{1}{x}$, $f(1) = 1$
        \item $f'(x) = -\dfrac{1}{x^2}$, $f'(1) = -1$
        \item $f''(x) = \dfrac{2}{x^3}$, $f''(1) = 2$
        \item $f^3 (x) = -\dfrac{6}{x^4}$, $f^3(1) = -6$
        \item $f^n(x) = (-1)^n \dfrac{n!}{x^{n+1}}$, $f^n(1) = (-1)^n n!$
    \end{enumerate}
    Therefore the Taylor series is 
    \[
    \sum_{n=0}^{\infty} \frac{f^n(a)}{n!}(x-a)^n = \sum_{n=0}^{\infty} \frac{(-1)^n n!}{n!} (x-1)^n = \sum_{n=0}^{\infty} (-1)^n (x-1)^n
    \]
    The answer is C.
\end{document}