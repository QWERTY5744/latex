\documentclass{article}
\usepackage{graphicx} % Required for inserting images
\usepackage{amsmath,amssymb,amsthm}
\usepackage{physics}
\usepackage{graphicx,float}
\graphicspath{{images/}}
\usepackage[none]{hyphenat}
\usepackage{blindtext}
\usepackage{parskip}
\usepackage[letterpaper,top=3cm, left= 3cm,bottom=3cm]{geometry}
\usepackage{subcaption}
\numberwithin{equation}{section}

\title{Finding Taylor Polynomial Approximations of Functions}
\author{assassin3552}
\date{2025/08/07}

\begin{document}

\maketitle
\section{What is Taylor Polynomial}
In the field of physics and engineering, approximations is as important as finding the actual value, therefore, there needs to be a way of approximating functions. Here is where Taylor Polynomials comes into place.

\subsection{Definition of Taylor Polynomial}
Consider a function $f$, which can be differentiate $n$ times at a constant number $a$. We define the Taylor Polynomial of this function $f$ around $x=a$ as 
\[
\boxed{p_n(x) = f(a) + f'(a)(x-a) + \frac{f''(a)}{2!}(x-a)^2 + ... + \frac{f^{n}(a)}{n!} (x-a)^n}
\]
We call this polynomial function "the $n$th degree Taylor Polynomial of $f$", if $a=0$, the polynomial is called Maclaurin Polynomial.

The details of this will be covered in 10.14 Taylor Series.

\subsection{Applications of Taylor Polynomial}
The most important real world applications of Taylor Polynomial is approximating a function.

Consider the sine function $f(x) = \sin x$ and its $9$th degree Taylor Polynomial (we will cover how to find this in the next section) $p_9(x) = x - \dfrac{x^3}{3!} + \dfrac{x^5}{5!} - \dfrac{x^7}{7!} + \dfrac{x^9}{9!}$

\begin{figure}[H]
    \centering
    \includegraphics[width=12cm]{pic/taylor1.png}
    \caption{The graph of $\sin x$ and $p_9(x)$}
\end{figure}

From the graph, one can see that for some part of the graph, the blue line and red line match almost perfectly, it is only after a certain value that the line starts to deviate.

This is the purpose of Taylor Polynomials, to approximate functions, the higher its degree, the better approximation it gets. Here is another example of $e^x$ and its $4$th degree Taylor Polynomial.

\begin{figure}[H]
    \centering
    \includegraphics[width=12cm]{pic/taylor2.png}
    \caption{The graph of $e^x$ and $p_4(x)$}
\end{figure}

\section{Finding Taylor Polynomial}
In this section, you will learn how to find Taylor Polynomial.

1. Find the $4$th degree Taylor Polynomial of $\sin x$ centered at $x=0$

To find the desired polynomial, first find the high order derivatives of $\sin x$ at $x=0$

\begin{enumerate}
        \item $f(x) = \sin x$, $f(0) = 0$
        \item $f'(x) = \cos x$, $f'(0) = 1$
        \item $f''(x) = -\sin x$, $f''(0) = 0$
        \item $f^3 (x) = -\cos x$, $f^3(0) = -1$
        \item $f^4(x) = \sin x$, $f^4(0) = 0$
\end{enumerate}
Now substitute the derivatives into the equation for Taylor Polynomial:
\[
p_n(x) = f(a) + f'(a)(x-a) + \frac{f''(a)}{2!}(x-a)^2 + ... + \frac{f^{n}(a)}{n!} (x-a)^n
\]
\[
p_4(x) = f(0) + f'(0)(x-0) + \frac{f''(0)}{2!}(x-0)^2 + \frac{f^3(0)}{3!}(x-0)^3 + \frac{f^4(0)}{4!}(x-0)^4
\]
\[
= 0 + 1x + \frac{0}{2!}(x-0)^2 + \frac{-1}{3!}(x-0)^3 + \frac{0}{4!}(x-0)^4
\]
\[
p_4(x)= x - \frac{x^3}{3!}
\]

2. Find the $4$th degree Taylor Polynomial of the function $f(x) = \ln x$ center at $x=1$

First find the high order derivatives of $\ln x$ at $x=1$

\begin{enumerate}
        \item $f(x) = \ln x$, $f(1) = 0$
        \item $f'(x) = \dfrac{1}{x}$, $f'(1) = 1$
        \item $f''(x) =  -\dfrac{1}{x^2}$, $f''(1) = -1$
        \item $f^3 (x) = \dfrac{2}{x^3}$, $f^3(1) = 2$
        \item $f^4(x) = -\dfrac{6}{x^4}$, $f^4(1) = -6$
\end{enumerate}
Now substitute the derivatives into the equatin for Taylor Polynomial,
\[
p_4(x) = f(1) + f'(1)(x-1) + \frac{f''(1)}{2!}(x-1)^2 + \frac{f^3(1)}{3!}(x-1)^3 + \frac{f^4(1)}{4!}(x-1)^4
\]
\[
= 0 + 1(x-1) - \frac{1}{2}(x-1)^2 + \frac{2}{3!}(x-1)^3 - \frac{6}{4!}(x-1)^4
\]
\[
p_4(x) = (x-1) - \frac{1}{2}(x-1)^2 + \frac{1}{3}(x-1)^3 - \frac{1}{4}(x-1)^4
\]

\section{Practice Problems}
1. Find the $3$rd order Taylor Polynomial of $f(x) = e^x$ centered at $x=0$

A. $1 + x + \dfrac{x^2}{2!} + \dfrac{x^3}{3!}$

B. $1 - x + \dfrac{x^2}{2!} - \dfrac{x^3}{3!}$

C. $1 + x + x^2 + x^3$

D. This Taylor Polynomial cannot be obtained

2. Find the $4$th order Taylor Polynomial of $f(x) = \ln x$ centered at $x=0$

A. $x - \frac{1}{2}x^2 + \frac{1}{3} x^3 - \frac{1}{4} x^4$

B. $1 - x + \frac{1}{2}x^2 - \frac{1}{3} x^3 + \frac{1}{4} x$

C. $-x + \frac{1}{2}x^2 - \frac{1}{3} + \frac{1}{4}x^4$

D. This Taylor Polynomial cannot be obtained

3. Use the $3$rd degree Taylor Polynomial of $f(x) = \cos x$ centered at $x=0$ to approximate $\cos(0.1)$

A. $0.0998$

B. $0.9947$

C. $0.9950$

D. This Taylor Polynomial cannot be obtained, hence $\cos(0.1)$ cannot be approximated

4. Use the $3$th degree Taylor Polynomial of $f(x) = \arcsin (x)$ centered at $x=0$ to find the approximate value of $\arcsin (0.1)$

A. $0.9998$

B. $0.1002$

C. $0.9767$

D. This Taylor Polynomial cannot be obtained, hence $\arcsin(0.1)$ cannot be approximated

5. Find the approximate value of $\sqrt[3]{30}$ using $3$rd order Taylor Polynomial centered at $x=27$

A. $2.89275$

B. $3.10767$

C. $3.10725$

D. This Taylor Polynomial cannot be found, therefore $\sqrt[3]{30}$ cannot be approximated.


\section{Solution}
1. First find the high order derivatives of $e^x$ at $x=0$

\begin{enumerate}
        \item $f(x) = e^x$, $f(0) = 1$
        \item $f'(x) = e^x$, $f'(0) = 1$
        \item $f''(x) =  e^x$, $f''(0) = 1$
        \item $f^3 (x) = e^x$, $f^3(0) = 1$
\end{enumerate}
Now substitute the derivatives into the equatin for Taylor Polynomial,
\[
p_3(x) = f(1) + f'(1)(x-1) + \frac{f''(1)}{2!}(x-1)^2 + \frac{f^3(1)}{3!}(x-1)^3
\]
\[
= 1 + 1x - \frac{1}{2}x^2 + \frac{1}{3!}x^3 
\]
\[
1 + x + \dfrac{x^2}{2!} + \dfrac{x^3}{3!}
\]
The answer is A.

2. The function $f(x) = \ln x$ is not defined at $x=0$, meaning that this Taylor Polynomial cannot be obtained.

The answer is D.

3. First find the high order derivatives of $\cos x$ at $x=0$

\begin{enumerate}
        \item $f(x) = \cos x$, $f(0) = 1$
        \item $f'(x) = -\sin x$, $f'(0) = 0$
        \item $f''(x) = -\cos x$, $f''(0) = -1$
        \item $f^3 (x) = \sin x$, $f^3(0) = 0$
\end{enumerate}
Now substitute the derivatives into the equation for Taylor Polynomial,
\[
p_3(x) = f(0) + f'(1)x + \frac{f''(1)}{2!}x^2 + \frac{f^3(1)}{3!}x^3
\]
\[
= 1 + 0x - \frac{1}{2}x^2 + 0x^3 
\]
\[
1 - \dfrac{x^2}{2!} 
\]
Substitute $x=0.1$,
\[
\cos (0.1) \approx 1 - \dfrac{0.1^2}{2} = 0.9950
\]
The answer is C.

4. First find the high order derivatives of $\arcsin x$ at $x=0$

\begin{enumerate}
        \item $f(x) = \arcsin x$, $f(0) = 0$
        \item $f'(x) = (1-x^2)^{-\frac{1}{2}}$, $f'(0) = 1$
        \item $f''(x) = x(1-x^2)^{-\frac{3}{2}}$, $f''(0) = 0$
        \item $f^3 (x) = (1-x^2)^{-\frac{3}{2}} + 3x^2 (1-x^2)^{-\frac{5}{2}}$, $f^3(0) = 1$
\end{enumerate}
Now substitute the derivatives into the equation for Taylor Polynomial,
\[
p_3(x) = f(0) + f'(1)x + \dfrac{f''(1)}{2!}x^2 + \frac{f^3(1)}{3!}x^3
\]
\[
= 0 + 1x - \dfrac{0}{2}x^2 + \frac{1}{3!}x^3 
\]
\[
= x - \dfrac{x^3}{3!} 
\]
Substitute $x=0.1$,
\[
\arcsin (0.1) \approx 0.1 + \dfrac{0.1^3}{3!} = 0.1002
\]
The answer is C.

5. First find the high order derivatives of $\sqrt[3]{x}$ at $x=27$

\begin{enumerate}
        \item $f(x) = \sqrt[3]{x}$, $f(27) = 3$
        \item $f'(x) = \dfrac{1}{3}x^{-2/3}$, $f'(27) = \dfrac{1}{27}$
        \item $f''(x) = -\dfrac{2}{9}x^{-5/3}$, $f''(27) = -\dfrac{2}{2187}$
        \item $f^3 (x) = \dfrac{10}{27}x^{-8/3} $, $f^3(27) = \dfrac{10}{177147}$
\end{enumerate}
Now substitute the derivatives into the equation for Taylor Polynomial,
\[
p_3(x) = f(0) + f'(1)x + \dfrac{f''(1)}{2!}x^2 + \frac{f^3(1)}{3!}x^3
\]
\[
= 3 + \dfrac{1}{27}(x-27) - \dfrac{2}{2187}\dfrac{1}{2!}(x-27)^3 + \dfrac{10}{177147}\dfrac{1}{3!}(30-27)^3 
\]

Substitute $x=0.1$,
\[
\sqrt[3]{30} \approx  3 + \dfrac{1}{27}(30-27) - \dfrac{2}{2187}\dfrac{1}{2!}(30-27)^3 + \dfrac{10}{177147}\dfrac{1}{3!}(30-27)^3 =3.10725
\] 
The answer is C.
\end{document}