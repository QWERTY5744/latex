\documentclass{article}
\usepackage{graphicx} % Required for inserting images
\usepackage{amsmath,amssymb,amsthm}
\usepackage{physics}
\usepackage{graphicx,float}
\graphicspath{{images/}}
\usepackage[none]{hyphenat}
\usepackage{blindtext}
\usepackage{parskip}
\usepackage[letterpaper,top=3cm, left= 3cm,bottom=3cm]{geometry}
\usepackage{subcaption}
\numberwithin{equation}{section}

\title{Absolute Convergence and Conditional Convergence}
\author{Polaris}
\date{2025/07/17}

\begin{document}

\maketitle

Welcome to this guide on Absolute Convergence and Conditional Convergence, this article will guide you through the following:

What is Absolute Convergence?

What is Conditional Convergence?

\section{Absolute/Conditional Convergence}
Let $\sum a_n$ be an infinte series, if 
\begin{enumerate}
    \item $\sum \abs{a_n}$ and $\sum a_n$ both converges, then the series is \textbf{absolutely convergent}.
    \item $\sum a_n$ converges but $\sum \abs{a_n}$ diverge, then the series is \textbf{conditionally convergent}
\end{enumerate}

\subsection{Example Questions}
\begin{enumerate}
    \item Determine if the series converge absolutely, conditionally, or diverge
    \[
    \sum_{n = 1}^{\infty} \frac{(-1)^n}{\sqrt[3]{n}}
    \]
    Apply the Alternating series test to this series
    \begin{enumerate}
        \item $\displaystyle \lim_{n\to\infty}\frac{1}{\sqrt[3]{n}} = 0$
        \item The function is decreasing
    \end{enumerate}
    Which means by Alternating Series Test, this series converge. 
    
    By p-series test, $\displaystyle \sum_{n=1}^{\infty}  \frac{1}{\sqrt[3]{n}}$ diverge, which means the original series \textbf{converge conditionally}.

    \item Determine if the series converge absolutely, conditionally or diverge
    \[
    \sum_{n = 1}^{\infty} \frac{\sin n}{n^2}
    \]
    Note that this series is \textbf{not} an alternating series. By direct comparison test, we have:
    \[
        \sum_{n = 1}^{\infty} \frac{\sin n}{n^2} \leq \sum_{n = 1}^{\infty} \frac{1}{n^2}
    \]
    Meaning the original series \textbf{converge absolutely}
\end{enumerate}

\newpage
\section{Practice Problems}
Determine if the following is absolute convergent or conditional convergent 

1. $\displaystyle \sum_{n=1}^{\infty} \frac{n}{3^{n-1}}$

A. Absolutely convergent

B. Conditionally convergent

C. Divergent

2. $\displaystyle \sum_{n=1}^{\infty} \frac{(-1)^{n-1}}{3} \frac{1}{2^n}$

A. Absolutely convergent

B. Conditionally convergent

C. Divergent

3. $\displaystyle \sum_{n=1}^{\infty} (-1)^{n-1} \frac{1}{\ln(n+1)}$

A. Absolutely convergent

B. Conditionally convergent

C. Divergent

4. $\displaystyle \sum_{n=1}^{\infty} (-1)^{n+1} \frac{2^{n^2}}{n!}$

A. Absolutely convergent

B. Conditionally convergent

C. Divergent

5. $\displaystyle \sum_{n=2}^{\infty} (-1)^{n+1} \frac{1}{n\ln n}$

A. Absolutely convergent

B. Conditionally convergent

C. Divergent

\newpage
\section{Solution}
1. First apply the alternating series test to test the convergence of the original series.
\begin{enumerate}
    \item 
    \[
    \lim_{n\to\infty} \frac{n}{3^{n-1}} = \lim_{n\to\infty} \frac{1}{\ln 3 \cdot 3^{n-1}} = 0
    \]
    Here we applied L'Hopital's Rule
    \item Let $\displaystyle f(n) = \frac{n}{3^{n-1}}$, find the first order derivative of this function.
    \[
    f'(n) = \frac{1}{\left(3^{n-1}\right)^2} \left(3^{n-1} - n\ln 3 \cdot 3^{n-1}\right) = \frac{1 - n\ln 3}{3^{n-1}}
    \]
    $f'(n) < 0$ when $\displaystyle n > \frac{1}{\ln 3}$, which means $a_{n+1} < a_n$

    Both condition satisfy, the series is convergent.

    Then check if $\abs{\sum a_n}$ converge, apply the ratio test for convergence.
    \[
    \lim_{n\to\infty} \abs{\frac{a_{n+1}}{a_n}} = \lim_{n\to\infty} \abs{\frac{n+1}{3^n} \frac{3^{n-1}}{n}} = \lim_{n\to\infty} \frac{1}{3} \frac{n+1}{n} = \frac{1}{3} < 1
    \]
    Which means this series converges as well, the series is absolutely convergent, the answer is A.
\end{enumerate}

2. First apply the alternating series test to test the convergence of the original series.
\begin{enumerate}
    \item 
    \[
    \lim_{n\to\infty} \frac{1}{2^n} = 0
    \]
    \item Let $\displaystyle f(n) = \frac{1}{2^n} = 2^{-n}$, find the first order derivative of this function.
    \[
    f'(n) = -\ln 2 \cdot 2^{-n}
    \]
    $f'(n) < 0$ for all real numbers, which means $a_{n+1} < a_n$

    Both condition satisfy, the series is convergent.

    Then check if $\abs{\sum a_n}$ converge, apply the ratio test for convergence.
    \[
    \lim_{n\to\infty} \abs{\frac{a_{n+1}}{a_n}} = \lim_{n\to\infty} \abs{\frac{2^n}{2^{n+1}}} = \frac{1}{2} < 1
    \]
    Which means this series converges as well, the series is absolutely convergent, the answer is A.
\end{enumerate}

3. First apply the alternating series test to test the convergence of the original series.
\begin{enumerate}
    \item 
    \[
    \lim_{n\to\infty} \frac{1}{\ln(n+1)} = 0
    \]
    \item Let $\displaystyle f(n) = \frac{1}{\ln(n+1)} = \left(\ln(n+1)\right)^{-1}$, find the first order derivative of this function.
    \[
    f'(n) = -\frac{\left(\ln(n+1)\right)^{-2}}{n+1}
    \]
    $f'(n) < 0$ for all real numbers, which means $a_{n+1} < a_n$

    Both condition satisfy, the series is convergent.

    Then check if $\abs{\sum a_n}$ converge, here the ratio test cannot draw any conclusion. We turn our attention to comparison test.
    
    Let $\displaystyle a_n = \frac{1}{n}$ and $\displaystyle b_n = \frac{1}{\ln n}$. When $n > 1$, $n > \ln n$, meaning $\displaystyle \frac{1}{\ln n} > \frac{1}{n}$, note $a_n$ is a divergent harmonic series, meaning $b_n$ is also divergent.

    This means this series diverge, the series is conditionally convergent, the answer is B.
\end{enumerate}

4. First apply the alternating series test to test the convergence of the original series.
    \[
    \frac{2^{n^2}}{n!} = \frac{2^n\cdot 2^n \cdot 2^n ... \cdot 2^n}{1\cdot 2 \cdot 3 ... \cdot n}
    \]
Where there are $n$ $2^n$ multiplying each other, also $2^n > n$, thus each $2^n$ is greater that each of the natrual number in the denominator, as $n$ increase, the fraction will only become larger, meaning the limit is not $0$.

This means this series diverge, the answer is C

5.  First apply the alternating series test to test the convergence of the original series.
\begin{enumerate}
    \item 
    \[
    \lim_{n\to\infty} \frac{1}{n\ln n} = 0
    \]
    \item Let $\displaystyle f(n) = \frac{1}{n\ln n} = (n\ln n)^{-1}$, find the first order derivative of this function.
    \[
    f'(n) = -\frac{\ln n + 1}{(n\ln n)^2}
    \]
    $f'(n) < 0$ for all real numbers, which means $a_{n+1} < a_n$

    Both condition satisfy, the series is convergent.

    Then check if $\abs{\sum a_n}$ converge, applying the integral test would be the most convenient
    \[
    \int_{2}^{\infty} \frac{1}{x\ln x} dx 
    \]
    Let $u = \ln x$, $\displaystyle du = \frac{1}{x} dx$, the upper bound turns into $\ln 2$ and $\infty$, thus

    \[
    \int_{2}^{\infty} \frac{1}{x\ln x} dx = \int_{\ln 2}^{\infty} \frac{1}{u} du = \lim_{b\to\infty} \left(\ln b - \ln (\ln 2)\right) = \infty
    \]
    This means this series diverge, the series is conditionally convergent, the answer is B.
\end{enumerate}
\end{document}