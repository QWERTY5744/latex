\documentclass{article}
\usepackage{graphicx} % Required for inserting images
\usepackage{amsmath,amssymb,amsthm}
\usepackage{physics}
\usepackage{graphicx,float}
\graphicspath{{images/}}
\usepackage[none]{hyphenat}
\usepackage{blindtext}
\usepackage{parskip}
\usepackage[letterpaper,top=3cm, left= 3cm,bottom=3cm]{geometry}
\usepackage{subcaption}
\numberwithin{equation}{section}

\title{Representing Functions as Power Series}
\author{Polaris}
\date{2025/08/26}

\begin{document}
\maketitle
\section{Functions and Power Series}
As one see in the Taylor Series section, a function (like trigonometric function) can be represented as a infinite series, for example
\[
e^x = \sum_{n=0}^{\infty} \frac{x^n}{n!}
\]
\[
\sin x = \sum_{n=0}^{\infty} \frac{(-1)^nx^{2n+1}}{(2n+1)!}
\]
This hints us that perhaps all functions can be written as a infinite series. 

\subsection{Example Question}
1. Represent $f(x) = \displaystyle \int e^{-x^2}dx$ as a power series

This function cannot be obtained easily, since directly integrating $e^{-x^2}$ requires techniques that is not taught in Calc BC, however, recall that
\[
e^{-x^2} = \sum_{n=0}^{\infty} (-1)^{n+1}\frac{x^{2n}}{n!} = 1 - x^2 + \frac{x^4}{2!} - \frac{x^6}{3!} + \frac{x^8}{4!} - ...
\]
We can integrate this term by term, and obtain this function
\[
f(x) = \int e^{-x^2}dx = \int \sum_{n=0}^{\infty} (-1)^{n+1} \frac{x^{2n}}{n!} dx
\]
Notice that $(-1)^{n+1}$ and $n!$ are all constant, therefore we only need to worry about the $x$ term in the series.
\[
f(x) = \sum_{n=0}^{\infty} \frac{(-1)^{n+1}}{n!} \int x^{2n} dx = \sum_{n=0}^{\infty} (-1)^{n+1} \frac{x^{2n+1}}{(2n+1)n!}
\]

Just like functions can be represented as a power series, a power series can also be turned into a function.

2. Represent $\displaystyle f(x) = \sum_{n=0}^{\infty} 3x^n$, where $\left|x\right|<1$

In the section on Taylor series, we derived that 
\[
\sum_{n=0}^{\infty} x^n = \frac{1}{1-x}
\]
Therefore
\[
f(x) = \sum_{n=0}^{\infty} 3x^n = \frac{3}{1-x}
\]
\section{Practice Problems}
1. Represent $\arctan x$ as a power series

A. $\displaystyle \sum_{n=0}^{\infty} \frac{x^{2n+1}}{(2n+1)!}$

B. $\displaystyle \sum_{n=0}^{\infty} (-1)^{n} \frac{x^{2n+1}}{(2n+1)!}$

C. $\displaystyle \sum_{n=0}^{\infty} \frac{x^{2n}}{2n!}$

D. $\displaystyle \sum_{n=0}^{\infty} (-1)^{n} \frac{x^{2n}}{2n!}$

2. Represent $f(x) = \dfrac{12}{3+x}$ as a power series

A. $\displaystyle \sum_{n=0}^{\infty} \frac{(-1)^{n+1}4}{3^n}x^{2n}$

B. $\displaystyle \sum_{n=0}^{\infty} \frac{4}{3^n}x^{2n}$

C. $\displaystyle \sum_{n=0}^{\infty} \frac{4}{3^n}x^n$

D. $\displaystyle \sum_{n=0}^{\infty} \frac{(-1)^n 4}{3^n}x^n$

3. Represent $f(x) = 2^x$ as a power series

A. $\displaystyle \sum_{n=0}^{\infty} \frac{x^n}{n!}$

B. $\displaystyle \sum_{n=0}^{\infty} \frac{(\ln 2)^n}{n!}x!$

C. $\displaystyle \sum_{n=0}^{\infty} \frac{x^{n+1}}{(n+1)!}$

D. $\displaystyle \sum_{n=0}^{\infty} \frac{(\ln 2)^{n+1}}{(n+1)!}x^{n+1}$

\section{Solution}
1. Notice that 
\[
\int \frac{1}{1+x^2} dx = \arctan x+C
\]
Therefore we can apply term by term integration and find the power series of $\arctan x$, We developed that
\[
\sum_{n=0}^{\infty} x^n = \frac{1}{1-x}
\]
Therefore 
\[
\sum_{n=0}^{\infty} (-1)^{n+1} x^n = \frac{1}{1+x} 
\]
Changing $x$ to $x^2$, the series turns to
\[
\frac{1}{1+x^2} = \sum_{n=0}^{\infty} (-1)^{n+1} x^{2n}
\]
Therefore the power series one desired is
\[
\arctan x = \int \frac{1}{1+x^2} dx = \sum_{n=0}^{\infty} (-1)^{n+1} \frac{x^{2n+1}}{(2n+1)!}
\]
The answer is A.

2. Divide both the denominator and numerator by $3$
    \[
    \dfrac{12}{3+x} = \frac{4}{1-\left(-\dfrac{x}{3}\right)}
    \]
    Which looks very similar to the Taylor series of $\displaystyle \sum_{n=0}^{\infty} a_0 x^n = \frac{a_0}{1-x}$, therefore we can construct the power series
    \[
    \frac{12}{3+x} = \sum_{n = 0}^{\infty} 4\left(-\frac{x}{3}\right)^n = \displaystyle \sum_{n=0}^{\infty} \frac{(-1)^n 4}{3^n}x^n
    \]
    The answer is D.

3. First, recall the Taylor series of $e^x$
\[
e^x = \sum_{n=0}^{\infty} \frac{x^n}{n!}
\]
Notice that 
\[
a^x = e^{x \ln a}
\]
Therefore we can turn the Taylor series of $a^x$ into something related to $e^x$
\[
a^x = \sum_{n=0}^{\infty} \frac{(\ln a)^n}{n!}x^n
\]
When $a=2$, this series turns to the Taylor Series of $2^x$
\[
2^x = \sum_{n=0}^{\infty} \frac{(\ln 2)^n}{n!}x^n
\]
The answer is B.
\end{document}