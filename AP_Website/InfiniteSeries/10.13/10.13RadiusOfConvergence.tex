\documentclass{article}
\usepackage{graphicx} % Required for inserting images
\usepackage{amsmath,amssymb,amsthm}
\usepackage{physics}
\usepackage{graphicx,float}
\graphicspath{{images/}}
\usepackage[none]{hyphenat}
\usepackage{blindtext}
\usepackage{parskip}
\usepackage[letterpaper,top=3cm, left= 3cm,bottom=3cm]{geometry}
\usepackage{subcaption}
\numberwithin{equation}{section}

\title{Radius of Convergence}
\author{Polaris}
\date{2025/08/24}

\begin{document}

\maketitle
\section{Radius and Interval of Convergence of Power Series}
A Power Series takes in a form of
\[
    \sum_{n = 1}^{\infty}a_n (x-c)^n
\]
Where $c$ is the center of the series, since there is an extra $x$, for some $x$ the series might diverge, for example, consider a series where $a_n = 1$ and $c=0$:
\[
\sum_{n=1}^{\infty} x^n
\]
If $x\geq1$, this series will diverge to infinity, if $0<x<1$, this series will converge.

There are 3 posibilities of the convergence of a power series:
\begin{enumerate}
    \item Converges only at the center (every power series converge at center)
    \item Converges for some value within a finite distance to the center (we call this distance Radius of Convergence)
    \item Converge for all values
\end{enumerate}

\subsection{Interval of Convergence}
Interval of Convergence is a fancy word for all the values that make the series converge, it is an important parameter of an power series.
\begin{enumerate}
    \item Within the radius of convergence, the series will converge \textbf{absolutely}
    \item At the endpoint of the interval, the series could \textbf{converge absolutely, conditionally or diverge.}
\end{enumerate}
To determine the radius of convergence, use the ratio test

If 
\[
\rho = \lim_{n\to \infty} \abs{\frac{a_{n+1}}{a_n}}
\]
Where $a$ is the coefficient in front of the power term. Then radius of convergence of the series is $R = \dfrac{1}{\rho}$, if $\rho = \infty$, then $R=0$; if $\rho = 0$, $R=\infty$
\subsection{Example Questions}
\begin{enumerate}
\item Find the radius of convergence of this power series:
\[
\sum_{n=0}^{\infty}(x+3)^n
\]

By ratio test:
\[
    \lim_{n\to \infty} \abs{\frac{a_{n+1}}{a_n}} = \lim_{n\to\infty} \abs{\frac{1}{1}} = 1
\]
Thus the radius of convergence is 1.

To find the interval of convergence for this series, notice that the center is at $x=-3$, and the radius of convergence is $1$, it means that for all values with in $-2$ and $-4$, the series will converge.

To test for the endpoints, substitute the value to the series and test if it converge, for example, to determine the convergence of this series at $x=-2$, substitute $x=-2$ in
\[
\sum_{n=0}^{\infty} (-1)^n
\]
This is a geometric series with a common ratio of $r=-1$, which diverge, same logic applies to $x=-4$, therefore the interval of convergence for this series is $(-4,-2)$
\item Find the radius of convergence of this power series:
\[
\sum_{n=1}^{\infty}\frac{(-1)^n x^{n+1}}{n^2}
\]

By ratio test:
\[
\lim_{n\to \infty} \abs{\frac{n^2}{(n+1)^2}} = 1
\]
Therefore radius of convergence is 1.
\end{enumerate}

\newpage
\subsection{Practice Question}
Determine the radius of convergence of the following power series.

1. $\displaystyle \sum_{n=1}^{\infty} n! x^n$

A. $\infty$

B. $0$

C. $1$

D. $10$

2. $\displaystyle \sum_{n=1}^{\infty} \frac{(x-1)^2}{2^n \cdot n}$

A. $\dfrac{1}{2}$

B. $\infty$

C. $2$

D. $0$

3. $\displaystyle \sum_{n=1}^{\infty} \frac{(2x)^n}{n\cdot 3^n}$

A. $\dfrac{3}{2}$

B. $\dfrac{2}{3}$

C. $2$

D. $3$

\subsection{Solution}

1. By ratio test
\[
\rho = \lim_{n\to \infty} \left|\frac{a_{n+1}}{a_n}\right| = \lim_{n\to \infty} \left|\frac{(n+1)!}{n!}\right| = \infty
\]
Therefore the radius of convergence is $0$, the answer is B.

2. By ratio test
\[
\rho = \lim_{n\to\infty} \left|\frac{2^n \cdot n}{2^{n+1}\cdot (n+1)}\right| = \lim_{n\to\infty} \left|\frac{1}{2}\frac{n}{n+1}\right| = \frac{1}{2}
\]
Therefore the radius of convergence of $R = 2$, the answer is C.

3. By ratio test
\[
\rho = \lim_{n\to\infty} \left|\frac{2^{n+1}}{(n+1)\cdot 3^{n+1}} \frac{n\cdot 3^n}{2^n}\right| = \frac{3}{2}\lim_{n\to \infty} \left|\frac{n}{n+1}\right| = \frac{2}{3}
\]
Therefore the radius of convergence is $\dfrac{3}{2}$, the answer is A.
\end{document}