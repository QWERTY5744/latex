\documentclass{article}
\usepackage{graphicx} % Required for inserting images
\usepackage{amsmath,amssymb,amsthm}
\usepackage{physics}
\usepackage{graphicx,float}
\graphicspath{{images/}}
\usepackage[none]{hyphenat}
\usepackage{blindtext}
\usepackage{parskip}
\usepackage[letterpaper,top=3cm, left= 3cm,bottom=3cm]{geometry}
\usepackage{subcaption}
\numberwithin{equation}{section}

\title{Alternating Series Test for Convergence}
\author{Polaris}
\date{2025/07/11}

\begin{document}

\maketitle
Previously, we are all working with sequence with only positive number, now we will deal with sequence with both positive and negative number.

\section{Alternating Series}
We define alternating series as a series where the sign of terms switch from positive to negative, or negative to positive.
ex.
\[
    \sum_{n = 0}^{\infty} (-1)^n a_n \text{ and } \sum_{n = 1}^{\infty} \cos(n\pi) a_n
\]

\subsection{Convergence Test}
Let $a_n > 0$, the alternating series:
\[
    \sum_{n = 1}^{\infty} (-1)^n a_n \text{ and } \sum_{n = 1}^{\infty} (-1)^{n+1} a_n
\]
converge if the following two requirements are met:
\begin{enumerate}
    \item $\lim_{n\to \infty} a_n = 0$
    \item $a_{n+1} \leq a_n$ ($a_n$ is decreasing)
\end{enumerate}
\textbf{Do not use this test to test for divergence}

\subsection{Example Questions}
\begin{enumerate}
    \item Determine if the series converge or diverge:
    \[
        \sum_{n = 1}^{\infty} (-1)^{n+1}\frac{1}{n}
    \]
    Check the two requirements:
    \begin{enumerate}
        \item $\lim_{n\to\infty} \frac{1}{n} = 0$
        \item $a_{n+1} \leq a_n$ (this function is decreasing in $[1,\infty)$)
    \end{enumerate}
    Both requirements fit, meaning that this series \textbf{converge}, although this series look like the harmoic series.
\end{enumerate}

\newpage
\section{Practice Question}
Determine if the following series converge using the Alternating Series Convergence Test:

1. 
\[
\sum_{n=1}^{\infty} (-1)^n\frac{n^2}{n^2+5}
\]
A. Converge 

B. Cannot be determined by Alternating Series Convergence Test

2. 
\[
\sum_{n=2}^{\infty} \frac{\cos(n\pi)}{\sqrt{n}}
\]
A. Converge 

B. Cannot be determined by Alternating Series Convergence Test

3. 
\[
\sum_{n=0}^{\infty}(-1)^{n-1} \frac{n}{3^{n-1}}
\]
A. Converge 

B. Cannot be determined by Alternating Series Convergence Test

4. 
\[
\sum_{n=0}^{\infty}(-1)^{n-1} \frac{1}{\ln (n-1)}
\]
A. Converge 

B. Cannot be determined by Alternating Series Convergence Test

5. 
\[
\sum_{n=0}^{\infty}(-1)^{n-1} \frac{n}{\ln (n-1)}
\]
A. Converge 

B. Cannot be determined by Alternating Series Convergence Test
\section{Solutions}
1.
\[
\lim_{n\to \infty} \frac{n^2}{n^2+5}
= \lim_{n\to \infty} \frac{2n}{2n}
= 1
\]
Which means one cannot determine the convergence of this series. Although this series diverge because $\displaystyle \lim_{n\to \infty} (-1)^n \frac{n^2}{n^2+5}$ DNE

The answer is B.

2. First, recognize that $\cos (n\pi) = (-1)^n$, thus the original series become
\[
\sum_{n=2}^{\infty} (-1)^n \frac{1}{\sqrt{n}}
\]
Apply the test to this series:
\[
\lim_{n\to\infty} \frac{1}{\sqrt{n}}
= 0
\]
and $\displaystyle \frac{1}{\sqrt{n}}$ is decreasing, thus the series is convergent, the answer is A.

3. Apply the test to this limit:
\[
\lim_{n\to\infty} \frac{n}{3^{n-1}} 
=\lim_{n\to\infty} \frac{1}{\ln 3 \cdot 3^{n-1}}
=0
\]
and the series is decreasing, meaning the series converge, the answer is A.

4. Apply the test to the limit:
\[
\lim_{n\to\infty} \frac{1}{\ln(n-1)} = 0
\]
and the series is decreasing, meaning that the series converge, the answer is A.

5. Apply the test to the limit:
\[
\lim_{n\to \infty} \frac{n}{\ln(n-1)}
= \lim_{n\to\infty} \frac{1}{\frac{1}{n-1}}
= \lim_{n\to\infty} n-1
= \infty
\]
Which means this test is not applicable to this series, the convergece of the series cannot be determined by Alternating Series Convergence Test
\end{document}