\documentclass{article}
\usepackage{graphicx} % Required for inserting images
\usepackage{amsmath,amssymb,amsthm}
\usepackage{physics}
\usepackage{graphicx,float}
\graphicspath{{images/}}
\usepackage[none]{hyphenat}
\usepackage{blindtext}
\usepackage{parskip}
\usepackage[letterpaper,top=3cm, left= 3cm,bottom=3cm]{geometry}
\usepackage{subcaption}
\numberwithin{equation}{section}

\title{Integral Test}
\author{Polaris}
\date{2025/06/27}

\begin{document}
\section{Integral Test for convergence}
If $f$ is a \textbf{positive}, \textbf{continous} and \textbf{decreasing} for $x \geq m$, where $m \geq 1$, and the n-th term expression $a_n = f(x)$, then:

\begin{equation}
    \sum_{n = 1}^{\infty} a_n \text{ and } \int_{1}^{\infty} f(x) \mathrm{d}x
\end{equation}

both converge or diverge.

\subsection{Example questions}
\begin{enumerate}
    \item Evaluate this series:
    \[
        \sum_{n = 1}^{\infty} \frac{n}{n^2 + 1} 
    \]
    First, check for the three criteria: positive, coutinous, decreasing
    \begin{enumerate}
        \item Positive: obviously $f(x) = \frac{x}{x^2 + 1}$ is positive in $[1 , \infty)$
        \item Continous: the function is continous for all real number
        \item Decreasing: $f'(x) = -\frac{x^2 - 1}{(x^2 + 1)^2}$, and $f'(x) < 0$ when $x > 1$
    \end{enumerate}
    Now evaluate the indefinite integral:
    \[
        \begin{split}
            \int_{1}^{\infty} \frac{x}{x^2 + 1} \mathrm{d}x &= \lim_{b\to \infty} \int_{1}^{b} \frac{x}{x^2 + 1} \mathrm{d}x\\
            & = \frac{1}{2} \lim_{b\to \infty} (\ln \abs{b^2 + 1} - \ln \abs{1^2 + 1})\\
            & = \infty
        \end{split}
    \]
    Here a u-substitution of $u = x^2+1$ and $du = 2x$ is performed. 
    The integral diverges, meaning that the series also diverges

    \item Evaluate this series:
    \[
        \sum_{n = 1}^{\infty} \frac{1}{n^2 + 1}
    \]
    Let 
    \[
        a_n = f(x) = \frac{1}{x^2 + 1}
    \]
    Checking if the method work is omitted, but it does work
    Then we can construct and solve this improper integral:
    \[
        \begin{split}
            \int_{1}^{\infty} \frac{1}{x^2 + 1} \mathrm{d}x & = \lim_{b\to \infty} \int_{1}^{b} \frac{1}{x^2 + 1}\mathrm{d}x\\
            & = \lim_{b\to \infty} \arctan b - \arctan 1\\
            & = \frac{\pi}{2} - \frac{\pi}{4} = \frac{\pi}{4}\\
        \end{split}
    \]
    Meaning the series converge, \textbf{but the series doesn't necessarily converge to} $\pi/4$
\end{enumerate}

\section{Practice}
Determine whether the following series converge or diverge using the integral test.
\[
\sum_{n=1}^{\infty} \frac{1}{n^2}
\]
A. Converge 

B. Diverge 

C. Insufficient information

\[
\sum_{n=1}^{\infty} \sin n 
\]
A. Converge 

B. Diverge 

C. Insufficient information

\[
\sum_{n=2}^{\infty} \frac{1}{n \ln n}
\]

A. Converge 

B. Diverge 

C. Insufficient information

\[
\sum_{n=1}^{\infty} \frac{6n^2}{n^3+1}
\]

A. Converge 

B. Diverge 

C. Insufficient information

\[
\sum_{n=1}^{\infty} e^{-n}
\]

A. Converge 

B. Diverge 

C. Insufficient information
\section{Solution}
1. First check if the series fit the requirement of integral test

Positive: obviously $\displaystyle f(x) = \frac{1}{x^2}$ is positive in $[1 , \infty)$

Continous: the function is continous for all real number except at $x=0$

Decreasing: $\displaystyle f'(x) = -\frac{2}{x^3}$, and $f'(x) < 0$ when $x > 1$

Thus meaning we can apply the integral test to test the convergence of this series:

$\displaystyle \int_{1}^{\infty} \frac{1}{x^2}dx$
$\displaystyle = \lim_{b\to \infty}\int_{1}^{b} \frac{1}{x^2}dx$
$\displaystyle = \lim_{b\to \infty}-\frac{1}{x}\Big|_1^b$
$\displaystyle = -\lim_{b\to \infty} \frac{1}{b} + 1$
$\displaystyle = 1$

Which means this integral converge, the answer is A.

2. First check if the series meet the requirement of integral test:

Positive: $\displaystyle f(x) = \sin x$ is positive in some interval, but not positive in $[1,\infty)$

Continous: the function is continous for all real number 

Decreasing: $\displaystyle f'(x) = \cos x$, $f'(x) < 0$ for only certain $x$

This means the integral test cannot be applied to this test, the answer is C.

3. First check if the series meet the requirement of integral test:

Positive: $\displaystyle f(x) = \frac{1}{x\ln x}$ is positive in $[2,\infty)$

Continous: the function is continous in its domain

Decreasing: $\displaystyle f'(x) = \frac{\ln x +1}{(x\ln x)^2}$, where $f'(x) < 0$ when $x>1$.

This means we can apply the integral test to test the convergence of this series

$\displaystyle \int_{2}^{\infty} \frac{1}{x\ln x}dx$
$=\displaystyle \int_{2}^{\infty} \frac{1}{x} \frac{1}{\ln x}dx$
$=\displaystyle \int_{\ln 2}^{\infty} \frac{1}{u} du$
$=\displaystyle \infty$

The integral diverge, thus this series diverge. The answer is B.

4. First check if the series meet the requirement of integral test:

Positive: $\displaystyle f(x) = \frac{6x^2}{x^3+1}$ is positive in $[1,\infty)$

Continous: the function is continous in its domain

Decreasing: $\displaystyle f'(x) = \frac{-6x(x^3-2)}{(x^3+1)^2}$, where $f'(x) < 0$ when $x>\sqrt[3]{2}$.

This means we can apply the integral test to test the convergence of this series.

Let $u = x^3+1$, $du = 3x^2$, thus 

$\displaystyle \int_{1}^{\infty} \frac{2}{u}du$ 
$\displaystyle = 2\lim_{b\to \infty}\int_{1}^{b} \frac{1}{u}du$
$\displaystyle = 2\lim_{b\to \infty} \ln b \Big|_1^b$
$\displaystyle = 2 \lim_{b\to \infty} \ln b$
$\displaystyle = \infty$

5. First check if the series meet the requirement of integral test:

Positive: $\displaystyle f(x) = e^{-x}$ is positive in $[1,\infty)$

Continous: the function is continous in its domain

Decreasing: $\displaystyle f'(x) = -e^{-x}$, where $f'(x) < 0$ when $x>1$.

This means we can apply the integral test to test the convergence of this series.

Let $u = -x$, $du = -dx$, thus 

$\displaystyle \int_{1}^{\infty} e^{-x} dx$ 
$\displaystyle = -\lim_{b\to \infty}\int_{1}^{b} e^{u}du$
$\displaystyle = -\lim_{b\to \infty} e^u \Big|_1^b$
$\displaystyle = -\lim_{b\to \infty} e^{-x} \Big|_1^b$
$\displaystyle = -\lim_{b\to \infty} e^{-x} + 0$
$\displaystyle = \frac{1}{e}$

The integral converge, this means the series also converge, the answer is A.
\end{document}