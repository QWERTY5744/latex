\documentclass{article}
\usepackage{graphicx} % Required for inserting images
\usepackage{amsmath,amssymb,amsthm}
\usepackage{physics}
\usepackage{graphicx,float}
\graphicspath{{images/}}
\usepackage[none]{hyphenat}
\usepackage{blindtext}
\usepackage{parskip}
\usepackage[letterpaper,top=3cm, left= 3cm,bottom=3cm]{geometry}
\usepackage{subcaption}
\numberwithin{equation}{section}

\title{n-th Term Test}
\author{Polaris}
\date{2025/06/27}

\begin{document}

\maketitle
This article will guide you through how to preform a n-th Term Test to determine a convergence of a series.

\section{n-th Term Test}
\[
    \text{For an infinte series} \sum_{n = 1}^{\infty} a_n \text{ , if }\lim_{x\to \infty} a_n \neq 0 \text{ or the limit does not exist, then} \sum_{n = 1}^{\infty} a_n \text{ diverge}
\]
This theorem is the \emph{n-th term test}, it is important to point out that if the limit equals to $0$, this test \emph{cannot draw any conclusion}.
\[
    \text{If } \lim_{x\to \infty} a_n = 0  \text{, then} \sum_{n = 1}^{\infty} a_n \text{ don't have to converge}
\]
We will not go over the proof of this test, but we can intuitively understand this test, for a series to diverge, the n-th term of a sequence 
needs to be larger than $0$, otherwise there is a chance of this series converging, hence we require that the limit of n-th term is not equal to $0$.
\section{Example Questions}
State if the following series diverge or converge
\begin{enumerate}
    \item \[\sum_{n = 1}^{\infty} \left(\frac{n}{n + 100}\right)\]
    For this series, if $\displaystyle\lim_{n\to \infty} \frac{n}{n+100} \neq 0$, the series diverges:
    \[
        \lim_{n\to \infty} \frac{n}{n+100} = \lim_{n\to \infty} \frac{1}{1} = 1
    \]
    Which means the original series diverge.

    \item \[\sum_{n = 1}^{\infty} \left(\frac{1}{n}\right)^2\]
    Examine this limit: 
    \[
        \lim_{x\to \infty} \left(\frac{1}{n}\right)^2 = 0
    \]
    By the n-th term test, we have no way to prove that the series converge or diverge through this method(though the series converge, it is the Basel Problem)
\end{enumerate}

\section{Practice Problem}
State if the following series diverge or converge.
\begin{enumerate}
    \item \[\sum_{n=1}^{\infty}\frac{1}{n}\]
    \item \[\sum_{n=1}^{\infty} \left(\frac{e^n}{\ln n}\right)\]
    \item \[\sum_{n=1}^{\infty} \left(\sin \frac{1}{n}\right)\]
    \item \[\sum_{n=1}^{\infty} \left(n\sin \left(\frac{1}{n}\right)\right)\]
    \item \[\sum_{n=1}^{\infty} n\tan n\]
\end{enumerate}

\section{Answer Key}
\begin{enumerate}
    \item Apply the n-th term test for this series: \[\lim_{n\to\infty}\frac{1}{n} = 0\]
    Which means this test is inconclusive, we cannot draw any conclusion from this test.
    \item Apply the n-th term test for this series
    \begin{align*}
        \lim_{x\to \infty} \frac{e^x}{\ln x} &= \lim_{n\to \infty} \frac{e^x}{1/x}\\
        &= \lim_{x\to \infty} xe^x\\
        &= \infty
    \end{align*}
    Which means this series diverge.
    \item Apply the n-th term test for this series:
    \[\lim_{x\to \infty} \sin \frac{1}{x}\]
    We know that 
    \[
    -\frac{1}{x} \leq \sin \frac{1}{x} \leq \frac{1}{x}
    \]
    By squeeze theorem, we have 
    \begin{align*}
        \lim_{x\to \infty} -\frac{1}{x} \leq &\lim_{x\to \infty}\sin \frac{1}{x} \leq \lim_{x\to \infty} \frac{1}{x}\\
        0 \leq &\lim_{x\to \infty} \sin \frac{1}{x} \leq 0\\
    \end{align*}
    Which gives \[\lim_{x\to \infty} \sin \frac{1}{x} = 0\]

    This means this test is inconclusive for this series.
    \item Apply the n-th term test for this series:
    \begin{align*}
        \lim_{x\to \infty} x\sin \frac{1}{x} &= \frac{\sin \frac{1}{x}}{\frac{1}{x}}\\
        &= 1
    \end{align*}
    Which means this test is inconclusive for this series.
    \item Apply the n-th term test for this series:
    \[\lim_{x\to \infty} x\tan x = \text{DNE}\]
    Which means this series diverge.
\end{enumerate}
\end{document}