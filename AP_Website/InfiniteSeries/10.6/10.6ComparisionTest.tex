\documentclass{article}
\usepackage{graphicx} % Required for inserting images
\usepackage{amsmath,amssymb,amsthm}
\usepackage{physics}
\usepackage{graphicx,float}
\graphicspath{{images/}}
\usepackage[none]{hyphenat}
\usepackage{blindtext}
\usepackage{parskip}
\usepackage[letterpaper,top=3cm, left= 3cm,bottom=3cm]{geometry}
\usepackage{subcaption}
\numberwithin{equation}{section}

\title{Comparision Test for Convergence}
\author{Polaris}
\date{2025/07/05}

\begin{document}

\maketitle

\section{Direct Comparison Test for Convergence}
Let $a_n$ and $b_n$ be two infinte series, if $0 < a_n \leq b_n$ for all $n>0$:

\begin{enumerate}
    \item If $\displaystyle \sum_{n = 1}^{\infty}$ converge, then $\displaystyle \sum_{n = 1}^{\infty} a_n$ converge.
    \item If $\displaystyle \sum_{n = 1}^{\infty} a_n$ diverge, then $\displaystyle \sum_{n = 1}^{\infty} b_n$ diverge.
\end{enumerate}
\subsection{Example Questions}
\begin{enumerate}
    \item Determine the convergence of this series
    \[
        \sum_{n = 1}^{\infty} \frac{1}{2 + 3^n}
    \]
    Note that this series looks similar to this series, which is a geometric series that converges:
    \[
        \sum_{n = 1}^{\infty} \frac{1}{3^n}
    \]
    It can be proven that this holds true when $n>1$ (this expression holds true if $n>0$)
    \[
        \frac{1}{3^n} > \frac{1}{2 + 3^n}
    \]
    Which means the original series converges.

    \newpage
    \item Determine the convergence of this series:
    \[
        \sum_{n = 1}^{\infty} \frac{1}{5 + \sqrt{n}}
    \]
    This series look similar to this series:
    \[
        \sum_{n = 1}^{\infty} \frac{1}{\sqrt{n}} 
    \]
    Notice that 
    \[
        \frac{1}{5 + \sqrt{n}} < \frac{1}{\sqrt{n}}
    \]
    When $n > 1$. But if we try to apply the convergence test here, we notice that we cannot draw any conclusion,
    because if the second series diverge, the first one don't have to. We must switch to a new series to solve the problem:
    \[
        \sum_{n = 1}^{\infty}\frac{1}{n}
    \]
    Notice that 
    \[
        \frac{1}{5 + \sqrt{n}} > \frac{1}{n}
    \]
    When $n > 8$, this means we can apply the direct comparison test, we know that the second series diverge, 
    which means the original series \textbf{diverge}.

    We said before that to apply the direct comparision test, $0<a_n\leq b_n$ must hold true for all $n$, but this is a little bit too strict, in fact the requirement of $a_n \leq b_n$ only needs to met eventually.

    Consider two series $a_n$ and $b_n$, for $n < N$, $a_n > b_n$, for $n > N$, $a_n < b_n$, if $b_n$ is a convergent series, then by comparision test $a_n$ is also a convergent series, this is because

    \[
    \sum_{n=1}^{\infty} a_n = \sum_{n=1}^{N} a_n + \sum_{n=N+1}^{\infty} a_n
    \]
    Since $N$ is a finite number, $\displaystyle \sum_{n=1}^{N} a_n$ is also a finite number, meaning that the original series is convergent/divergent only if $\displaystyle \sum_{n=N+1}^{\infty} a_n$ converge/diverge. We can then apply the comparision test for this series to determine the convergence/divergence of this series.
\end{enumerate}

\section{Limit Comparison Test}

\[
    \text{If } a_n > 0 \text{ and } b_n > 0 \text{ and } \lim_{n\to \infty} \frac{a_n}{b_n} = L 
\]
where $L$ is finite and positive, then the series both converge or diverge. 

If $L=0$ and $b_n$ converges, then $a_n$ also converge.

If $L=\infty$ and $b_n$ diverge, then $a_n$ also diverge.

\subsection{Example Questions}
\begin{enumerate}
    \item Determine the convergence of this series:
    \[
        \sum_{n = 1}^{\infty} \frac{\sqrt{n}}{n^2 + 1}
    \]
    Let's examine this series:
    \[
        \sum_{n = 1}^{\infty} \frac{\sqrt{n}}{n^2} = \sum_{n = 1}^{\infty} \frac{1}{n^{\frac{3}{2}}}
    \]
    Which is a convergent p-series, now we apply the limit comparison test and use L'Hôpital's rule:
    \[
        \begin{split}
            \lim_{x\to \infty} \frac{\sqrt{n}}{n^2 + 1} \frac{n^2}{\sqrt{n}} & = \lim_{x\to \infty} \frac{n^2}{n^2 + 1}\\
            & = 1
        \end{split}
    \]
    Which means the original series \textbf{converges}

    \item Determine the convergence of this series:
    \[
        \sum_{n = 1}^{\infty} \frac{n2^n}{4n^3 + 1}
    \]
    Let's examine this series, to find its convergence, we use the n-th term test:
    \[
        \sum_{n = 1}^{\infty} \frac{2^n}{n^2}
    \]
    \[
    \begin{split}
        \lim_{x\to \infty} \frac{2^n}{n^2} & = \lim_{x\to \infty} \frac{\ln 2\cdot 2^n}{2n}\\
        & = \lim_{x\to \infty} \frac{\ln 2 \cdot \ln 2 \cdot 2^n}{2} > 0\\
    \end{split}
    \]
    Meaning that this series diverges, then we apply the limit comparison test:
    \[
        \begin{split}
            \lim_{x\to \infty} \frac{n\cdot 2^n}{4n^3 + 1} \frac{n^2}{2^n} & = \lim_{n\to \infty} \frac{n^3}{4n^3 + 1}\\
            & = \frac{1}{4}
        \end{split}
    \]
    Which means the original series \textbf{diverge}
\end{enumerate}

\section{Tricks and Tips at selecting the compared series}
You should see that in order to apply the comparison test, choosing the right series to compare is the key, here are some tricks for choosing the compared series:
\begin{enumerate}
    \item If $\displaystyle a_n = \frac{k_1}{k_2 + f(n)}$, where $f(n)$ is a function on $n$, try choosing the compared series as $\displaystyle b_n = \frac{1}{f(n)}$
    \item If $\displaystyle a_n = \frac{f(n) + k_1}{g(n) + k_2}$ and both $f(n)$ and $g(n)$ are polynomial functions, try choosing the compared series as $\displaystyle b_n = \frac{f_1(n)}{g_1(n)}$, where $f_1(n)$ is the highest term of $f(n)$, $g_1(n)$ is the highest term of $g(n)$
    \item If $\displaystyle a_n = \frac{f(n)g(n) + k_1}{h(n) + k_2}$, where $f(n)$ and $g(n)$ are polynomials while $g(n)$ is not, try choosing the compared series as $\displaystyle b_n = \frac{f_1(n)}{h_1(n)}\cdot g(n)$, where $f_1(n)$ is the highest term of $f(n)$, $h_1(n)$ is the highest term of $h(n)$
\end{enumerate}
However those are not firm guidelines you should follow, be creative when you do problems!

\section{Practice Problems}
Determine if the following series converge using the comparison test:

$\displaystyle \sum_{n=1}^{\infty} \frac{1}{5+2^n}$

A. Converge 

B. Diverge 

$\displaystyle \sum_{n=2}^{\infty} \frac{1}{n^2-1}$

A. Converge 

B. Diverge 

$\displaystyle \sum_{n=1}^{\infty} \frac{n^2+n}{n^3-n^2+n-1}$

A. Converge 

B. Diverge 

$\displaystyle \sum_{n=1}^{\infty} \frac{\ln n^n}{n^4}$

A. Converge 

B. Diverge 

$\displaystyle \sum_{n=1}^{\infty} \frac{4n}{(n+1)^3}$

A. Converge 

B. Diverge 

\section{Solution}

Consider this series:
\[
\sum_{n=1}^{\infty} \frac{1}{2^n}
\]
Which is a convergent geometric series

When $n>1$, this always holds true:
\[
\frac{1}{2^n} > \frac{1}{5+2^n}
\]

Meaning that the original series converge, the answer is A.

Consider this series
\[
\sum_{n=1}^{\infty} \frac{1}{n^2}
\]
Which is a convergent p-series, by limit comparison test:

\[
\lim_{n\to \infty} \frac{b_n}{a_n} = \lim_{n\to \infty} \frac{n^2-1}{n^2} = 1
\]
Which means the original series converge by limit comparison test. The answer is A.

Consider this series
\[
\sum_{n=1}^{\infty} \frac{1}{n}
\]
By limit comparison test
\[
\lim_{n\to \infty} \frac{b_n}{a_n} = \lim_{n\to \infty}\frac{n^2+n}{n^3 - n^2 + n -1} \frac{1}{\frac{1}{n}} = \lim_{n\to \infty}\frac{n^3+n^2}{n^3-n^2+n-1} = 1
\]
By limit comparison test the original series diverge, the answer is B.

The original series can be written as:
\[
\sum_{n=1}^{\infty} \frac{n\ln n}{n^4} = \sum_{n=1}^{\infty}\frac{\ln n}{n^3}
\]

Consider this series:
\[
\sum_{n=1}^{\infty} \frac{1}{n^2}
\]

\[
\lim_{n\to \infty} \frac{\ln n}{n^3} \frac{1}{\frac{1}{n^2}} = \lim_{n\to \infty} \frac{\ln n}{n} = \lim_{n\to \infty} \frac{1}{n} = 0
\]
By limit comparison test, the original series converge, the answer is A.

Consider this series:
\[
\sum_{n=1}^{\infty} \frac{1}{n^2}
\]
By limit comparison test
\[
\lim_{n\to \infty} \frac{4n}{(n+1)^3} \frac{1}{\frac{1}{n^2}} = \lim_{n\to \infty} \frac{4n^3}{(n-1)^3} = 4
\]
Meaning the original series converge, the answer is A.
\end{document}