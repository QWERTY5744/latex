\documentclass{article}
\usepackage{graphicx} % Required for inserting images
\usepackage{amsmath,amssymb,amsthm}
\usepackage{physics}
\usepackage{graphicx,float}
\graphicspath{{images/}}
\usepackage[none]{hyphenat}
\usepackage{blindtext}
\usepackage{parskip}
\usepackage[letterpaper,top=3cm, left= 3cm,bottom=3cm]{geometry}
\usepackage{subcaption}
\numberwithin{equation}{section}

\title{Geometric Series}
\author{Polaris}
\date{2025/04/20}

\begin{document}

\maketitle

\section{Definition}
Consider this sequence
\[
\text{{$3,6,12,24,48$...}}
\]
It is easy to see that there is a common factor of $2$ between every term.
Thus we can rewrite the sequence like this:
\[
\text{{$3\cdot 2^0, 3\cdot 2^1, 3\cdot 2^2, 3\cdot 2^3, 3\cdot 2^4$}}
\]
In general, the n-th term expression can be written as (where $n$ starts from 0):
\[
a_n = 3\cdot 2^n
\]
If we take the sum of all terms, we arrive at what we called \textbf{geometric series},
the quotient of two neighboring terms of a geometric series is a constant ($2$ in the previous example).

With this knowledge, we can generalize the n-th term expression given before:
\[
a_n = a_0 \cdot r^n
\]
Where $a_n$ is the n-th term, $a_0$ is the first term, $r$ is the \textbf{common ratio}

\section{Convergence of Geometric Series}
Consider a geometric sequence of
\[
\sum_{n=0}^{\infty} a_0 \cdot r^n
\]
This series diverge if $\abs{n} \geq 1$, converge if $0 < \abs{n} < 1$. If the series converge, it converge
to a value of 
\[
S = \sum_{n=0}^{\infty} a_0 \cdot r^n = \frac{a_0}{1-r}
\]
\subsection{Proof}
You don't need to know this for AP exam, this section is for the completeness of knowledge. Let $S$ be the number the series converge to
\begin{align*}
    S &= a + ar + ar^2 + ... + ar^n +...\\
    rS &= ar + ar^2 + ar^3 + ... + ar^n +...\\
    rS &= S - a\\
    (1-r)S &= a\\
    S &= \frac{a}{1-r}\\
\end{align*}

\subsection{Example questions}
State if the following geometric series diverge or find the value it converge to:
\begin{enumerate}
    \item \[\sum_{n = 0}^\infty 1.1^n \]
    since $1.1 > 1$, this series diverge

    \item \[\sum_{n = 1}^\infty \frac{3}{2^n} \]
    since $\frac{1}{2} < 1$, this series converges, by equation $(2.2)$:
    \[
    \sum_{n = 1}^{\infty} 3\left(\frac{1}{2}\right)^n = \frac{\frac{3}{2}}{1-\frac{1}{2}} = 3
    \]
\end{enumerate}
\section{Practice Problem}
Determine if the following series converge, if it converge, find the value it converge to

1. $\displaystyle \sum_{n = 1}^{\infty} \left(\frac{3}{2} \right)^n$

A. Converge to $3$

B. Converge to $\dfrac{1}{3}$

C.Diverge

2. $\displaystyle \sum_{n = 0}^{\infty} \left( -\frac{e}{\pi}\right)^n$

A. Converge to $\dfrac{\pi}{e}$

B. Converge to $-\dfrac{\pi}{e}$

C. Diverge

3. Consider this geometric series $4 + 3 + \dfrac{9}{4} + \dfrac{27}{16} + ...$, which value does this series converge to?

A. $16$

B. $\dfrac{16}{3}$

C. diverge

4. A ball is dropped from a height of $10 \text{ m}$, everytime it bounces back to $\dfrac{4}{5}$ of its original height, what is the total distance (both up and down) the ball travelled when it stops?

A. $90 \text{ m}$

B. $50 \text{ m}$ 

C. This value is infinite

\section{Solution}
1. The common ratio $r=\dfrac{3}{2}>1$, therefore this series diverge. The answer is C

2. Since $0 < \abs{\dfrac{-e}{\pi}} < 1$, the series converge to $\displaystyle \sum_{n = 0}^{\infty} \left( -\frac{e}{\pi}\right)^n = \frac{1}{1 - \frac{-e}{\pi}} = \frac{\pi}{e}$, the answer is A.

3. The general term of this series is $4\left(\dfrac{3}{4}\right)^n$, the common ratio $r = \dfrac{3}{4} < 1$, which means the series converge to $S = \dfrac{4}{1-\frac{3}{4}} = 16$, the answer is A.

4. Everytime, the ball will reach a height of $10 \left(\dfrac{4}{5}\right)^n$, where $n$ is the number of bounce. The total distance it bounce is thus 
\[
2 \sum_{n=0}^{\infty} 10 \cdot \left(\frac{4}{5}\right)^n - 10
\]
Mulitpling the series by $2$ is because the ball bounces both up and down, and subtract $10$ because it began to travel through falling, therefore the total distance travelled is $2 \dfrac{10}{1-\frac{4}{5}} - 10 = 90 \text{ m}$
\end{document}