\documentclass{article}
\usepackage{graphicx} % Required for inserting images
\usepackage{amsmath,amssymb,amsthm}
\usepackage{physics}
\usepackage{graphicx,float}
\graphicspath{{images/}}
\usepackage[none]{hyphenat}
\usepackage{blindtext}
\usepackage{parskip}
\usepackage[letterpaper,top=3cm, left= 3cm,bottom=3cm]{geometry}
\usepackage{subcaption}
\numberwithin{equation}{section}

\title{Lagrange Error Bound}
\author{assassin3552}
\date{2025/08/09}

\begin{document}
\section{Lagrange Error Bound}
Sometimes finding error is also an important part of calculations, Lagrange Error Bound provided a method to estimate the error.

Consider a function $f$ and its $n$th order Taylor Polynomial, we define the error between the two as such:
\[
    R_n(x) = f(x) - P_n(x)
\]
Where $P_n(x)$ is the Taylor Polynomial to nth degree, Lagrange proved that this error function can be written as:
\[
    R_n(x) = \frac{f^{(n+1)}(z)}{(n+1)!}(x-c)^{n+1}
\]
Where $z$ is a number within $x$ and $c$. Practically, it is often impossible to find $z$, therefore there needs to be some ways to apprximate it.

Let $M$ be a number that satisfy
\[
    \left|f^{(n+1)}(z) \right|\leq M
\]
Thus the maximum value of the error is
\[
    R_n(x) \leq M\frac{{\left|{(x-c)}^{n+1}\right|}}{(n+1)!}
\]
This offers a neat way to estimate the error.

In other words, $M$ is the maximum value of the derivative within the interval of $x$ and $c$.

Example 1:

Estimate the Lagrange error bound of $e^x$ and its $3$rd order Taylor polynomial centered at $x=0$:

The Lagrange error bound can be expressed as such
\[
R_3(x) = M \frac{\left|x-c\right|^{4}}{4!}
\]
Note the interval of estimation is not directly given, but can be figure out anyway, the Taylor polynomial is centered at $x=0$, and the apprximation ends at $x$, therefore the interval is $(0,x)$

Now one need to find $M$, the maximum value of the derivative within the interval $(0,x)$, the $n$th derivative of $e^x$ is always $e^x$, and $e^x$ is an increase function, which means the maximum value $M$ occurs at the end of the interval.
\[
\left|f^{n+1}(z) \right| \leq \left|f^{n+1}(x)\right| \leq \left|e^x\right|\leq M
\]
The Lagrange error bound turns to
\[
R_3(x) = \frac{e^x x^{4}}{4!}
\]

To understand this error, it means that the apprximated difference between the real function and apprximated function at point $x$ is $R_3(x)$

\newpage
\section{Practice Problems}
1. Estimate the Lagrange error bound of $\sin x$ and its $3$rd order Taylor polynomial centered at $x=0$ over the interval of $(0,\pi)$

A. $\dfrac{\pi x^4}{4!}$

B. $\dfrac{x^4}{4!}$

C. $\dfrac{\pi x^5}{5!}$

D. $\dfrac{x^5}{5!}$

2. Estimate the Lagrange error bound of $\ln(x+1)$ and its $3$rd order Taylor polynomial centered at $x=0$

A. $\dfrac{x^4}{4(x+1)^4}$

B. $\dfrac{x^4}{(x+1)^4}$

C. $\dfrac{x^3}{3(x+1)^3}$

D. $\dfrac{x^3}{(x+1)^3}$

\section{Solution}
1. The Lagrange error bound can be expressed as such
\[
R_3(x) = M \frac{\left|x-c\right|^{4}}{4!}
\]

Now one need to find $M$, the maximum value of the derivative within the interval $(0,\pi)$. The $4$th derivative of $\sin x$ is $\sin x$.
\[
\left|f^{4}(z) \right| \leq \left|\sin z \right| \leq 1\leq M
\]
The Lagrange error bound turns to
\[
R_3(x) = \frac{x^{4}}{4!}
\]

2. The Lagrange error bound can be expressed as such
\[
R_3(x) = M \frac{\left|x-c\right|^{4}}{4!}
\]
Now one need to find $M$, the maximum value of the derivative within the interval $(0,x)$. The $4$th derivative of $\ln(x+1)$ is $-\dfrac{6}{(x+1)^4}$.
\[
\left|f^{4}(z) \right| \leq \left|-\dfrac{6}{(x+1)^4}\right| \leq M
\]
The Lagrange error bound turns to
\[
R_3(x) = \dfrac{6}{(x+1)^4}\dfrac{x^{4}}{4!} = \frac{x^4}{4(x+1)^4}
\]
\end{document}