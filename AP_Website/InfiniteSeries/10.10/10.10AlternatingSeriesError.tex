\documentclass{article}
\usepackage{graphicx} % Required for inserting images
\usepackage{amsmath,amssymb,amsthm}
\usepackage{physics}
\usepackage{graphicx,float}
\graphicspath{{images/}}
\usepackage[none]{hyphenat}
\usepackage{blindtext}
\usepackage{parskip}
\usepackage[letterpaper,top=3cm, left= 3cm,bottom=3cm]{geometry}
\usepackage{subcaption}
\numberwithin{equation}{section}

\title{Absolute Convergence and Conditional Convergence}
\author{Polaris}
\date{2025/07/25}

\begin{document}

\maketitle

\section{Alternating Series Error Bound}
Consider an alternating series:
\[
    \sum_{n = 1}^{\infty} (-1)^{n+1}\frac{1}{n}
\]
This series converge, but how far is the first n-th (say 5) terms from the actual convergence value? 
We introduce the Alternating Series Error Bound Theorem:
\begin{equation}
    \text{If an alternating series converge, then } \sigma = \abs{S - S_n} \leq \abs{a_{n+1}} 
\end{equation}
Where $\sigma$ is the error.

\subsection{Intuitive Understanding}
This section will introduce an intuitive understanding (not rigorius proof) to the theorem introduced earlier. Consider this series:
\[
\sum_{n=0}^{\infty} (-1)^n a_n = a_0 - a_1 + a_2 - a_3 + a_4 - a_5 + ... = L
\]
Let the sum of the first 4 terms of this series be $S_n$, thus
\[
\sum_{n=0}^{\infty} (-1)^n a_n = S_n + a_4 - a_5 + ... = L
\]
Since this series converge, $a_{n+1} < a_n$, which means that $\abs{a_4} > \abs{-a_5 + a_6 - ...}$, or $\abs{a_4} > \abs{\sigma - a_4}$thus error $\sigma = \abs{a_4 - a_5 + a_6 - ...} < 2\abs{a_4}$, meaning that
\[
S_n - a_{n+1} \leq L \leq S_n + a_{n+1}
\]
\subsection{Example}
\begin{enumerate}
    \item Estimate the convergence value of this series (give an estimation on upper and lower bound):
    \[
    \sum_{n = 1}^{\infty} (-1)^{n+1}\frac{1}{n}
    \]
    Let's first add up the first 5 term of the series:
    \[
        1 - \frac{1}{2} + \frac{1}{3} - \frac{1}{4} + \frac{1}{5} = \frac{47}{60}
    \]
    By the introduced theorem, we have $\sigma \leq \abs{a_6}$, or $\sigma \leq \abs{\dfrac{1}{6}}$, meaning that:
    \[
    \frac{47}{60} - \frac{1}{6} \leq \sum_{n = 1}^{\infty} (-1)^{n+1}\frac{1}{n} \leq \frac{47}{60} + \frac{1}{6}
    \]
    \[
        \frac{37}{60} \leq \sum_{n = 1}^{\infty} (-1)^{n+1}\frac{1}{n} \leq \frac{19}{20} 
    \]
    This series actually converge to $\ln(2) \approx 0.693$, which is with in the error bound
\end{enumerate}

\section{Practice Problems}
Estimate the error bound of the following series using the first 5 terms

1. $\displaystyle \sum_{n=1}^{\infty} (-1)^n \frac{1}{n^2}$

A. $\displaystyle -\frac{3019}{3600} \leq \sum_{n = 1}^{\infty} (-1)^{n}\frac{1}{n^2} \leq -\frac{2731}{3600} $

B. $\displaystyle -\frac{3119}{3600} \leq \sum_{n = 1}^{\infty} (-1)^{n}\frac{1}{n^2} \leq -\frac{973}{1200} $

C. This series diverge

2. $\displaystyle \sum_{n=2}^{\infty} (-1)^n \frac{n}{\ln n}$

A. $\displaystyle 11.360 \leq \sum_{n=2}^{\infty} (-1)^n \frac{n}{\ln n} \leq 18.554$

B. $\displaystyle 8.260 \leq sum_{n=2}^{\infty} (-1)^n \frac{n}{\ln n} \leq 14.957 $

C. This series diverge

3. $\displaystyle \sum_{n=1}^{\infty} (-1)^n\frac{n}{e^n}$

A. $\displaystyle -0.222 \leq \sum_{n=1}^{\infty} (-1)^n\frac{n}{e^n} \leq -0.192$

B. $\displaystyle -0.207 \leq \sum_{n=1}^{\infty} (-1)^n\frac{n}{e^n} \leq -0.140$

C. This series diverge
\newpage
\section{Solution}
1. First apply the convergence test for this series, the limit is 0 and the function is decreasing.

Find the sum of the first 5 terms
\[
\sum_{n=1}^{5} (-1)^{n}\frac{1}{n^2} = -\frac{3019}{3600}
\]
Thus the error turns to 
\[
-\frac{3019}{3600}- \frac{1}{36} \leq \sum_{n = 1}^{\infty} (-1)^{n+1}\frac{1}{n} \leq -\frac{3019}{3600} + \frac{1}{36}
\]
Which is equivalent to 
\[
-\frac{3119}{3600} \leq \sum_{n = 1}^{\infty} (-1)^{n}\frac{1}{n^2} \leq -\frac{973}{1200}
\]
The answer is B. 

This series converge to $\displaystyle \sum_{n=1}^{\infty} (-1)^n \frac{1}{n^2} = -\frac{\pi^2}{12}$, which is within the error bound. Finding the convergent value of this series can be neatly solved with the Basel Problem.

2. First apply the convergence test for this series,
\[
\lim_{n\to \infty} \frac{n}{\ln n} = \lim_{n\to \infty} \frac{1}{1/n} = \infty
\]
Here we applied the L'Hopital Rule, the limit DNE, meaning the series diverge.

The answer is C.

3. First apply the convergence test for this series, the limit is 0 and the function is decreasing.

Find the sum of the first 5 terms
\[
\sum_{n=1}^{5} (-1)^n\frac{n}{e^n} = -0.207
\]
Thus the error turns to 
\[
-0.207- \frac{6}{e^6} \leq \sum_{n = 1}^{\infty} (-1)^{n+1}\frac{1}{n} \leq -0.207 + \frac{6}{e^6}
\]
Which is equivalent to 
\[
-0.222 \leq \sum_{n=1}^{\infty} (-1)^n\frac{n}{e^n} \leq -0.192
\]
The answer is A. This series actually converge to $-\dfrac{e}{(e+1)^2} \approx -0.197$, which is within the error bound.
\end{document}