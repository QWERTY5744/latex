\documentclass{article}
\usepackage{graphicx} % Required for inserting images
\usepackage{amsmath,amssymb,amsthm}
\usepackage{physics}
\usepackage{graphicx,float}
\graphicspath{{images/}}
\usepackage[none]{hyphenat}
\usepackage{blindtext}
\usepackage{parskip}
\usepackage[letterpaper,top=3cm, left= 3cm,bottom=3cm]{geometry}
\usepackage{subcaption}
\numberwithin{equation}{section}

\title{Celestial Mechanics}
\author{Polaris}
\date{2025/11/06}

\begin{document}
\maketitle

\section{Polar Coordinate}
Polar coordinates are convienent when dealing with celestial mechanics problem, the key here is to find the velocity and acceleration in polar coordinate.

Consider a radial direction, denoted as $\mathbf{r}$; a tangential direction, denoted as $\boldsymbol{\theta}$. On each direction, the vectors can be written as
\begin{equation}
    \begin{cases}
        \mathbf{r} = r \hat{r}\\
        \boldsymbol{\theta} = \theta \hat{\theta}\\
    \end{cases}
\end{equation}

\subsection{Velocity}
To find the tangential and radial velocity. Recall that $\hat{r} = \cos \theta \hat{x} + \sin \theta \hat{y}$, and $\hat{\theta} = -\sin \theta \hat{x} + \cos \theta \hat{y}$ (this is because the two vectors are perpendicular to each other). Where $\hat{x}$ and $\hat{y}$ are base vectors in a Cartesian coordinate system.
\begin{align*}
    \derivative{t}\hat{r} &=  \derivative{t} (\cos \theta \hat{x} + \sin \theta \hat{y})\\
    &= \sin \theta \dot{\theta} \hat{x} + \cos \theta \dot{\theta} \hat{y}\\
\end{align*}
Therefore
\begin{equation}
    \boxed{\dot{\hat{r}} = \dot{\theta} \hat{\theta}}
\end{equation}
Similarly, one can find $\dot{\hat{\theta}}$.
\begin{align*}
    \derivative{t}\hat{\theta} &= \derivative{t} (-\sin \theta \hat{x} + \cos \theta \hat{y})\\
    &= -\cos \theta \dot{\theta} \hat{x} - \sin \theta \dot{\theta} \hat{y}\\
\end{align*}
Therefore 
\begin{equation}
    \boxed{\dot{\hat{\theta}} = -\dot{\theta} \hat{r}}
\end{equation}

Velocity is therefore 
\begin{align*}
    \mathbf{v} = \derivative{t}\mathbf{r} &= \derivative{t}(r\hat{r})\\
    &= r \dot{\hat{r}} + \hat{r} + \hat{r}\derivative{t}r\\
    &= r\dot{\theta} \hat{\theta} + \dot{r}\hat{r}
\end{align*}
Here $v_r = \dot{r}$ is the radial component, $v_\theta = r\dot{\theta}$ is the tangential component.

\subsection{Acceleration}
Diffrientiate velocity against time to find acceleration:
\begin{align*}
    \mathbf{a} = \derivative{t}\mathbf{v} &= \derivative{t} (r\dot{\theta} \hat{\theta} + \dot{r}\hat{r})\\
    &= \hat{\theta} \derivative{t} (r\dot{\theta}) + r\dot{\theta} \derivative{t} \hat{\theta} + \hat{r} \derivative{t}\dot{r} + \dot{r} \derivative{t} \hat{r}\\
    &= (r\ddot{\theta} + \dot{r} \dot{\theta})\hat{\theta} - r\dot{\theta} \dot{\theta} \hat{r} + \ddot{r} \hat{r} + \dot{r} \dot{\theta} \hat{\theta}\\
    &= (\ddot{r} - r\dot{\theta}^2) \hat{r} + (\dot{r}\dot{\theta} + r\ddot{\theta}) \hat{\theta}
\end{align*}
Here $a_r = \ddot{r} - r\dot{\theta}^2$ is the radial component, $a_\theta = \dot{r} \dot{\theta} + r \ddot{\theta}$ is the tangential component.

\section{Equation of Orbit}
By chain rule
\[
\derivative{r}{\theta} = \derivative{r}{t} \derivative{t}{\theta} = \frac{v_r}{v_\theta / r} = r \frac{v_r}{v_\theta}
\]
By the conservation of angular momentum and conservation of energy
\begin{equation}
    \begin{cases}
        L = mrv_\theta\\
        E = \dfrac{1}{2} m (v_r^2 + v_\theta^2) - \dfrac{GMm}{r}
    \end{cases}
\end{equation}
Therefore tangential and radial velocity can be given as
\begin{equation}
    \begin{cases}
        v_\theta = \dfrac{L}{mr}\\

        v_r = \sqrt{\dfrac{2E}{m} + \dfrac{2GM}{r} - \dfrac{L^2}{m^2 r^2}}
    \end{cases}
\end{equation}

Thus the differential equation can be written as
\begin{equation}
    \derivative{r}{\theta} = \frac{mr^2}{L} \sqrt{\frac{2E}{m} + \frac{2GM}{r} - \frac{L^2}{m^2 r^2}}
\end{equation}
Which has a solution of 
\begin{equation}
    r(\theta) = \frac{p}{1+e\cos \theta}
\end{equation}
This is the solution of a conic section. $p$ is the semi latus rectum, $p = \dfrac{L^2}{GMm^2}$; $e$ is the eccentricity; $\displaystyle e = \sqrt{1+\frac{2EL^2}{G^2 M^2 m^3}}$.

\section{Laplace-Runge-Lenz Vector}
Laplace-Runge-Lenz Vector (LRL vector for short) is a constant vector in celestial mechanics. 

Consider the equation of motion of a body under gravitational attraction, let $\mu = GM$:

\begin{equation}
    \ddot{\mathbf{r}} = - \frac{\mu}{r^3} \mathbf{r}
\end{equation}
Let $\mathbf{h} = L/m$, cross product $\mathbf{h}$ on both sides.
\[
\ddot{\mathbf{r}} \times \mathbf{h} = -\frac{\mu}{r^3} \mathbf{r} \times \mathbf{h}
\]
Notice that 
\begin{align*}
    \derivative{t} (\dot{\mathbf{r}} \times \mathbf{h}) = \derivative{t} \mathbf{\dot{r}} \times \mathbf{h} + \dot{\mathbf{r}} \derivative{t}\mathbf{h} = \ddot{\mathbf{r}} \times \mathbf{h}
\end{align*}

The right hand side of the equation can be written as 
\begin{align*}
    - \frac{\mu}{r^3} \mathbf{r} \times \mathbf{h} &= - \frac{\mu}{r^3} \mathbf{r} \times (\mathbf{r} \times \mathbf{\dot{r}})\\
    &= - \frac{\mu}{r^3} (\mathbf{r} (\mathbf{r} \cdot \mathbf{\dot{r}}) - \dot{\mathbf{r}} (\mathbf{r} \cdot \mathbf{r}))\\
\end{align*}
Notice that 
\[
\derivative{t}(\mathbf{r} \times \mathbf{r}) = \derivative{t}(r^2) = 2r\dot{r}
\]
and 
\[
\derivative{t}(\mathbf{r} \times \mathbf{r}) = \mathbf{r} \derivative{t} \mathbf{r} + \mathbf{r} \derivative{t} \mathbf{r} = 2\mathbf{r} \mathbf{\dot{r}}
\]
Therefore one can conclude $r\dot{r} = \mathbf{r} \dot{\mathbf{r}}$

Return to the original equation above, one get 
\begin{align*}
    &= - \frac{\mu}{r^3} (\mathbf{r} (\mathbf{r} \cdot \mathbf{\dot{r}}) - \dot{\mathbf{r}} (\mathbf{r} \cdot \mathbf{r}))\\
    &= - \frac{\mu}{r^3} (\mathbf{r} r\dot{r} - \dot{\mathbf{r}} r^2)\\
    &= -\frac{\mu}{r^2} (\mathbf{r} \dot{r} - \mathbf{\dot{r}}r)\\
    &= - \frac{\mu}{r^2} \derivative{t} \left(\frac{\mathbf{r}}{r}\right)
\end{align*}
Therefore 
\begin{align*}
    \derivative{t} (\dot{\mathbf{r}} \times \mathbf{h}) &= \mu \derivative{t} \left(\frac{\mathbf{r}}{r}\right)\\
    \mathbf{\dot{r}} \times \mathbf{h} - \mu \frac{\mathbf{r}}{r} &= C
\end{align*}
We called the above vector Laplace-Runge-Lenz vector, a cosntant vector in celestial mechanics.
\begin{equation}
    \boxed{\mathbf{\dot{r}} \times \mathbf{h} - \mu \frac{\mathbf{r}}{r} = C}
\end{equation}
The eccentricity of the orbit with a given Laplace-Runge-Lenz vector is 
\begin{equation}
    \boxed{e = \frac{\abs{\mathbf{C}}}{\mu}}
\end{equation}

\section{Example Problems}
\subsection{Free Fall}
Consider two bodies of mass with mass $M$ and $m$ with a distance $r$, where $M \gg m$. Determine the time it takes for the smaller body to fall to the larger body.

\textbf{Solution}: Since $M\gg m$, the larger body can be considered as stationary. The velocity of the smaller object is $v = \displaystyle dx/dt$. The mechanical energy of the system is 
\[
\epsilon = \frac{1}{2} v^2 - \frac{GM}{x}
\]
Where $x$ is the distance to the central object.

Initially, everything is at rest, so the mechanical energy of the system is the graviational potential energy of the system. Which means
\[
\epsilon = \frac{1}{2}v^2 - \frac{GM}{x} = -\frac{GM}{r}
\]
Rearranging the equation, notice that the distance is decreasing, hence the negative sign
\[
\frac{dx}{dt} = -\sqrt{2GM \left(\frac{r-x}{rx}\right)}
\]
Rearrange again and integrate on both side
\[
\int_r^0 \sqrt{\frac{x}{r-x}} dx= \int_{0}^{T} -\sqrt{\frac{2GM}{r}} dt
\]
Preform a trig sub on the LHS. Let $x = r \sin^2 \theta$, thus $dx = 2r \sin x \cos x d\theta$
\begin{align*}
    \int_{0}^{r} \sqrt{\frac{r \sin^2 \theta}{r \cos^2 \theta}} 2r\sin x \cos x d\theta &= \sqrt{\frac{2GM}{r}}T\\
    \int_{0}^{\frac{\pi}{2}} \frac{\sin \theta}{\cos \theta} \sin \theta \cos \theta d\theta = \sqrt{\frac{GM}{4r^3}} T\\
    \int_{0}^{\frac{\pi}{2}} \sin^2 d\theta = \sqrt{\frac{GM}{4r^3}} T\\
\end{align*}
This gives the final result
\begin{equation}
    T = \pi \sqrt{\frac{r^3}{8GM}}
\end{equation}
Notice that this can also be derived from Kepler's Third Law. The trajectory of $m$ can be seen as an ellipse whose $e \to 1$, therefore, $2a = d$ and $2T = P$. 

\section{Stability and completeness of an orbit}
Consider a central force with the equation of 
\begin{equation}
    F(r) = -\frac{k}{r^n}
\end{equation}
Consider a object of mass $m$ moving around the central object with a circular orbit of radius $r_0$. Radially, the equation of motion is 
\[
m\left(\ddot{r} - r\dot{\theta}^2\right) = -\frac{k}{r^n}
\]
Now, consider a tiny wobble such that the object deviates from the original orbit. Radially, the deviation is $x$, which means 
\[
\begin{cases}
    r = r_0 + x\\
    \ddot{r} = \ddot{x}\\
\end{cases}
\]
Therefore, the radial equation of motion turns into
\[
\ddot{x} - (r_0 + x) \dot{\theta}^2 = -\frac{k}{m(r_0+x)^n}
\]
Using the small value approximation of $(1+x)^n \approx 1 + nx$ when $x \ll 1$, the equation turns into
\[
\ddot{x} - r\dot{\theta}^2 = -\frac{k}{mr_0^2} \left(1 - n \frac{x}{r_0}\right)
\]
For a circular orbit, $\ddot{r} = 0$, which means
\[
\dot{\theta}^2 = \frac{k}{mr_0^{n+1}} \text{ and } \dot{\theta} = \frac{h}{r_0^2}
\]
This means that for a circular
\[
h^2 = \frac{k}{m} r_0^{3-n}
\]
This will be useful in a second.

The equation of motion can be further simplified by substituting $\dot{\theta}$
\[
\ddot{x} - \frac{h^2}{(r_0+x)^3} = - \frac{k}{mr_0^n} \left(1 - \frac{nx}{r_0}\right)
\]
Since the deviation is small, $h^2$ in the circular orbit is nearly identical to the $h^2$ in the deviated orbit. Substituting $h^2$ inside
\[
\ddot{x} - \frac{k}{m} r_0^{3-n} \frac{1}{r^3_0} \left(1-3\frac{x}{r_0}\right) = - \frac{k}{mr_0^n} \left(1-\frac{nx}{r_0}\right)
\]
Move the RHS to the LHS and do some simplification
\[
\ddot{x} - \frac{k}{mr_0^n} + \frac{k}{mr_0^n} \frac{3x}{r_0} + \frac{k}{mr_0^2} - \frac{knx}{mr_0^{n+1}} = 0
\]
Finally, we obtain the equation of motion of SHM
\[
\ddot{x} + \frac{k}{m r_0^{n+1}} \left(3-n\right) x = 0
\]
The radial angular velocity is therefore 
\begin{equation}
    \omega_r = \sqrt{\frac{k}{m r_0^{n+1}} (3-n)}
\end{equation}
Which only make sense when $3-n > 0$. In other words, the orbit is stable when $n < 3$

The orbital angular velocity is 
\begin{equation}
\omega_{\theta} = \dot{\theta} = \sqrt{\frac{k}{mr_0^{n+1}}}
\end{equation}

For an orbit to be closed, after $n$ radial wobbles, 1 revolution needs to be completed, or 
\[
\frac{\omega_r}{\omega_\theta} = \sqrt{3-n}
\]
Which means for an orbit to be closed, $\sqrt{3-n}$ must be a integer. 

It just so happens that $n=2$ (which corresponds to gravity) satisfy both the conditions shown. This explains why the orbit of planets are closed.

However, due to effects like relativity and tides and even the shape of the central star, the total force on the orbiting object might not exactly equals to $2$. This explains the precession of the orbits of planets like Mercury.
\end{document}