\documentclass{article}
\usepackage{graphicx} % Required for inserting images
\usepackage{amsmath,amssymb,amsthm}
\usepackage{physics}
\usepackage{graphicx,float}
\graphicspath{{images/}}
\usepackage[none]{hyphenat}
\usepackage{blindtext}
\usepackage{parskip}
\usepackage[letterpaper,top=3cm, left= 3cm,bottom=3cm]{geometry}
\usepackage{subcaption}
\numberwithin{equation}{section}

\begin{document}
\section{Estimate the solar core number density}
Assume that the Sun is in static equalibrium, and the solar core is composed of ideal gas, 
estimate the solar core number density of particles and compare it with the number density of Earth's atmosphere.
The temperature of solar core is $T_s = 10^7K$

Hint: The hydrostatic equation due to gravity is:
\[
\frac{dp}{dr}=-\frac{GM(r)\rho(r)}{r^2}
\]
The left hand side of the equation is a derivative, but it can be approximated (meaning that you can neglect some coefficients) as such:
$dp$ is the  difference in pressure from the core to surface, 
$dr$ is the difference of radius from core to surface. $M$ is the mass of the Sun, 
$\rho$ is the average density of the Sun.

\newpage
Solution:

Rewrite the hydrostatic equation given:
\[
\frac{P_c - P_s}{R_c - R_s} \approx -\frac{GM\overline{\rho}}{R^2}
\]
Notice that pressure on the surface is 0 and radius in the center is 0, hence:
\[
\frac{P_c}{-R} \approx -\frac{GM\overline{\rho}}{R^2}
\]
The average density of the Sun is:
\[
\overline{\rho} = \frac{M}{\frac{4}{3}\pi R^3}
\]
Subsitute this into the equation derived ealier:
\[
\frac{P_c}{R} \approx \frac{GM}{R^2}\frac{M}{\frac{4}{3}\pi R^3}
\]
Thus the pressure in the core is:
\[
P_c \approx \frac{GM^2}{R^4} = 1.135\cross10^{15}Pa
\]

By Ideal Gas Law:
\[
PV = Nk_BT
\]
The number density of particles is:
\[
n_s = \frac{P}{k_BT} = 8.221\cross10^{30}/m^3
\]
Assume that the Earth atmosphere is composed of ideal gas, the number density of Earth's atmosphere is:
\[
n_E = \frac{P_e}{k_BT_E} = 2.414 \cross 10^{25}/m^3
\]
Compare the two numbers:
\[
\frac{n_s}{n_E} = 340555
\]

\newpage
\section{Supernovas and blackholes}
Blackholes are celestial objects such that light cannot even escape its gravitational pull. 
It is believed that blackholes form because massive stars collapse under gravity in their late stage of stellar evolution.

The minimum mass of a blackhole is given by \textbf{Tolman-Oppenheimer-Volkoff limit}, which is around 3 solar mass. 
This mass corresponds to a star mass of around 20 solar mass. In other word, a star needs to be at least 20 times more massive than the sun in order to form a blackhole.

Consider a star with a mass of 20 solar mass:

\begin{enumerate}
    \item A massive star will end its life with a supernova explosion, assume this explosion last 1 minute, and all mass of the star is either converted to energy or remain as a blackhole,
    find the average luminosity of this explosion.

    \item After the explosion, a blackhole is form. What is the Schwarzschild radius of the blackhole?
    
    \item In Newtonian mechanics, nothing can escape the graviational pull of blackholes, but after considering quantum effects, photons can actually escape blackholes,
    when the photons escape from blackholes, their wavelength is approximately the same as Schwarzschild radius, the temperature of a photon can be given as:
    \[
    T = \frac{E}{k_B}
    \]
    Where $E$ is the energy of the photon and $k_B$ is the Boltzmann constant, find the temperature of the photon that escaped the blackhole formed from the supernova explosin mentioned ealier.

    \item If blackholes emit photon in the previously calculated way, find the radiating power of the blackhole.

    \item At what rate does the blackhole lose mass?
\end{enumerate}

\newpage
Solutions:
\begin{enumerate}
    \item The mass converted to energy is:
    \[
    M_{converted} = M_{star}-M_{blackhole} = 18 M_{sun} = 3.5964\cross 10^{31} \text{ kg}
    \]
    By Einstein's mass-energy equivalence, the energy released during this supernova explosion is:
    \[
    E_{released} = M_{converted}c^2 = 3.232\cross 10^{48} \text{ J}
    \]
    Where $c$ is the speed of light in vaccum.
    Thus the average luminosity of this supernova explosion is:
    \[
    \overline{L} = \frac{E_{released}}{\Delta t} = 5.387\cross 10^{46} \text{ W}
    \]
    Which is $1.408\cross 10^{20}$ times larger than the luminosity of the Sun.

    \item The Schwarzschild radius of a blackhole is given by:
    \[
    R_s = \frac{2GM}{c^2} = 5934 \text{ m}
    \]

    \item The energy of a photon is:
    \[
    E_p = \frac{hc}{\lambda} = 3.347\cross 10^{-29} \text{ K}
    \]
    Where $h$ is the Planck constant, $c$ is the speed of light, $\lambda$ is the wavelength of the photon.

    The temperature of the photon is:
    \[
    T_p = \frac{E_p}{k_B} = 2.424\cross 10^{-6} \text{ K}
    \]

    \item Blackhole radiation can be treated as a blackbody radiation, thus the power of radiation is:
    \[
    P = 4\pi R_s^2 \sigma T^4 = 8.668\cross 10^{-22} \text{ W}
    \]

    \item By the definition of radiation power:
    \[
    P = \frac{\Delta E}{\Delta t}
    \]
    By Einstein's mass-energy equivalence, $\Delta E$ is related to mass loss due to radiation:
    \[
    \Delta E = \Delta M c^2
    \]
    Thus the rate which mass is lost in blackhole is:
    \[
    \frac{\Delta M}{\Delta t} = \frac{P}{c^2} = 9.644\cross 10^{-39} \text{ kg/s}
    \]
    At this rate, it will take the blackhole around 3 years to lose a mass of electron.
\end{enumerate}
\end{document}