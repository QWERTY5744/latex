\documentclass[UTF8]{ctexart}
\usepackage{graphicx} % Required for inserting images
\usepackage{amsmath,amssymb,amsthm}
\usepackage{physics}
\usepackage{graphicx,float}
\graphicspath{{images/}}
\usepackage[none]{hyphenat}
\usepackage{blindtext}
\usepackage{parskip}
\usepackage[letterpaper,top=3cm, left= 3cm,bottom=3cm]{geometry}
\usepackage{subcaption}
\numberwithin{equation}{section}

\begin{document}
\section{疲倦的光子(高年组)}
在大爆炸模型提出的早期,有人为了反对大爆炸模型同时解释哈勃定律,提出了“疲倦的光子”假说。
该假说认为宇宙是静态的,而光子会随着自己在宇宙中穿梭而失去能量。单位距离内损失的能量由下式表达:
\[
\frac{dE}{dr}=-kE
\]
其中$k$为一常数,$E$为光子的能量。

请证明当$z<<1$,上述假说会给出一个线性的红移-距离关系(即哈勃定律),并求出满足这个条件的$k$。

\subsection{参考答案}
解微分方程,有:
\begin{align*}
    \frac{dE}{dr}&=-kE\\
    \frac{dE}{E}&=-k dr\\
    \int_{E_0}^{E} \frac{dE}{E}&= -k \int_{0}^{r} dr\\
    \ln \frac{E}{E_0} &= -kr\\
    E(r) &= E_0 e^{-kr}
\end{align*}
其中$E_0$为光子初始的能量,$r$是光子走过的距离。

红移和能量也有关系:
\begin{align*}
    z = \frac{\lambda - \lambda_0}{\lambda_0} &= \frac{\frac{hc}{E}-\frac{hc}{E_0}}{\frac{hc}{E_0}}\\
    &= \frac{\frac{1}{E}-\frac{1}{E_0}}{\frac{1}{E_0}}\\
    &= E_0\cdot\frac{E_0-E}{EE_0}\\
    &= \frac{E_0 - E}{E}
\end{align*}
带入能量的表达式,有:
\begin{align*}
    z &= \frac{E_0 - E_0 e^{-kr}}{E_0 e^{-kr}}\\
    &= e^{kr}\left(1-e^{-kr}\right)\\
    &= e^{kr} - 1\\
\end{align*}
当$z<<1$,即$kr<<1$时,有
\begin{align*}
    z \approx kr
\end{align*}
同时带入多普勒效应($v = cz$)
\begin{align*}
    cz &= H_0 r\\
    c kr & = H_0 r\\
    k &= \frac{H_0}{c} = 2.267\cdot 10^{-4} \text{Mpc}^{-1}
\end{align*}
\end{document}