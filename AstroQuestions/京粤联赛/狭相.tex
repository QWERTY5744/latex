\documentclass[UTF8]{ctexart}
\usepackage{graphicx} % Required for inserting images
\usepackage{amsmath,amssymb,amsthm}
\usepackage{physics}
\usepackage{graphicx,float}
\graphicspath{{images/}}
\usepackage[none]{hyphenat}
\usepackage{blindtext}
\usepackage{parskip}
\usepackage[letterpaper,top=3cm, left= 3cm,bottom=3cm]{geometry}
\usepackage{subcaption}
\usepackage{tikz}
\usepackage{pgfplots}
\pgfplotsset{compat=1.18}
\usetikzlibrary{positioning,calc,patterns,angles,quotes}
\numberwithin{equation}{section}

\title{狭义相对论}
\author{勾陈一}
\date{2025/08/18}

\begin{document}
\maketitle
本题中,你将应用狭义相对论解决一些实际问题,下面是狭义相对论中的时空变换,又称洛伦兹变换,你会频繁的用到它。注意:狭义相对论只在惯性系中成立。

考虑一个坐标系,其中有3个空间坐标,1个时间坐标,即$(x,y,z,t)$。这样,在坐标系中发生的一个事件就可以完整的被记录下来。为了完成下面的题目,你将了解狭义相对论下的坐标转换公式。

如图1所示,考虑两个坐标系,$S$系与$S'$系,$S'$系相对$S$系的$+x$轴有$v$的相对速度,两个参考系的坐标转换公式由下式给出
\[
    \begin{cases}
        x' = \dfrac{x - vt}{\sqrt{1-\beta^2}}\\
        t' = \dfrac{t - \dfrac{v}{c^2}x}{\sqrt{1-\beta^2}}\\
        y' = y\\
        z' = z\\
    \end{cases}
\]
其中$\beta = \dfrac{v}{c}$。

\begin{figure}[H]
    \centering
    \includegraphics[width = 8cm]{coord.png}
    \caption{$S$系与$S'$系示意图}
\end{figure}

\section{试题}
完成下面问题:
\begin{enumerate}
    \item 以下哪个实验说证明了以太不存在,同时也证明了相对论基本原理(即光速不变)?
    \begin{enumerate}
        \item 迈克耳孙-莫雷实验
        \item 卡文迪什扭秤实验
        \item 托里拆利水银柱实验
        \item 油滴实验
    \end{enumerate}
    \item 考虑一辆高速行驶的火车,火车在进入隧道时的速度为$v = 0.8c$,隧道长$50$m,地面的隧道管理员在火车完全进入隧道时同时关闭隧道两边的大门,在火车驾驶员眼中,两扇门关闭的时间差是什么?火车进入隧道的门称为前门,火车离开隧道的门称为后门,哪扇门先关闭?
    \begin{enumerate}
        \item $222.376$ ns,前门
        \item $222.376$ ns,后门
        \item $222.376 \text{ }\mu$s,前门
        \item $222.376 \text{ }\mu$s,后门
    
    \end{enumerate}
    \item 尺缩钟慢是狭义相对论一个重要的效应。定义尺子的长度为一个参考系(S系)内两个地点同一时刻的坐标差,在一个相对于S系运动的系中测量尺子,会发现尺子的长度会发生改变。同理,定义时钟测量的时间差为同一坐标不同时刻的差,在一个运动的参考系中测量时间差也会发现时间差发生了改变。请根据洛伦兹变换求出这个改变量。记$L_0$为尺子的静长度,$t_0$为静止钟测量的时间;$L$为尺子的动长度,$t$为运动钟测量的时间,请给出它们二者的关系
    \begin{enumerate}
        \item $L = L_0\sqrt{1-\beta^2}$,$t_0 = t\sqrt{1-\beta^2}$
        \item $L = L\sqrt{1-\beta^2}$,$t = t_0\sqrt{1-\beta^2}$
        \item $L_0 = L\sqrt{1-\beta^2}$,$t_0 = t\sqrt{1-\beta^2}$
        \item $L = L_0\sqrt{1-\beta^2}$,$t = t_0 \sqrt{1-\beta^2}$
    \end{enumerate}

    \newpage
    \item 狭义相对论中,多普勒效应也会发生改变,请推导出完整的狭义相对论多普勒效应的公式。
    
    提示:考虑一个以$v$速度运动的源,观测者位于$\phi$方位角处,在源参考系和观测者参考系中,单位时间接受到的波峰数一样,即$\nu_0 dt_s = \nu dt_{obs}$,你应该求出$dt_s$与$dt_{obs}$的关系

    \begin{center}
        \begin{tikzpicture}
            \draw [->](0,4) -- node[below]{$vdt$} (2,4) node[black,right]{$v$} node[black,below]{$S'$};
            \draw (0,4) node[black,left]{$S$} -- node[black, above]{$r$} (4,7) node[black,right]{$O$};
            \draw[dotted] (2,4) --node[below]{$r'$} (4,7);

            \draw (2,4) coordinate (A)--(0,4) coordinate (B)--(4,7) coordinate (C)
            pic [draw,"$\phi$",black,angle eccentricity = 1.5]{angle = A--B--C};
        \end{tikzpicture}
    \end{center}

    \begin{enumerate}
        \item $\nu = \dfrac{\sqrt{1-\beta^2}}{1-\beta \cos \phi} \nu_0$
        \item $\nu = \dfrac{1-\beta \cos \phi}{\sqrt{1-\beta^2}} \nu_0$
        \item $\nu = \left(1-\beta \cos \phi\right)\sqrt{1-\beta^2} \nu_0$
        \item $\nu_0 = \left(1-\beta \cos \phi\right)\sqrt{1-\beta^2}\nu$
    \end{enumerate}

    \item 考虑一个面向观测者以$v = 0.99c$运动的H$\alpha$源,观测者眼中的H$\alpha$线波长为多少(实验室系中H$\alpha$发射线的波长为$\lambda = 656.28$nm)
    \begin{enumerate}
        \item $9528$ nm
        \item $465.224 \text{ }\mu$m
        \item $46.522$ nm
        \item $0.926$ nm
    \end{enumerate}

    \item 考虑两个坐标系,$S'$系相对$S$系有$u$的相对速度,如果一个物体在$S$系$x$轴的速度分量为$v$,求$S'$系中该物体$x'$轴的速度分量$v'$。提示:求出$\dfrac{dx}{dt}$和$\dfrac{dx'}{dt'}$,并尝试联立二者。
    \begin{enumerate}
        \item $v' = \dfrac{v + u}{1 - \dfrac{uv}{c^2}}$
        \item $v' = \dfrac{v + u }{1 + \dfrac{uv}{c^2}}$
        \item $v' = \dfrac{v-u}{1 - \dfrac{uv}{c^2}}$
        \item $v' = \dfrac{v-u}{1+\dfrac{uv}{c^2}}$
    \end{enumerate}

    \item 推导狭义相对论中动质量和静质量的关系式。
    
    提示:考虑两个小球,一个小球以$v$的速度撞上了一个静止的小球,整个过程质量和动量守恒,碰撞为完全非弹性碰撞。
    \begin{enumerate}
        \item $m = m_0 \sqrt{1-\beta^2}$
        \item $m = \dfrac{m_0}{\sqrt{1-\beta^2}}$
        \item $m = (1-\beta)m_0 $
        \item $m = \dfrac{m_0}{1-\beta}$
    \end{enumerate}

    \item 想必你一定知道相对论的双生子悖论。考虑一对双胞胎,一位兄弟乘坐飞船以$0.99c$的速度远离地球,10年后转向并再花10年回到地球。此时地球上的兄弟认为飞船上的兄弟在高速运动,因此飞船上的兄弟经历的时间更慢;而飞船上的兄弟认为地球上的兄弟在高速远离自己,因此地球上的兄弟经历的时间更慢。这就产生了悖论,请解释悖论为什么不合理
    \begin{enumerate}
        \item 年龄是以大部分人类所经历的时间定义的,所以飞船上兄弟对于时间流速的观点不成立
        \item 因为飞船要转向,所以飞船上的兄弟不一直处于一个惯性系,因此狭义相对论不适用
        \item 因为在极高速情况下,狭义相对论失效了,需要考虑广义相对论
        \item 因为光速有限,A 看到 B 的钟走得慢,B 看到 A 的钟走得快
    \end{enumerate}
\end{enumerate}

\newpage
\section{参考答案}
\begin{enumerate}
    \item 迈克耳孙-莫雷实验证明了以太不存在,从而引出光速不变,答案是A。卡文迪许扭秤实验测出了万有引力常数值,托里拆利水银柱实验测出了大气压,油滴实验实验测出了基本电荷的大小
    \item 在地面系中,记前门关闭的坐标为$(x_1, t_1)$,记后门关闭的坐标为$(x_2, t_2)$,根据时间的变换式,有
    \[
    t'_2 - t'_1 = \frac{t_2 - t_1 - \dfrac{v}{c^2}(x_2 - x_1)}{\sqrt{1-\beta^2}}
    \]
    地面系中,两事件同时发生,即$t_2 - t_1 = 0$,而$x_2 - x_1 = 50$,因此,在火车参考系中,两扇门关闭的时差为
    \[
    \Delta t' = \frac{v}{c^2} \frac{x_1 - x_2}{\sqrt{1-\beta^2}} = 222.376 \text{ns}
    \]
    $\Delta t' > 0$说明$t'_1 < t'_2$,也就是后门先关,答案是B。

    \item 定义尺子为两个地点同一时刻的坐标差,假设尺子相对一参考系有$v$的速度,因此有
    \[
    x'_2 - x'_1 = \frac{x_2 - vt_2 - x_1 + vt_1}{\sqrt{1-\beta^2}}
    \]
    其中$L_0 = x'_2 - x'_1$是尺子系中尺子的长度,即尺子的静长度,$L = x_2 - x_1$是静止系中尺子的长度,即尺子的动长度,因此
    \[
    L_0 = \frac{L - v(t_1 - t_2)}{\sqrt{1-\beta^2}} = \frac{L}{\sqrt{1-\beta^2}}
    \]
    因此有
    \[
    L = L_0\sqrt{1-\beta^2}
    \]
    可以发现$\sqrt{1-\beta^2} < 1$,因此尺子的动长度永远比静长度短。

    同理,定义钟测量的时间差为同一地点两次测量的时间差,因此有
    \[
    t'_2 - t'_1 = \frac{t_2 - t_1 - \dfrac{v}{c^2}(x_2 - x_1)}{\sqrt{1-\beta^2}}
    \]
    即
    \[
    t = t_0\sqrt{1-\beta^2}
    \]
    答案选D。

    \item 当$t=0$时,波源位于$S$点并发射一束光,$dt_s$时刻后,波源运行到了$S'$点并发射第二束光。在观测者$O$眼中,此段时间为
    \[
    dt = \frac{dt_s}{\sqrt{1-\beta^2}}
    \]
    \begin{center}
        \begin{tikzpicture}[scale=1.2]
            % 定义三角形的三个顶点
            \coordinate (S) at (0,4);
            \coordinate (S') at (2,4);
            \coordinate (O) at (4,7);
            
            % 绘制三角形的三条边
            \draw[->] (S) -- node[below]{$vdt$} (S');
            \draw (S) -- node[above left]{$r$} (O);
            \draw [dotted] (S') -- node[below right]{$r'$} (O);
            
            % 添加顶点标签
            \node[below left] at (S) {$S$};
            \node[below right] at (S') {$S'$};
            \node[above] at (O) {$O$};
            
            % 在S'处添加速度标记
            \node[right] at (S') {$v$};
            
            % 绘制角度φ (在顶点S处)
            \pic [draw, "$\phi$", angle eccentricity=1.3, angle radius=0.8cm] {angle = S'--S--O};
            
            % 过S'点做到SO边的高
            \coordinate (H) at ($(S)!(S')!(O)$);  % H是S'在SO边上的垂足
            \draw[dashed] (S') -- (H) node[midway, right]{};
            
            % 标记垂足
            \node[left] at (H) {$H$};
            
            % 标记直角符号
            \pic [draw, angle radius=3mm] {right angle = S--H--S'};    
        \end{tikzpicture}
    \end{center}
    $t = \dfrac{r_0}{c}$时,该光束到达$O$点。下面我们要求在$O$参考系中,光束何时到达,源首先用了$dt$时间运动到了$S'$处并发射一束光,运动的距离为$vdt$,这束光在发射后$\dfrac{r'}{c}$被观测者接受。其中
    \[
    r' = r-vdt \cos \phi
    \]
    这里,我们假设$vdt_s$十分的小,这样$r$和$r'$几乎重合,那么在观测者眼中,两束光到达的时间差为
    \[
    dt_{obs} = dt + \frac{r - v dt \cos \phi}{c} - \frac{r}{c} = \left(1-\beta \cos \phi\right)dt = \frac{1-\beta \cos \phi}{\sqrt{1-\beta^2}}dt_s
    \]
    因为两个参考系内接受到的波峰数不变,所以
    \[
    \nu_0 dt_s = \nu dt_{obs}
    \]
    则
    \[
    \nu = \nu_0 \frac{dt_s}{dt_{obs}} = \nu_0 \frac{\sqrt{1-\beta^2}}{1-\beta \cos \phi}
    \]
    答案选A。
    \item 波频率和波速有如下关系:
    \[
    \lambda \nu = c
    \]
    其中$c$为光速,由上一问求出的多普勒效应公式
    \[
    \lambda = \frac{1-\beta \cos \phi}{\sqrt{1-\beta^2}} \lambda_0
    \]
    带入$\lambda_0 = 656.28$ nm,$\beta = 0.99$,得出观测到的频率为$\lambda = 46.522$ nm,属于极紫外线。答案选C。
    \item 已知
    \[
        \begin{cases}
            x' = \dfrac{x - vt}{\sqrt{1-\beta^2}}\\
            t' = \dfrac{t - \dfrac{v}{c^2}x}{\sqrt{1-\beta^2}}\\
        \end{cases}
    \]
    因此
    \[
    \frac{dx'}{dt'} = \frac{d\left(\dfrac{x - ut}{\sqrt{1-\beta^2}}\right)}{d\left(\dfrac{t - \dfrac{u}{c^2}x}{\sqrt{1-\beta^2}}\right)}
    \]
    化简后得出
    \[
    \frac{dx'}{dt'} = \frac{dx - udt}{dt - \dfrac{u}{c^2} dx}
    \]
    上下同除$dt$,得到$x$轴速度变换式
    \[
    v' = \frac{\dfrac{dx}{dt} - u}{1 - \dfrac{u}{c^2} \dfrac{dx}{dt}v} = \frac{v - u}{1-\dfrac{u}{c^2}v}
    \]

    \item 这道题有点先射箭再画靶的嫌疑。首先是要证明质量真的和速度有关,然后才能去推导这个动质量和静质量的关系。不过具体证明十分麻烦,就直接当质量和速度有关就好了。
    
        设两个球的静质量为$m_0$,以$v$运动时,动质量为$m$;碰撞后的大球质量为$M$,以$u$的速度运动。
        
        根据质量守恒和动量守恒,有
        \[
            \begin{split}
                m+m_0 = M\\
                mv = Mu
            \end{split}
        \]
        可以解出
        \[
        m = \dfrac{m_0}{\dfrac{v}{u}-1}
        \]
        接下来要求出$v$和$u$的关系,将两个球分别命名为A球和B球,以A球为参考系,B球以$v$的速度运动,碰撞后大球速度为$u$;如果以B球为参考系,那么A球就以$-v$的速度运动,碰撞后大球的速度为$-u$。两个参考系的相对速度为$v$,根据速度变换,有
        \[
        -u = \frac{u-v}{1-\dfrac{v}{c^2}u}
        \]
        变形后,有
        \[
        \frac{uv}{c^2} - 2 + \frac{v}{u} = 0
        \]
        令$r = \dfrac{v}{u}$,带入后有
        
            \begin{align*}
                \frac{ru^2}{c^2} - 2 + r &= 0\\
                u^2 &= \frac{c^2}{r} - 2 + r\\
                \frac{v^2}{r^2} &= \frac{c^2}{r}(2-r)\\
            \end{align*}
        再次化简会得到一个二次方程
        \[
        r^2 - 2r + \frac{v^2}{c^2} = 0
        \]
        取这个二次方程的正解
        \[
        r = \frac{v}{u} = \frac{2+\sqrt{4-4\frac{v^2}{c^2}}}{2} = 1 + \sqrt{1-\frac{v^2}{c^2}}
        \]
        因此动质量和静质量的关系为
        \[
        m = \frac{m_0}{\sqrt{1-\beta^2}}
        \]
        答案是B。

    \item 狭义相对论的前提是惯性系,换句话说洛伦兹变换只在两个有相对运动的惯性系中成立,其中飞船上的兄弟转向时改变的速度矢量,因此他不一直在一个惯性系中,所以飞船上的兄弟不能如此简单的应用狭义相对论,需要考虑广义相对论。答案是B。
\end{enumerate}

\end{document} 