\documentclass[UTF8]{ctexart}
\usepackage{graphicx} % Required for inserting images
\usepackage{amsmath,amssymb,amsthm}
\usepackage{physics}
\usepackage{graphicx,float}
\graphicspath{{images/}}
\usepackage[none]{hyphenat}
\usepackage{blindtext}
\usepackage{parskip}
\usepackage[letterpaper,top=3cm, left= 3cm,bottom=3cm]{geometry}
\usepackage{subcaption}
\usepackage{tikz}
\usepackage{pgfplots}
\pgfplotsset{compat=1.18}
\usetikzlibrary{positioning,calc,patterns,angles,quotes}
\numberwithin{equation}{section}

\title{AGN}
\author{李晨}
\date{2025/09/13}

\begin{document}
\maketitle

考虑一个中心天体质量为$M$,光学薄的吸积盘,周围质量以$\dot{M}$的速率吸入盘中。盘中一质量元距离中心天体距离为$r$,质量为$dm$,下落了一小段距离$dr$,因此改变的机械能是
\[
\dv{E}{r} = -\dv{r}\frac{GMdm}{2r} = \frac{Gmdm}{2r^2}
\]
假设吸积盘是稳定的,那么加入环和离开环的质量应该一样,即$dm = \dot{M}dt$,因此,距离中心天体为$r$,宽度为$dr$的环因引力势能产生的辐射的光度为
\[
dL = \frac{GM\dot{M}}{2r^2}dr
\]
吸积盘的光度可以由积分给出
\[
\int_{R}^{\infty} dL = \int_{R}^{\infty} \frac{GM\dot{M}}{2r^2}dr = \frac{GM\dot{M}}{2R}
\]
根据Stefan-Boltzmann定律,可以写出
\[
2(2\pi r dr) \sigma T^4 = \frac{GM\dot{M}}{2r^2} dr
\]
其中这段环的面积$A = 2 (2\pi r)$,因为环可以在上下两个方向辐射。因此这段环的温度为
\[
T(r) = \left(\frac{GM\dot{M}}{8\pi\sigma r^3}\right)^{1/4}
\]
考虑到气体下落到主星表面附近遇到的湍流后,温度关于距离的关系式可以这样写出
\[
T(r) = \left(\frac{3GM\dot{M}}{8\pi \sigma r^3} \left(1-\sqrt{\frac{r}{R}}\right)\right)^{1/4}
\]
其中$R$是吸积盘的内半径。当$r \leq R$时,上式给出的结果近似等于上上式推导的结果。

同时,根据Planck定律,单位频率的辐射能量密度和温度与频率有关
\[
B_\nu (T) \propto \nu^3 \left[\exp \left(\frac{h\nu}{kT} - 1\right)\right]^{-1}
\]
积分可以得到单位频率的辐射强度
\[
S_\nu \propto \int_{R_{\text{min}}}^{R_{\text{max}}} B_\nu (T) 2\pi r dr
\]
\end{document}