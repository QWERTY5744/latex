\documentclass{article}
\usepackage{graphicx} % Required for inserting images
\usepackage{amsmath,amssymb,amsthm}
\usepackage{physics}
\usepackage{graphicx,float}
\graphicspath{{images/}}
\usepackage[none]{hyphenat}
\usepackage{blindtext}
\usepackage{parskip}
\usepackage[letterpaper,top=3cm, left= 3cm,bottom=3cm]{geometry}
\usepackage{subcaption}
\usepackage{tikz}
\usepackage{pgfplots}
\pgfplotsset{compat=1.18}
\usetikzlibrary{positioning,calc,patterns,angles,quotes}
\numberwithin{equation}{section}

\title{New Year Homework}
\author{Chen Li}
\date{2025/09/29}

\begin{document}
By Planck Law, the energy the star emitted at the specific waveband per unit area is 
\[
F = \int_{5\cdot 10^{14}}^{6\cdot 10^{14}} \left(\frac{2h\nu^3}{c^2}\frac{1}{\exp{\frac{h\nu}{k_B T}}-1} - 1\right) d\nu = 7.99\cdot 10^6 ~\mathrm{W/m^2}
\]
The total energy that comes out of the star in the selected waveband is 
\[
L = 4\pi R^2 F = 4.86\cdot 10^{25} ~\mathrm{W}
\]
The photon flux arriving at the telescope is therefore 
\[
N = \frac{AQLt}{4 h \pi \bar{\nu} d^2}
\]
Where $\bar{\nu}$ is the average wavelength of the photon.

The flux due to the background light is 
\[
F_b = 6\cdot 10^{-29} \mathrm{W/m^2 \cdot Hz\cdot sr} \cdot 1\cdot 10^{14} \mathrm{Hz} \cdot \left(\frac{1}{206265}\right)^2 \mathrm{sr} = 1.41\cdot 10^{-25} \mathrm{W/m^2}
\]
The incomming photon over the integration period is therefore
\[
N_b = \frac{F_b}{h\bar{\nu}}\pi r^2 Qt = 0.131 ~\mathrm{photon}
\]
By the definition of SNR 
\[
\mathrm{SNR} = \frac{N}{\sqrt{N_b}} = 10
\]
Therefore, distance is related to SNR as such
\[
\frac{QL}{4 h \pi \bar{\nu} d^2} = 10 \sqrt{N_b}
\]
Or 
\[
d = \sqrt{\frac{AQLt}{40 \sqrt{N_b}h\pi \bar{\nu}}} = 9.98 \cdot 10^{23} ~\mathrm{m} = 32 ~\mathrm{Mpc}
\]

If the photon from the star is also accounted for the calulation of SNR, one have 
\[
\mathrm{SNR} = \frac{N}{\sqrt{N + N_b}} = 10
\]
This requires 
\[
N^2 - 100N - N_b = 0
\]
Taking the positive solution, $N = 100.00131$ and the distance is 
\[
d' = \sqrt{\frac{AQLt}{40 \sqrt{N_b}h\pi \bar{\nu}}} = 1.898 \cdot 10^{23} ~\mathrm{m} = 6.15 ~\mathrm{Mpc}
\]

\newpage
The magnitude and flux is related as 
\[
m = -2.5\lg S + C
\]
By introducing noise, the magnitude will deviate 
\[
m \pm \delta m = -2.5 \lg (S\pm N) + C
\]
Therefore, the deviation in magnitude is 
\[
\delta m = \pm 2.5 \lg \left(1+\frac{N}{S}\right) = \pm 2.5 \lg \left(1+\frac{1}{\sqrt{S}}\right)
\]
Therefore the signal needed to register a deviation of $0.02$ will be 
\[
S = \left(\frac{1}{10^{0.4 \delta m }-1}\right)^2 = 2893 ~\mathrm{photons/s}
\]
For a $15$ magnitude star, the telescope can capture $S_{15} = F\cdot A = 7854$ photons/s. The time needed is therefore 
\[
t = \frac{S}{S_{15}} = 0.37 ~\mathrm{s}
\]

\newpage
By hydrostatic equilibrium:
\[
\frac{dP}{dr} = -\frac{GM(r)\rho(r)}{r^2}
\]
Multiply $dV = 4\pi r^2 dr$ on both side 
\[
4\pi r^2 dP = -\frac{GM(r)\rho(r)}{r^2} 4\pi r^2 dr
\]
Rearrange and integrate both side
\[
\int \frac{4}{3}\pi r^3 dP = \int -\frac{GM(r)dm}{3r}
\]
The LHS can be solved by integration by parts and the RHS is the total gravitaional potential energy.
\[
VP - \int PdV = -\frac{1}{3}\Omega
\]
$VP$ should equal to $0$ based on the boundary condition of zero volume in center and zero pressure at outer radius. Therefore 
\[
\int PdV = -\frac{1}{3} \Omega
\]
The pressure of a gas is related to the average speed of particles
\[
P = \frac{1}{3} \frac{N}{V} m \bar{v}^2
\]
The internal energy density for ideal gas is the sum of all kinetic energy of particles 
\[
u = \frac{1}{2} \frac{N}{V} m \bar{v}^2
\]
Compare the two, we arrive at 
\[
\frac{P}{u} = \frac{2}{3}
\]
Rearrange and integrate both side with respect to volume 
\[
U = \frac{3}{2} \int P dV
\]
For photon gas, consider the pressure integral 
\[
P = \frac{1}{3}\int_0^\infty n_p v p dp 
\]
For photon $v = c$ and $E = pc$, therefore the integral can be turned into 
\[
P = \frac{1}{3} \int_0^\infty n_p E dp
\]
Internal energy density is 
\[
u = \int_0^\infty n_p E dp
\]
Therefore, for photon gas, the internal energy is 
\[
U = 3\int PdV
\]
Since the photon pressure is half of gas pressure, the gas pressure is $2/3$ of total pressure and the photon pressure is $1/3$ of total pressure.
\[
U_{tot} = U_{\mathrm{gas}} + U_{\mathrm{photon}} = \frac{3}{2} \int \frac{2}{3} PdV + 3 \frac{1}{3}\int PdV = 2 \int PdV = -\frac{2}{3} \Omega
\]
The total gravitaional potential energy is given by integration
\[
\Omega = -\int_0^R \frac{GM(r)}{r} dm = \int_0^R \frac{GM(r)\rho(r)}{r} 4\pi r^2 dr
\]
The mass as a function of radius is 
\[
M(r) = \int_0^r 4\pi r^2 \rho(r) dr = 4\pi \rho_0 \int_0^r \left(r^2 - \frac{r^{\alpha + 2}}{R^\alpha}\right) dr = 4\pi \rho_0 \left(\frac{1}{3}r^3 - \frac{1}{\alpha + 3}\frac{r^{\alpha + 3}}{R^{\alpha+3}}\right)
\]
The gravitaional energy integral turns into 
\[
\Omega = -16G\pi^2 \rho^2_0 \int_0^R \left(r- \frac{r^{\alpha+1}}{R^\alpha}\right)\left(\frac{1}{3}r^3 - \frac{1}{\alpha + 3}\frac{r^{\alpha + 3}}{R^{\alpha+3}}\right) dr
\]
This integral gives 
\[
\Omega = -16G\pi^2 \rho^2_0 \frac{\alpha^2 (2\alpha + 11)}{15 (\alpha + 5)(\alpha + 3)(2\alpha + 5)}
\]
Finally, solve $\rho_0$ in terms of $M$, notice that $M(R) = M$, this gives 
\[
\rho_0 = \frac{3M(\alpha + 3)}{4 \pi R^3 \alpha}
\]
The total gravitational potential energy of the star is 
\[
\Omega = -16G\pi^2 \frac{9M^2 (\alpha + 3)^2}{16\pi^2 \alpha^2 R^6 } \frac{\alpha^2 (2\alpha + 11) R^5}{15 (\alpha + 5)(\alpha + 3)(2\alpha + 5)}
\]
After simplification 
\[
\Omega = -\frac{3}{5} \frac{GM^2}{R} \frac{(2\alpha +11)(\alpha+3)}{(\alpha + 5)(2\alpha + 5)}
\]
The internal energy of the star is therefore 
\[
U = \frac{2}{3}\Omega = \frac{2}{5}\frac{GM^2}{R} \frac{(2\alpha +11)(\alpha+3)}{(\alpha + 5)(2\alpha + 5)}
\]
\end{document}