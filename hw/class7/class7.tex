\documentclass{article}
\usepackage{graphicx} % Required for inserting images
\usepackage{amsmath,amssymb,amsthm}
\usepackage{physics}
\usepackage{graphicx,float}
\graphicspath{{images/}}
\usepackage[none]{hyphenat}
\usepackage{blindtext}
\usepackage{parskip}
\usepackage[letterpaper,top=3cm, left= 3cm,bottom=3cm]{geometry}
\usepackage{subcaption}
\usepackage{tikz}
\usepackage{mathtools}
\usepackage{pgfplots}
\pgfplotsset{compat=1.18}
\usetikzlibrary{positioning,calc,patterns,angles,quotes}
\numberwithin{equation}{section}

\title{Homework 1}
\author{Chen Li}
\date{2026/02/01}

\begin{document}
\maketitle
\section{Protostellar Phase in the Formation of Stars}
(a) The mean luminosity of the star during its contraction period is equal to the released energy over the free fall time scale.

By Virial theorem, the energy released during contraction will be half of the avaliable gravitational potential energy
\[
E_{\mathrm{released}} = \frac{GM^2}{2R}
\]
The average luminosity of the cloud in this period is therefore 
\[
L = \frac{E_{\mathrm{released}}}{t_{ff}} = \frac{GM^2}{2R} \sqrt{\frac{32G\rho}{3\pi}}
\]

(b) I am not entirely sure about what the question is asking
\section{H II regions and Jeans stability}
The density of the cloud is $\rho = \mu m_\mathrm{H} n_{\mathrm{HII}} = 8.365\cdot 10^{-19} ~\mathrm{kg/ m^{-3}}$. Here, $\rho = 0.5$ because the mass of electron is neglected.

The Jeans mass of the cloud is therefore 
\[
M_J = \left(\frac{5kT}{G\mu m_{\mathrm{H}}}\right)^{3/2} \left(\frac{3}{4\pi \rho}\right)^{1/2} = 1.662 \cdot 10^{37} ~\mathrm{kg}
\]
The mass of the cloud is 
\[
M = \frac{4}{3}\pi R^3 \rho = 2.966\cdot 10^{33} ~\mathrm{kg}
\]
Which is smaller than the Jean's mass, meaning that the cloud will not collapse under its own gravity.
\section{Magnetic Support in a Molecular Cloud}
The gravitational binding energy density is 
\[
u_G = \frac{U}{V} = \frac{3}{5}\frac{GM^2}{R} \frac{3}{4\pi R^3} = \frac{9GM^2}{20\pi R^4}
\]
For the magnetic to have enough support, $u_B = u_G$, which gives 
\[
\frac{B^2}{8\pi} = \frac{9GM^2}{20\pi R^4}
\]
Which leads to 
\[
B_{min} = \sqrt{\frac{18GM^2}{5R^4}}
\]
The radius of the cloud can be obtained through the mass and the number density of the cloud.
\[
V = \frac{N}{n} = \frac{M}{n_{\mathrm{H}_2} \mu m_p}
\]
Combined with the volume of sphere, this gives the radius of sphere to be $5.216\cdot 10^{18}$ cm.

The minimum magnetic field is $B_{min} = 37$ $\mu$G, which is very small. Therefore, magnetic pressure is needs to be accounted for while determining the collapse of molecular clouds.
\section{Horizontal-Branch Lifetime of the Sun}
The total number of avaliable Helium atom for triple $\alpha$ process is 
\[
N_{\mathrm{He}} = \frac{0.1 M_{s}}{m_{\mathrm{He}}} = 2.993\cdot 10^{55}
\]
Each $3\alpha$ reaction requires $3$ helium atom and each reaction release $E_0 = 1.165 \cdot 10^{-12}$ J, the total energy produced by all the helium atom fusing will be 
\[
E_{3\alpha} = \frac{N_{\mathrm{He}}}{3} E_0 = 1.162 \cdot 10^{43} ~\mathrm{J}
\]
The time sun will spend on horizontal-branch is therefore 
\[
\tau = \frac{E_{3\alpha}}{L_{\mathrm{HB}}} = 3.038 \cdot 10^{14} ~\mathrm{s} = 231 ~\mathrm{Myr}
\]
\section{Lifetime of a Black Hole (Hawking Radiation)}
(a) The blackhole lose mass through a rate of 
\[
\dot{M} = \frac{\dot{E}}{c^2} = \frac{L}{c^2}
\]
The luminosity of the black hole can be given through Stefan-Boltzmann Law
\[
L = 4\pi R_c^2 \sigma T^4 = \frac{1}{256\pi^3} \frac{\hbar^4 c^8 \sigma}{G^2 M^2 k_B^4}
\]
Where $C = 1.0576\cdot 16{-19}$

Therefore 
\[
\frac{dM}{dt} = \frac{1}{256\pi^3} \frac{\hbar^4 c^6 \sigma}{G^2 M^2 k_B^4} = \frac{C}{M^2}
\]
Where $C = 3.963 \cdot 10^{15}$

Integrating both sides with respect to time yields 
\[
\int_{0}^{M} M^2 dM = C \int_{0}^{\tau} dt
\]
Which gives 
\[
\tau = \frac{M^3}{3C} = 8.410 \cdot 10^{-17} M^3
\]
(b) For a proton mass black hole, the lifespan is 
\[
\tau_{p} = 3.94\cdot 10^{-97}s
\]
Which is significantly smaller than Planck time

For a black hole with mass of $1000$ kg, the evapoartion time is 
\[
\tau_{1000} = 8.41 \cdot 10^{-8} ~\mathrm{s} = 84.1 ~\mathrm{ns}
\]
This is around the same as the period of 10 MHz wave.

For a black hole with mass of Earth, the evapoartion time is 
\[
\tau_{E} = 1.793\cdot 10^{58} ~\mathrm{s} = 1.363 \cdot 10^{52} ~\mathrm{yr}
\]
Which is significantly longer than the lifespan of the universe.

For a black hole with 5 solar mass, the evaporation time is 
\[
\tau_{5S} = 8.385\cdot 10^{76} ~\mathrm{s} = 6.377 \cdot 10^{70} ~\mathrm{yr}
\]
\section{Equilibrium Abundance of $\prescript{3}{}{\mathrm{He}}$ in the ppI Chain}
The rate of creation of $\prescript{3}{}{\mathrm{He}}$ is $\lambda_{pd}n_{p} n_{d}$ and the rate of destruction is $\lambda_{33} n_3^2/2$. For $\prescript{3}{}{\mathrm{He}}$ to be at steady state, the sum of the two rates is zero. Which gives 
\[
\lambda_{pd}n_{p} n_{d} - \lambda_{33} \frac{n_3^2}{2} = 0
\]
Algebraic manipulation yields
\[
\frac{\lambda_{pd}}{\lambda_{33}} = \frac{n_3^2}{2n_p n_d}
\]
Substitute $n_{pd}$ in terms of $n_{pp}$, one get 
\[
\frac{\lambda_{pp}}{\lambda_{33}} = \frac{n_3^2}{n_p^2}
\]
This means that the number density of $\prescript{3}{}{\mathrm{He}}$ is 
\[
n_3 = n_p \sqrt{\frac{\lambda_{pp}}{\lambda_{33}}}
\]
Divide both sides by $n_4$, the ratio between $\prescript{3}{}{\mathrm{He}}$ and $\prescript{4}{}{\mathrm{He}}$ is
\[
\frac{n_3}{n_4} = \frac{n_p}{n_4} \sqrt{\frac{\lambda_{pp}}{\lambda_{33}}}
\]
In sun, $3n_4 \approx n_p$, therefore 
\[
\frac{n_3}{n_4} = 3\sqrt{\frac{\lambda_{pp}}{\lambda_{33}}}
\]
\end{document}