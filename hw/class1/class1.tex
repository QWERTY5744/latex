\documentclass{article}
\usepackage{graphicx} % Required for inserting images
\usepackage{amsmath,amssymb,amsthm}
\usepackage{physics}
\usepackage{graphicx,float}
\graphicspath{{images/}}
\usepackage[none]{hyphenat}
\usepackage{blindtext}
\usepackage{parskip}
\usepackage[letterpaper,top=3cm, left= 3cm,bottom=3cm]{geometry}
\usepackage{subcaption}
\usepackage{tikz}
\usepackage{pgfplots}
\pgfplotsset{compat=1.18}
\usetikzlibrary{positioning,calc,patterns,angles,quotes}
\numberwithin{equation}{section}

\title{Homework 1}
\author{Chen Li}
\date{2025/09/29}

\begin{document}
\maketitle
\section{Time equation}

Time equation is given as 
\[
ET = (M-\nu_0) - \arctan(\cos \varepsilon \tan(\nu - \nu_0))
\]
Where $\nu_0$ is the true anomaly of vernal equinox, the ascenion of periapsis of Earth $\varpi = 103^{\circ}$. This number is heliocentric, to convert into geocentric ecliptic longitude, $\lambda = 180^{\circ} + \varpi = 287^{\circ}$. Which means $\nu_0 = \lambda = 283^{\circ}$

By the relation between true anomaly and eccentric anomaly
\[
\cos \nu = \frac{\cos E - e}{1-e\cos E}
\]
Therefore eccentric anomaly of vernal equinox can be easily calculated
\[
\cos E = \frac{\cos \nu + e}{1+e\cos \nu}
\]
This gives $E = 72.087^{\circ}$.

By Kepler's Equation, the mean anomaly is
\[
M = E - e\sin E = 72.071^{\circ}
\]
This means when Earth reaches periapsis, it is $72.071^{\circ}/360^{\circ} \cdot 365.2425 \text{d} = 73$ days before vernal equinox, therefore the date when Earth reach periapsis is January 7th.

The time difference between Sept 28th and Jan 7th is 264 days, therefore the mean anomaly is 
\[
M' = \frac{264 \text{days}}{365.2425 \text{days}} \cdot 360^{\circ} = 260.211^{\circ}
\]

Solving the Kepler's Equation numerically, the eccentric anomaly of Sun at Sept 28th is $E' = 260.194 ^{\circ}$, which gives the true anomaly of the Sun
\[
\nu' = \arccos \left(\frac{\cos E - e}{1-e\cos E}\right) = 259.252^{\circ}
\]

Time equation is therefore
\[
ET = (M' - \nu_0) - \arctan(\cos \varepsilon \tan(\nu' - \nu_0)) = -0.816^{\circ} = -3.264 \text{min}
\]

\section{Moon}
The maximum difference of azimuth on two points of lunar disk is $0.5^\circ$, when differnce between RA is half of it, it means $\Delta \alpha = 0.25^\circ$, which means the latitude line makes an angle of
\[
\theta = \arctan \left(\frac{0.25/2}{0.25}\right) = 26.565^{\circ}
\]
with respect to the "horizontal line" of the lunar disk.

Since the moon is on the $5^\circ$ latitude line at this day, this means that the angle between the "altitude velocity" and "azimuth velocity" must be the same as $\theta$

The relation between altitude and hour angle is 
\[
\sin h = \sin \phi \sin \delta \cos \phi \cos \delta \cos HA
\]
Therefore the rate which altitude change is
\[
\dot{h} = -\omega \cos \phi \frac{\cos \delta \sin HA}{\cos h}
\]
By sine law and five-parts formula
\[
\begin{cases}
    \cos h \sin A = -\cos \delta \sin H\\
    \cos h \cos A = \cos \phi \cos \delta - \sin \phi \cos \delta \cos H
\end{cases}
\]
Differnetiate both side against time
\[
\begin{cases}
    -\sin A  \sin h \dot{h} + \cos A \cos h \dot{A} = -\omega \cos \delta \cos H\\
    -\cos A \sin h \dot{h} - \sin A \cos h\dot{A} = \omega \cos \phi \cos \delta \sin H
\end{cases}
\]
Multiply the upper equation by $\cos A$ and lower equation by $\sin A$ to eliminate $\dot{h}$, the expression for $\dot{A}$ is then
\[
\dot{A} = -\omega \frac{\cos \delta \cos H \cos A + \cos \phi \sin H \sin A \cos \delta}{\cos h}
\]
Factoring in the relations given previously, this can be further simplified to
\[
\dot{A} = -\omega \frac{\cos \phi \cos \delta \cos H + \sin \phi \sin \delta}{\cos h}
\]
Compare the two velocities
\[
\frac{\dot{h}}{\dot{A}} = \frac{\cos \phi \cos \delta \sin HA}{\cos \phi \cos \delta \cos HA + \sin \phi \sin \delta} = \tan \theta
\]
Solving this expression numerically, one obtain $\mathrm{HA} = 27.859^\circ$ or $\mathrm{HA} = 332.141^\circ$ because this situation is symmetrical with respect to meridian.
\section{Moving Stick Shadow}
At local noon, the hour angle of the Sun $HA = 0^\text{h}$ and reaches a minimum zenith angle of $z = \phi - \delta_S = 6^{\circ} 34'$, therefore the altitude of the Sun is $h = 90^{\circ} - z = 83^{\circ} 26'$

The movement of the direction of the shadow is directly related to the change in azimuth of the Sun

The relationship between azimuth, hour angle, declination of a celetial body and local latitude can be given from sine law:
\[
\frac{\sin \mathrm{HA}}{\cos h} = -\frac{\sin A}{\cos \delta}
\]
Isolating $A$ and take derivative on both side 
\[
\frac{d}{dt} \sin A = -\frac{d}{dt} \left(\frac{\sin \delta}{\cos h} \sin \mathrm{HA}\right)
\]
This expression comes out to
\[
\cos A \dot{A} = - \cos \delta \frac{\omega_S \cos \mathrm{HA} \cos h + \sin h \sin \mathrm{HA} \dot{h}}{\cos^2 h}
\]
Where $\omega_S = 2\pi / 24$hr is the angular velocity of Sun, at zenith, since Sun reaches its maximum altitude, $\dot{h} = 0$, therefore the expression simplifies to
\[
\dot{A} = - \frac{\cos HA}{\cos A\cos h} \omega_S \cos \delta
\]
At zenith, HA = $0$ and $A = 180^{\circ}$, therefore the derivative with respect to time of azimuth is
\[
A = 5.835 \cross 10^{-5} \mathrm{rad/s} = 120.348''\text{/s}
\]
\section{Jupiter Opposition}
The conversion between declination and RA and ecliptic latitude and ecliptic longitude is
\[
\begin{cases}
    \sin \beta = \cos \varepsilon \sin \delta - \sin \varepsilon \cos \delta \sin \alpha\\
    \cos \lambda = \cos \alpha \dfrac{\cos \delta}{\cos \beta}
\end{cases}
\]
Therefore it is easy to find the ecliptic coordinate of Jupiter
\[
\begin{cases}
    \beta_J = 0^{\circ}\\
    \lambda_J = 169.193^{\circ}
\end{cases}
\]
By definiiton of opposition, the ecliptic longitude of Sun has a difference of $180^{\circ}$, therefore the ecliptic longitude of Sun is $\lambda_S = 180^{\circ} + \lambda_J = 23^{\text{h}} 16^{\text{m}} 46^{\text{s}}$, therefore the ecliptic coordinate of Sun is
\[
\begin{cases}
    \beta_S = 0^{\circ}\\
    \lambda_S = 23^{\text{h}} 16^{\text{m}} 46^{\text{s}}
\end{cases}
\]

\section{Bob and his well}
The opening angle of the well is
\[
\Delta \theta = 2\arctan \left(\frac{0.5m}{50m}\right) = 0.02 \text{rad}
\]
On the celestial sphere, a solid angle element is 
\[
d\Omega = \sin \phi d\phi d\theta
\]
\begin{figure}[H]
    \centering
    \includegraphics[width=10cm]{pics/scs.png}
\end{figure}
Therefore the total solid angle of this strip of sky is 
\[
\Omega = \int_{0}^{\pi/6} \int_{\phi - \Delta\theta }^{\phi + \Delta \theta}\sin \phi d\phi d\theta = 7.33\cross 10^{-3} \text{sr}
\]
Where $\phi = 90^\circ - \phi_T$
The percentage of the star Bob can see can be obtained through comparing $\Omega$ with $4\pi$:
\[
\eta = \frac{\Omega}{4\pi} \cdot 100\% = 0.058\%
\]

\section{Diver in the Sea}
I don't know how to solve this question.

\section{Sun at Toronto}
The ecliptic longitude of the Sun is
\[
\sin \lambda = \frac{\sin \delta}{\sin \varepsilon} = -0.867
\]
Which gives $\lambda = 299.871^{\circ}$ or $\lambda =240.129^{\circ}$, the first solution is not possible as it is past winter solstice, therefore take the second solution. This means the number of days after vernal equinox is
\[
d = \frac{365.25 \text{ days}}{360^{\circ}} \cdot \lambda = 243 \text{ days}
\]
Therefore on that date, sideral time is ahead of solar time by $12 \text{h} + 243 \text{ days} \cdot 3^\text{m} 56^\text{s} /\text{day} - 24 \text{h}= 3^\text{h} 55^\text{m} 44^\text{s}$ behind solar time. 

Toronto is located at GMT-5 with a central meridian of $l = 75^{\circ}$ W, therefore the time difference between local time and zone time is
\[
\Delta t = \Delta l \cdot 1^{\text{h}}/15^\circ/ \text{h} = -0^\text{h} 4^\text{m} 24^\text{s}
\]
Therefore the local time of Toronto at 10:15 is $10^\text{h} 15^\text{m} 00^\text{s} + 0^\text{h} 4^\text{m} 24^\text{s} = 10^\text{h} 19^\text{m} 24^\text{s}$, the sideral time at is therefore $ST = 14^\text{h} 15^\text{m} 8^\text{s}$

The hour angle of Sun can be obtained
\[
t = ST - \alpha = -1^\text{h} 36^\text{m} 38^\text{s} = 22^\text{h} 23^\text{m} 22^\text{s}
\]
The relationship between hour angle and declination of star and altitude of the star is
\[
\sin h = \sin \phi \sin \delta + \cos \phi \cos \delta \cos t
\]
Therefore the altitude of Sun at Toronto at 10:15 is
\[
\sin h = \sin \phi \sin \delta + \cos \phi \cos \delta \cos t = 0.381
\]
This gives $h = 22^\circ 23' 30''$

\section{Shadow and Wall}
There are 153 days betweem March 21st and August 21st, he ecliptic longitude of Sun is
\[
\lambda = \frac{360^{\circ}}{365.25} \cdot 153 = 150.801^{\circ}
\]
Therefore the declination of Sun is
\[
\delta = \arcsin(\sin \delta \sin \lambda) = +11.187^{\circ}
\]
The RA of Sun is (for Sun $\beta = 0$)
\[
\alpha = \arccos \frac{\cos \beta \cos \alpha}{\cos \delta} = 152.853^{\circ}
\]
The hour angle of Sun when it just risen is 
\[
0 = \sin \phi \sin \delta + \cos \phi \cos \delta \cos \mathrm{HA}
\]
Which gives
\[
\mathrm{HA} = \arccos (-\tan \phi \tan \delta) = 280.894^{\circ}
\]
The street will remain in shadow if the altitude of the Sun does not exceed $h = \arctan 2.5/6 = 22.620^{\circ}$, the hour angle this occur is therefore
\[
\mathrm{HA} = \arccos \left(\frac{\sin h - \sin \phi \sin \delta}{\cos \phi \cos \delta}\right) = 290.690^\circ
\]
Therefore the time that the whole street remained shadowed is
\[
\Delta t = \frac{\Delta \mathrm{HA}}{15^\circ/\mathrm{h}} = 0.653 \mathrm{h} = 39.2 \mathrm{min}
\]
\end{document}